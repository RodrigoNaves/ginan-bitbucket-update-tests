\chapter{Ionosphere Modelling}
\label{ch:ionosphere_modelling}
%
%\begin{fullwidth}
\textit{Ionospheric delay} is the most important nuisance parameter on GNSS processing. 
GNSS processing algorithm needs to account for it by estimating, correcting or cancelling its effects. 
The objective of the project is to develop Ionospheric modelling modules to aid in GNSS data processing. 
he modules will allow end user position algorithms correct the effect of ionosphere delay. 
The modules also allow the network processing more effectively estimate the ionospheric delays.
\\
%
Two potential ionosphere models have been implemented to date. 
Both are dual-layer models, each layer containing an ionosphere density map. 
The maps are represented by spherical cap harmonics in one case and 3rd order b-splines in another. 
The effectiveness of the maps has been tested using measurements from 60 Australian stations distributed. 
The geometry free combination of carrier smoothed pseudorange have been used as ionosphere delay measurements. 
The spherical cap harmonics tested so far can represent the ionospheric delays with an accuracy of about 2-3 TECu, slightly better than the model used for IONEX (single layer, grid based map), but not enough for the 0.2 TECu accuracy needed for PPP. 
The specific parameters (number of layers, map resolution, layer heights and order of the base function) of each model will be refined to improve model accuracy.
\\
%
%\end{fullwidth}

%\section{Ionosphere-Free Observations}
