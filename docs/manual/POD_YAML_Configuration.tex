\chapter{POD YAML Configuration}
\label{ch:pod_yaml_configuration}
\section{Procesing mode:}
%
In this section only one of the following options listed below can be set to true, the remainder must be set to false.\\
\begin{table}[h!]
\begin{tabular}{|p{2.5cm}|p{2.5cm}|p{5cm}|}
	\hline
    Option & Values & Comments \\
    \hline
    %\multicolumn{3}{|p{13cm}|}{In this section only one of the following options listed below can be set to true, the remainder must be set to false.}\\
    %\hline
	fit & true or false & \\
	predict & true or false & \\
	eqm\_int & true or false & \\
	ic\_int & true or false & initial condition integration \\
	\hline
\end{tabular}
\caption{caption for this table}
\label{table:label_name}
\end{table}
%
\section{Time scale}
%
\begin{table}[h!]
\begin{tabular}{|p{2.5cm}|p{2.5cm}|p{5cm}|}
\hline
Option & Values & Comments \\
\hline
Terrestrial (TT) & true or false & \\
Universal (UTC)& true or false & \\
Satellite (GPS)& true or false & \\
Atomic (TAI)& true or false & \\
\hline
\end{tabular}
\caption{caption for this table}
\label{table:label_name}
\end{table}
%
%
\section{Initial Conditions (IC)}
\subsection{IC input format}
 one is true, the other is false. If icf selected the ic\_filename value specifies the path to the file.
\begin{table}[h!]
\begin{tabular}{|p{2.5cm}|p{2.5cm}|p{5cm}|}
	\hline
	Option & Values & Comments \\
	\hline
   sp3 & true or false & sp3 format file\\
   icf & true or false & initial conditions file\\
   \hline
\end{tabular}
\caption{Initial Conditions input format options}
\label{table:yaml}
\end{table}
%
\subsection{IC input reference system (ic\_input\_refsys)}
reference system for the initial conditions one is true, the other is false. 
\begin{table}[h!]
	\begin{tabular}{|p{2.5cm}|p{2.5cm}|p{5cm}|}
		\hline
		Option & Values & Comments \\
		\hline
		ITRF & true or false & terestrial\\
		ICRF & true or false & celestial \\
		kepler & true or false & polar form of celestial\\
		\hline
	\end{tabular}
	\caption{Initial Conditions reference system}
	\label{table:yaml}
\end{table}


%
\section{Using an external orbit}
In this section only one of the following options listed below can be set to true, the remainder must be set to false.\\
\begin{table}[h!]
\begin{tabular}{|p{4.5cm}|p{2cm}|p{3.5cm}|}
	\hline
	Option & Values & Comments \\
	\hline
	ext\_orbit\_type\_sp3  & true or false & \\
	ext\_orbit\_type\_interp & true or false & \\
	ext\_orbit\_type\_kepler & true or false & \\
	ext\_orbit\_type\_lagrange & true or false & \\
	ext\_orbit\_type\_position\_sp3 & true or false & \\
    \hline
\end{tabular}
	\caption{External orbit options}
\label{table:yaml}
\end{table}
%
\subsection{External orbit Options}
\begin{table}[h!]
	\begin{tabular}{|p{4cm}|p{2cm}|p{4cm}|}
		\hline
		Option & Values & Comments \\
		\hline
        \textbf{ext\_orbit\_filename} & filename & comparison orbit filename \\
        \textbf{ext\_orbit\_interp\_step} & int & \\ 
        \textbf{ext\_orbit\_interp\_points} & int & see pseudobs descriptions \\
        \textbf{ext\_orbit\_frame} & & as per initial conditions, but applied to the comparison orbit\\ 
        \hline
\end{tabular}
\caption{External orbit options}
\label{table:yaml}
\end{table}
%        
\section{Earth Orientation Parameters}
In this section only one of the following options listed below can be set to true, the remainder must be set to false.\\
\subsection{EOP type}
\begin{table}[h!]
	\begin{tabular}{|p{2.5cm}|p{2.5cm}|p{5cm}|}
		\hline
		Option & Values & Comments \\
		\hline
		\textbf{eop\_soln\_c04} & true or false & C04 is the IERS solution\\
		\textbf{eop\_soln\_rapid} & true or false & Rapid is the rapid/prediction center solution\\
		\textbf{eop\_soln\_igs} & true or false & igs is the ultra-rapid solution using partials. To use this you need both the rapid file and partials file.\\
		\hline
	\end{tabular}
	\caption{Earth Orientation Parameters}
	\label{table:yaml}
\end{table}
%
\subsection{EOP model options}
\begin{table}[h!]
	\begin{tabular}{|p{2.5cm}|p{2.5cm}|p{5cm}|}
		\hline
		Option & Values & Comments \\
		\hline
        \textbf{eop\_soln\_interp\_points} & int & number of data points to be used in an eop interpolation (at least 2!)\\
        \textbf{iau\_model\_2000} & true or false & \\
        \textbf{iau\_model\_2006} & true or false & \\
		\hline
	\end{tabular}
	\caption{EOP model options}
	\label{table:yaml}
\end{table}

\subsection{EOP output options}
\begin{table}[h!]
	\begin{tabular}{|p{2.5cm}|p{2.5cm}|p{5cm}|}
		\hline
		Option & Values & Comments \\
		\hline
        \textbf{eop\_soln\_co4\_file} & filename & c04 solution file\\
        \textbf{eop\_soln\_rapid\_file} & filename & rapid solution file\\
        \textbf{erp\_soln\_igs\_file} & filename & partials solution file\\
		\hline
	\end{tabular}
	\caption{EOP output options}
	\label{table:yaml}
\end{table}

\section{Input files}
\begin{table}[h!]
	\begin{tabular}{|p{2.5cm}|p{2.5cm}|p{5cm}|}
		\hline
		Option & Values & Comments \\
		\hline
        \textbf{DE\_fname\_header} & filename &  Emphemeris header file \\
        \textbf{DE\_fanme\_data} & filename & Emphemeris data file \\
        \textbf{ocean\_tides\_model\_file} & filename & \\
        \textbf{leapsec\_filename} & filename & leapseconds to be added\\
        \textbf{satsinex\_filename} & filename & sinex file with satellite meta-data\\
        \textbf{gravity\_model\_file} & filename & \\
        \textbf{pseudobs\_orbit\_filename} & filename & \emph{path to the observations file}\\
		\hline
	\end{tabular}
	\caption{Input files}
	\label{table:label_name}
\end{table}

\section{Output options}
\begin{table}[h!]
	\begin{tabular}{|p{2.5cm}|p{2.5cm}|p{5cm}|}
		\hline
		Option & Values & Comments \\
		\hline
        \textbf{sp3\_velocity} & true or false & if you wish to write out the velocities for comparison \\
        \textbf{partials\_velocity}: true or false & if you wish to write velocity vector partials to the output file\\
		\hline
	\end{tabular}
	\caption{Input files}
	\label{table:label_name}
\end{table}

\section{General Options}
\begin{table}[h!]
	\begin{tabular}{|p{2.5cm}|p{2.5cm}|p{5cm}|}
		\hline
		Option & Values & Comments \\
		\hline
        \textbf{estimator\_iterations} & int & integrate this number of times, using the generated initial conditions from the previous run as a start point\\
        \textbf{veq\_integration}& true or false &   pod mode overrides it anyway. Ignore.\\
        \textbf{veq\_refsys}& &  as per ic option\\
        \hline
        \textbf{ITRF} & true or false & reference\_frame\\
        \textbf{ICRF} & true or false &  one must be true\\       
        \hline
        \textbf{pseudobs\_interp\_step}& int & \emph{stepsize of the observations (in seconds)}\\
        \textbf{pseudobs\_interp\_points}& int & \emph{number of observations used in any interpolation (at least 2!)}\\
        \textbf{orbit\_arc\_determination}& int & \emph{number of hours to integrate}\\
        \textbf{orbit\_arc\_prediction}& int & \emph{number of hours to predict at end of orbit arc}\\
        orbit\_arc\_backwards& & number of hours to check before start of orbit arc\\
        ext\_orbit\_enabled&  true or false & \\ 
		\hline
	\end{tabular}
	\caption{general options}
	\label{table:label_name}
\end{table}

\section{Integration Step}
\begin{table}[h!]
	\begin{tabular}{|p{2.5cm}|p{2.5cm}|p{5cm}|}
		\hline
		Option & Values & Comments \\
		\hline
		RK4 & true or false &  Do not use RK4 for veq as it is not implemented\\ 
		RKN7 & true or false & only one can be true\\ 
		RK8  & true or false & \\
		\hline
		\textbf{integrator\_step} & int & stepsize in seconds \\
		\hline
	\end{tabular}
	\caption{caption for this table}
	\label{table:label_name}
\end{table}
%
\section{Gravitational Forces}
%
\subsubsection{gravity\_model: exactly one must be true}
\begin{table}[h!]
	\begin{tabular}{|p{2.5cm}|p{2.5cm}|p{5cm}|}
		\hline
		Option & Values & Comments \\
		\hline
		enabled & true or false & \\
		central\_force & true or false &  \\ 
		static & true or false &  \\
		time variable & true or false &  \\
		geopotential & true or false &  \\
		\hline
	\end{tabular}
	\caption{caption for this table}
	\label{table:label_name}
\end{table}
%

\subsubsection{Gravity Field}
\begin{table}[h!]
	\begin{tabular}{|p{2.5cm}|p{2.5cm}|p{5cm}|}
		\hline
		Option & Values & Comments \\
		\hline
        enabled & true or false & and if true:\\
        gravity\_degree\_max &  & maximum model terms in spherical harmonic expansion \\ 
        timevar\_degree\_max &  & maximum time variable model terms in spherical harmonic expansion \\
        \hline
	\end{tabular}
	\caption{caption for this table}
	\label{table:label_name}
\end{table}
%
\subsection{planetary\_perturbations:}
\begin{table}[h!]
	\begin{tabular}{|p{2.5cm}|p{2.5cm}|p{5cm}|}
		\hline
		Option & Values & Comments \\
		\hline
	    enabled & true/false & Uses the emphemeris \\ %TODO specify the file
		\hline
	\end{tabular}
	\caption{planetary\_perturbations}
	\label{table:label_name}
\end{table}  
%
\subsection{tidal\_effects:}
\begin{table}[h!]
	\begin{tabular}{|p{2.5cm}|p{2.5cm}|p{5cm}|}
	\hline
	Option & Values & Comments \\
	\hline
	solid\_tides\_nonfreq & True or False & frequency independent Solid Earth Tides \\
	solid\_tides\_freq & True or False & frequency dependent Solid Earth Tides \\
	ocean\_tides & True or False & uses the ocean tides file \\
	solid\_earth\_pole\_tides & True or False & tide induced earth spin rotation not about the centre of the ellipsoid \\
	ocean\_pole\_tide & True or False & ocean response to the above \\
	ocean\_tides\_degree\_max & True or False & maximum model term in spherical harmonic expansion \\
	\hline
\end{tabular}
\caption{caption for this table}
\label{table:label_name}
\end{table}
%
\subsection{stochastic pulse:}
\begin{table}[h!]
	\begin{tabular}{|p{2.5cm}|p{2.5cm}|p{5cm}|}
		\hline
		Option & Values & Comments \\
		\hline
		enabled & true or false & then if true: \\
		\hline
		epoch\_number & int & number of epochs to apply pulses each day \\
		offset & int & seconds until the first pulse of the day \\
		interval & int & seconds between each pulse (after the first)\\
		\hline
		pulses in X/Y/Z (celestial) & true or false & \\
		pulses in R/T/N (terrestrial) & true or false & Don't mix them and only directions for the selected frame can be set true\\
		\hline
	\end{tabular}
	\caption{caption for this table}
	\label{table:label_name}
\end{table}

%
\section{relativistic\_effects:}
\begin{table}[h!]
	\begin{tabular}{|p{2.5cm}|p{2.5cm}|p{5cm}|}
	\hline
	Option & Values & Comments \\
	\hline
	enables & true or false & Lens\_Thirring, SchwarzChild and deSitter effects, there are no means to separate these effects currently. The Lens Thirring effect is calculated but subsequently ignored in the POD.\\
	\hline
\end{tabular}
\caption{relativistic\_effects}
\label{table:label_name}
\end{table}
%
\section{non\_gravitational\_effects:}
\subsection{Models to be applied:}
\begin{table}[h!]
	\begin{tabular}{|p{2.5cm}|p{2.5cm}|p{5cm}|}
	\hline
	Option & Values & Comments \\
	\hline
    solar\_radiation: & true or false & radiation push from the sun \\
    earth\_radiation:  & true or false & radiation push from the earth \\
    antenna\_thrust:   & true or false & reverse thrust from antenna radiation \\
	\hline
\end{tabular}
\caption{caption for this table}
\label{table:label_name}
\end{table}
%
%
%
\subsection{Apriori Solar radiation models}   
\begin{table}[h!]
	\begin{tabular}{|p{2.5cm}|p{2.5cm}|p{5cm}|}
		\hline
		Option & Values & Comments \\
		\hline
		none & true or false & \\
		cannonball & true or false & see \nameref{sec:cannonball_srp}\\
		simple\_boxwing & true or false & \\
		full\_boxwing & true or false & \\
		\hline
	\end{tabular}
	\caption{srp\_apriori\_model; Exactly one option must be true}
	\label{table:label_name}
\end{table}
%
\subsection{Estimated Solar radiation models}
\begin{table}[h!]
	\begin{tabular}{|p{2.5cm}|p{2.5cm}|p{5cm}|}
		\hline
		Option & Values & Comments \\
		\hline
		ECOM1  & true or false & \\ 
		ECOM2  & true or false & \\
		hybrid & true or false &  mix of ECOM1 and ECOM2\\
		SBOXW  & true or  false & Simple box wing model\\
		Empirical models. & true or false & Empirical is independent of the other four\\
		\hline
	\end{tabular}
	\caption{srp\_modes}
	\label{table:label_name}
\end{table}
%
\subsection{Empirical parameters}
\begin{table}[h!]
	\begin{tabular}{|p{2.5cm}|p{2.5cm}|p{5cm}|}
	\hline
	Option & Values & Comments \\
	\hline
	ecom\_d\_bias & true or false & \\ 
	ecom\_y\_bias & true or false & \\ 
	ecom\_b\_bias & true or false & \\ 
	ecom\_d\_cpr & true or false & (only ECOM1\/hybrid)\\ 
	ecom\_y\_cpr & true or false & (only ECOM1\/hybrid)\\ 
	ecom\_b\_cpr & true or false & \\
	ecom\_d\_2\_cpr & true or false &  (only ECOM2\/hybrid)\\ 
	ecom\_d\_4\_cpr & true or false & (only ECOM2\/hybrid) \\
	emp\_r\_bias & true or false & \\
	emp\_t\_bias & true or false & \\ 
	emp\_n\_bias  & true or false & \\
	emp\_r\_cpr  & true or false & \\
	emp\_t\_cpr  & true or false & \\
	emp\_n\_cpr & true or false & \\
	cpr\_count  & int & empirical cpr count \\
	\hline
\end{tabular}
\caption{Configuration options for solar radiation pressure models}
\label{table:label_name}
\end{table}

NB EQM and VEQ srp parameters MUST be identical. May move into pod\_options in future.
overrides are not implemented yet. Ignore for now. We imagine overrides at the system, block (sat type) and individual satellite level
