% This is a modified version of the tufte-latex book example in which the title page and the contents page resemble Tufte's VDQI book, using Kevin Godby's code from this thread at https://groups.google.com/forum/#!topic/tufte-latex/ujdzrktC1BQ.
\documentclass[nohyper,nobib]{tufte-book}
\usepackage{nameref}
%\hypersetup{colorlinks}% uncomment this line if you prefer colored hyperlinks (e.g., for onscreen viewing)

% \usepackage{hyphenat}
\usepackage{url}
\usepackage[backend=biber, natbib=true, style=numeric]{biblatex}

% see https://www.overleaf.com/learn/latex/Glossaries
% make glossaries must be written before the first glossary entry
\usepackage[utf8]{inputenc}
\usepackage[acronym, toc]{glossaries}

\loadglsentries{glossary.tex}
\makeglossaries

\usepackage{listings}
%\usepackage[backend=biber, natbib=true, style=numeric]{biblatex}
%\addbibresource{sample-handout.bib}
\usepackage{xargs}
\renewcommandx{\cite}[3][1={0pt},2={}]{\sidenote[][#1]{\fullcite[#2]{#3}}}

%%
% Book metadata
\title{Ginarn}
\date{Version 0.0}
\author[The Ginarn team]{
Sebastien Allegyer,
Rupert Brown,
John Donovan,
Ken Harima,
Aaron Hammond,
Lavanya Kumarappan,
Tao Li,
Ronald Maj,
Bogdan Matviichuk,
Simon McClusky,
Michael Moore,
Thomas Papanikolaou,
Tzupang Tseng
}
\publisher{Geoscience Australia and Frontier-SI}

%%
% If they're installed, use Bergamo and Chantilly from www.fontsite.com.
% They're clones of Bembo and Gill Sans, respectively.
\IfFileExists{bergamo.sty}{\usepackage[osf]{bergamo}}{}% Bembo
\IfFileExists{chantill.sty}{\usepackage{chantill}}{}% Gill Sans

%\usepackage{microtype}

%%
% Just some sample text
\usepackage{lipsum}

%%
% For nicely typeset tabular material
\usepackage{booktabs}

%%
% For graphics / images
\usepackage{graphicx}
\setkeys{Gin}{width=\linewidth,totalheight=\textheight,keepaspectratio}
\graphicspath{{graphics/}}

% The fancyvrb package lets us customize the formatting of verbatim
% environments.  We use a slightly smaller font.
\usepackage{fancyvrb}
\fvset{fontsize=\normalsize}

\usepackage{tikz}
%\usetikzlibrary{external}
%\tikzexternalize[prefix=tikz/]
\usetikzlibrary{shapes.geometric, arrows}
%%
% Prints argument within hanging parentheses (i.e., parentheses that take
% up no horizontal space).  Useful in tabular environments.
\newcommand{\hangp}[1]{\makebox[0pt][r]{(}#1\makebox[0pt][l]{)}}

%%
% Prints an asterisk that takes up no horizontal space.
% Useful in tabular environments.
\newcommand{\hangstar}{\makebox[0pt][l]{*}}

%%
% Prints a trailing space in a smart way.
\usepackage{xspace}

\usepackage{amsmath}
\usepackage[colorlinks]{hyperref}
%%
% Some shortcuts for Tufte's book titles.  The lowercase commands will
% produce the initials of the book title in italics.  The all-caps commands
% will print out the full title of the book in italics.
%\newcommand{\vdqi}{\textit{VDQI}\xspace}
%\newcommand{\ei}{\textit{EI}\xspace}
%\newcommand{\ve}{\textit{VE}\xspace}
%\newcommand{\be}{\textit{BE}\xspace}
%\newcommand{\VDQI}{\textit{The Visual Display of Quantitative Information}\xspace}
%\newcommand{\EI}{\textit{Envisioning Information}\xspace}
%\newcommand{\VE}{\textit{Visual Explanations}\xspace}
%\newcommand{\BE}{\textit{Beautiful Evidence}\xspace}

\newcommand{\TL}{Tufte-\LaTeX\xspace}

% Prints the month name (e.g., January) and the year (e.g., 2008)
\newcommand{\monthyear}{%
  \ifcase\month\or January\or February\or March\or April\or May\or June\or
  July\or August\or September\or October\or November\or
  December\fi\space\number\year
}


% Prints an epigraph and speaker in sans serif, all-caps type.
\newcommand{\openepigraph}[2]{%
  %\sffamily\fontsize{14}{16}\selectfont
  \begin{fullwidth}
  \sffamily\large
  \begin{doublespace}
  \noindent\allcaps{#1}\\% epigraph
  \noindent\allcaps{#2}% author
  \end{doublespace}
  \end{fullwidth}
}

% Inserts a blank page
\newcommand{\blankpage}{\newpage\hbox{}\thispagestyle{empty}\newpage}

\usepackage{units}

% Typesets the font size, leading, and measure in the form of 10/12x26 pc.
\newcommand{\measure}[3]{#1/#2$\times$\unit[#3]{pc}}

% Macros for typesetting the documentation
\newcommand{\hlred}[1]{\textcolor{Maroon}{#1}}% prints in red
\newcommand{\hangleft}[1]{\makebox[0pt][r]{#1}}
\newcommand{\hairsp}{\hspace{1pt}}% hair space
\newcommand{\hquad}{\hskip0.5em\relax}% half quad space
\newcommand{\TODO}{\textcolor{red}{\bf TODO!}\xspace}
\newcommand{\ie}{\textit{i.\hairsp{}e.}\xspace}
\newcommand{\eg}{\textit{e.\hairsp{}g.}\xspace}
\newcommand{\na}{\quad--}% used in tables for N/A cells
\providecommand{\XeLaTeX}{X\lower.5ex\hbox{\kern-0.15em\reflectbox{E}}\kern-0.1em\LaTeX}
\newcommand{\tXeLaTeX}{\XeLaTeX\index{XeLaTeX@\protect\XeLaTeX}}
% \index{\texttt{\textbackslash xyz}@\hangleft{\texttt{\textbackslash}}\texttt{xyz}}
\newcommand{\tuftebs}{\symbol{'134}}% a backslash in tt type in OT1/T1
\newcommand{\doccmdnoindex}[2][]{\texttt{\tuftebs#2}}% command name -- adds backslash automatically (and doesn't add cmd to the index)
\newcommand{\doccmddef}[2][]{%
  \hlred{\texttt{\tuftebs#2}}\label{cmd:#2}%
  \ifthenelse{\isempty{#1}}%
    {% add the command to the index
      \index{#2 command@\protect\hangleft{\texttt{\tuftebs}}\texttt{#2}}% command name
    }%
    {% add the command and package to the index
      \index{#2 command@\protect\hangleft{\texttt{\tuftebs}}\texttt{#2} (\texttt{#1} package)}% command name
      \index{#1 package@\texttt{#1} package}\index{packages!#1@\texttt{#1}}% package name
    }%
}% command name -- adds backslash automatically
\newcommand{\doccmd}[2][]{%
  \texttt{\tuftebs#2}%
  \ifthenelse{\isempty{#1}}%
    {% add the command to the index
      \index{#2 command@\protect\hangleft{\texttt{\tuftebs}}\texttt{#2}}% command name
    }%
    {% add the command and package to the index
      \index{#2 command@\protect\hangleft{\texttt{\tuftebs}}\texttt{#2} (\texttt{#1} package)}% command name
      \index{#1 package@\texttt{#1} package}\index{packages!#1@\texttt{#1}}% package name
    }%
}% command name -- adds backslash automatically
\newcommand{\docopt}[1]{\ensuremath{\langle}\textrm{\textit{#1}}\ensuremath{\rangle}}% optional command argument
\newcommand{\docarg}[1]{\textrm{\textit{#1}}}% (required) command argument
\newenvironment{docspec}{\begin{quotation}\ttfamily\parskip0pt\parindent0pt\ignorespaces}{\end{quotation}}% command specification environment
\newcommand{\docenv}[1]{\texttt{#1}\index{#1 environment@\texttt{#1} environment}\index{environments!#1@\texttt{#1}}}% environment name
\newcommand{\docenvdef}[1]{\hlred{\texttt{#1}}\label{env:#1}\index{#1 environment@\texttt{#1} environment}\index{environments!#1@\texttt{#1}}}% environment name
\newcommand{\docpkg}[1]{\texttt{#1}\index{#1 package@\texttt{#1} package}\index{packages!#1@\texttt{#1}}}% package name
\newcommand{\doccls}[1]{\texttt{#1}}% document class name
\newcommand{\docclsopt}[1]{\texttt{#1}\index{#1 class option@\texttt{#1} class option}\index{class options!#1@\texttt{#1}}}% document class option name
\newcommand{\docclsoptdef}[1]{\hlred{\texttt{#1}}\label{clsopt:#1}\index{#1 class option@\texttt{#1} class option}\index{class options!#1@\texttt{#1}}}% document class option name defined
\newcommand{\docmsg}[2]{\bigskip\begin{fullwidth}\noindent\ttfamily#1\end{fullwidth}\medskip\par\noindent#2}
\newcommand{\docfilehook}[2]{\texttt{#1}\index{file hooks!#2}\index{#1@\texttt{#1}}}
\newcommand{\doccounter}[1]{\texttt{#1}\index{#1 counter@\texttt{#1} counter}}

% Generates the index
\usepackage{makeidx}
\makeindex

%%%% Kevin Godny's code for title page and contents from https://groups.google.com/forum/#!topic/tufte-latex/ujdzrktC1BQ
\makeatletter
\renewcommand{\maketitlepage}{%
\begingroup%
\setlength{\parindent}{0pt}

{\fontsize{24}{24}\selectfont\textit{\@author}\par}

\vspace{1.75in}{\fontsize{36}{54}\selectfont\@title\par}

\vspace{0.5in}{\fontsize{14}{14}\selectfont\textsf{\smallcaps{\@date}}\par}

\vfill{\fontsize{14}{14}\selectfont\textit{\@publisher}\par}

\thispagestyle{empty}
\endgroup
}
\makeatother

\titlecontents{part}%
    [0pt]% distance from left margin
    {\addvspace{0.25\baselineskip}}% above (global formatting of entry)
    {\allcaps{Part~\thecontentslabel}\allcaps}% before w/ label (label = ``Part I'')
    {\allcaps{Part~\thecontentslabel}\allcaps}% before w/o label
    {}% filler and page (leaders and page num)
    [\vspace*{0.5\baselineskip}]% after

\titlecontents{chapter}%
    [4em]% distance from left margin
    {}% above (global formatting of entry)
    {\contentslabel{2em}\textit}% before w/ label (label = ``Chapter 1'')
    {\hspace{0em}\textit}% before w/o label
    {\qquad\thecontentspage}% filler and page (leaders and page num)
    [\vspace*{0.5\baselineskip}]% after
%%%% End additional code by Kevin Godby

\begin{document}

% Front matter
\frontmatter

% r.3 full title page
\maketitle


% v.4 copyright page
\newpage
\begin{fullwidth}
~\vfill
\thispagestyle{empty}
\setlength{\parindent}{0pt}
\setlength{\parskip}{\baselineskip}
Copyright \copyright\ \the\year\ \thanklessauthor

\par\smallcaps{Published by \thanklesspublisher}

\par\smallcaps{tufte-latex.googlecode.com}

\par Licensed under the Apache License, Version 2.0 (the ``License''); you may not
use this file except in compliance with the License. You may obtain a copy
of the License at \url{http://www.apache.org/licenses/LICENSE-2.0}. Unless
required by applicable law or agreed to in writing, software distributed
under the License is distributed on an \smallcaps{``AS IS'' BASIS, WITHOUT
WARRANTIES OR CONDITIONS OF ANY KIND}, either express or implied. See the
License for the specific language governing permissions and limitations
under the License.\index{license}

\par\textit{First printing, \monthyear}
\end{fullwidth}
% r.5 contents
\tableofcontents
%
\chapter{Welcome}
\label{ch:Ginan}

% We would like to thank the Wardaman people for permission in the use of their word Ginan. 

Ginan is the fifth-brightest star in the Southern Cross (Epsilon Crucis) . It represents a red dilly-bag filled with special songs of knowledge.
Indigenous Australians often used songs to convey and to pass on knowledge to others, song were also often used as a way to navigate the country.

We hope that you find this software tool kit will convey our understanding on how to process GNSS signals and will also help you to navigate the country!

the story Ginan was found by Mulugurnden (the crayfish), who brought the red flying foxes from the underworld to the sky. The bats flew up the track of the Milky Way and traded the spiritual song to Guyaru, the Night Owl (the star Sirius). The bats fly through the constellation Scorpius on their way to the Southern Cross, trading songs as they go.

The song informs the people about initiation, which is managed by the stars in Scorpius and related to Larawag (who ensures the appropriate personnel are present for the final stages of the ceremony).

The brownish-red colour of the dilly bag is represented by the colour of Epsilon Crucis, which is an orange giant that lies 228 light years away.
\newthought{This manual} is divided into three major sections: the preliminary matter which gives a quick overview on how to install and use the software with examples. 
The next part contains the background theory on the models that have been implemented into Ginarn. 
The back matter contains information about configuration files, and file formats used by the software.
This manual is far from complete, and we will be adding to it as we continue our development work on Ginarn. Our intention is to provide examples on how to use the tool kit , and then to provide the background theory as we can.
%
\listoffigures
%
\listoftables
%
% r.9 introduction
\cleardoublepage
%\part{Overview}
%
\chapter{Introduction to Ginan GNSS Processing Toolkit}
\label{ch:introduction}
%


Ginan is a collection of source code that is currently made up two distinct software repositories, the POD \sidenote[][]{https://bitbucket.org/geoscienceaustralia/ pod/wiki/Home} and the PEA\sidenote[][]{https://bitbucket.org/geoscienceaustralia/ pea/wiki/Home}.
Using the POD and PEA together will allow you to estimate your own satellite orbits from a global tracking network.
\newthought{The POD} (precise orbit determination) contains all of the source code needed to determine a GNSS satellite's orbit. You can establish the initial conditions of an orbit from a broadcast ephemeris file, or from an IGS SP3 file. It can then estimate it's own orbital trajectory based upon the models specified in configuration files, and output an SP3 file, or provide a partial files which can then be updated from a tracking network. 



\newthought{The PEA} (parameter estimation algorithm) takes raw observations in RINEX format or in RTCM format, to estimates the parameters you are interested in. You can run it a single user mode, taking in orbit and clocks supplied by real-time streams to SP3 files obtained from the IGS to estimate your own position in static and kinematic mode. 
You can also run the PEA in a network mode, and take in a global network of observations to determine your own orbits and satellite clocks to support your application.

\newthought{The software} is aimed at supporting Australia's implementation of a national positioning infrastructure that supports the objective of 'instantaneous GNSS positoning anywhere, anytime, with the highest possible accuracy and the highest possible integrity.

\newthought{Carrier phase ambiguity}. The underlying signals transmitted by the Global Navigation Satellite Systems (GNSS) can be considered as waves, just like repeating sine waves from high school mathematics. Measurements of these waves are referred to as carrier phase observations, and they are used to provide the precise distance, with mm precision and accuracy, between the orbiting satellites and user’s receiver that are subsequently used to compute position. However, a complicating factor is that carrier phase observations have an ambiguous component where the total whole number of waves, or integer cycles, between the satellite and the user’s receiver cannot be measured, only the fractional part. The unknown number of integer cycles is called the carrier phase ambiguity. Fortunately, the ambiguities can be estimated, and the mathematical and statistical solution to this problem is known as integer ambiguity estimation. While there is a long history of research in this area, which has largely focused on GPS applications, the most optimal solution to this problem when simultaneously combining data from all the GNSS remains unresolved.

\newthought{Atmosphere delay of GNSS signals}. The Earth is surrounded by layers of gases held by Earth's gravity. Signals, such as those transmitted by GNSS, propagated from space are delayed as they pass through the atmosphere. In the troposphere, the region from the Earth’s surface to approximately 20 km altitude, the delay is proportional to temperature, pressure and humidity. The ionosphere, the region from 50-1000 km altitude, causes delay as a function of the frequency of the signal. The composition of both the troposphere and ionosphere vary both in space and time, and this variability currently limits the accuracy, speed and reliability of positioning. But it’s not all bad news, and like a CAT scan in medical science, the new GNSS signals and satellites can potentially be combined to provide a more complete three dimensional picture of the atmospheric delay as a function of time. Models that more completely remove the nuisance atmospheric signals will lead to improved accuracy, speed and reliability of positioning.

\newthought{Precise Point Positioning (PPP) and Real Time Kinematic (RTK)}. Conventional positioning technologies almost exclusively use a technique called Real Time Kinematic (RTK). The RTK technique takes information from nearby Continuously Operating Reference Stations (CORS) to generate corrections for measurements made by users. One of the most important of these is the carrier phase ambiguity correction. The ability to correctly determine carrier phase ambiguities with RTK is determined by many factors, such as the distance between the reference stations and the user and also atmospheric effects. Consequently, RTK relies on relatively dense CORS networks with a typical spacing of 30 to 70 km. An alternative to RTK is Precise Point Positioning (PPP). The PPP technique, rather than directly using measurements from nearby reference stations, uses global satellite orbit and clock information such as that provided by the International GNSS Service. The major advantage of PPP is that it doesn’t require a dense CORS network nearby the user’s location just access to global products. Unfortunately, the PPP technique can have difficulties in resolving carrier phase ambiguities in real time, but additional research focused on greater exploitation of multi-GNSS data and more regional approaches may in the future overcome this limitation.
%
\chapter{Installation}
\label{ch:installation}


\section{To Install} 

In this section we will describe how to install the PEA and POD from source. An alternative option to installing all of the dependencies and the source code would be to use one of our docker images available from Docker Hub. Instructions on how to do this are in (see Docker).

\subsection{PEA}


\newthought{Dependencies} the following packages need to be installed with the minimum versions as shown below. This guide will outline the preferred method of installation.

CMAKE  > 3.0 requires openssl-devel to be installed (requires openssl-devel)
YAML   > 0.6
Boost  > 1.70
gcc    > 4.1
Eigen3
Build
To build the PEA Precise Estimation Algorithm...

We suggest using the following directory structure when installing the Ginarn toolkit. It will be created by following this guide.

%/data/
%└── acs/
%    ├── pea/
%    └── pod/

The following is an example procedure to install the dependencies necessary to run the pea on a base ubuntu linux distribution

Update the base operating system:

\begin{lstlisting}[language=bash]
$ sudo apt update
$ sudo apt upgrade
\end{lstlisting}

Install base utilities gcc, gfortran, git, openssl, blas, lapack, etc
\begin{lstlisting}[language=bash]
$ sudo apt install -y git gobjc gobjc++ gfortran libopenblas-dev openssl curl net-tools openssh-server cmake make \
liblapack-dev gzip vim libssl1.0-dev python3-cartopy python3-scipy python3-matplotlib python3-mpltoolkits.basemap
Create a temporary directory structure to make the dependencies in:
$ sudo mkdir -p /data/tmp
$ cd /data/tmp
\end{lstlisting}

\newthought{YAML}
We are using the YAML library to parse the configuration files used to run many of the programs found in this library (https://github.com/jbeder/yaml-cpp). Here is an example of how we have installed the yaml library from source:
\begin{lstlisting}[language=bash]
$ cd /data/tmp
$ sudo git clone https://github.com/jbeder/yaml-cpp.git
$ cd yaml-cpp
$ sudo mkdir cmake-build
$ cd cmake-build
$ sudo cmake .. -DCMAKE\_INSTALL\_PREFIX=/usr/local/ -DYAML\_CPP\_BUILD\_TESTS=OFF
$ sudo make install yaml-cpp
$ cd ../..
$ sudo rm -fr yaml-cpp
\end{lstlisting}

\newthought{Boost}
We rely on a number of the utilities provided by boost (https://www.boost.org/), such as their time and logging libraries.
\begin{lstlisting}[language=bash]
$ cd /data/tmp/
$ sudo wget -c https://dl.bintray.com/boostorg/release/1.73.0/source/boost_1_73_0.tar.gz
$ sudo gunzip boost_1_73_0.tar.gz
$ sudo tar xvf boost_1_73_0.tar
$ cd boost_1_73_0/
$ sudo ./bootstrap.sh
$ sudo ./b2 install
$ cd ..
$ sudo rm -fr boost_1_73_0/ boost_1_73_0.tar

\end{lstlisting}

\newthought{Eigen3} is used for performing matrix calculations, and has a very nice API.
\begin{lstlisting}[language=bash]
$ cd /data/tmp/
$ sudo git clone https://gitlab.com/libeigen/eigen.git
$ cd eigen
$ sudo mkdir cmake-build
$ cd cmake-build
$ sudo cmake ..
$ sudo make install
$ cd ../..
$ sudo rm -rf eigen
Installing PEA
PEA Executable
$ cd /data/acs/
\end{lstlisting}
Clone the repository via https:

\begin{lstlisting}[language=bash]
$ git clone https://bitbucket.org/geoscienceaustralia/pea.git
\end{lstlisting}
You should now have...

%pea
%├── INSTALL.md
%├── LICENSE.md
%├── README.md
%├── aws/                - for automated builds in aws
%├── config/
%│   ├── Ex00-UnitTest.yaml
%│   ├── Ex01-PPP.yaml
%│   ├── Ex02-Network.yaml
%│   ├── Ex03-Network_Orbits.yaml
%│   ├── Ex04-Ionosphere.yaml
%│   ├── Ex05-Realtime.yaml
%│   ├── iontest_20115w.yaml
%│   └── PPP-iontest.yaml
%├── cpp/
%│   ├── CMakeLists.txt
%│   ├── cmake           - files to help cmake find dependencies
%│   ├── docs            - automatic code documentation configuration
%│   └── src/
%│       ├── 3rdparty/   - see ACKNOWLEDGEMENTS in README.md
%│       ├── common/     - libraries used by the pea
%│       ├── iono/       - routines for ionosphere modelling
%│       ├── pea/        - main for `pea`
%│       └── rtklib/     - subset of modified routines from RTKlib see ACKNOWLEDGEMENTS in README.md
%└── python
%    ├── config
%    ├── README.md
%    └── source
%        ├── download_examples.py
%        ├── install_examples.py
%        └── other helper programs
Prepare a directory to build in, its better practise to keep this separated from the source code.
\begin{lstlisting}[language=bash]
$ cd pea/cpp
$ mkdir -p build
$ cd build
\end{lstlisting}

Run cmake to find the build dependencies and create the makefile. You have the choice of adding in a couple of compile options. 
Using the flag -DENABLE\_MONGODB=TRUE will set up the mongodb utilities, adding the flag -DENABLE\_OPTIMISATION=TRUE will set up the compiler to run optimisation O3. 
Enabling the optimisation flag will speed up the processing by a factor of 3, however this can lead to compile errors depending on the system you are compiling on, if this happens remove this option.

\begin{lstlisting}[language=bash]
$ cmake ..
or to enable MONGODB utilities
$ cmake -DENABLE_MONGODB=TRUE ..
and to enable Optimisation
$ cmake -DENABLE_MONGODB=TRUE -DENABLE_OPTIMISATION=TRUE ..
\end{lstlisting} 
Now build the pea

\begin{lstlisting}[language=bash]
$ cmake --build $PWD --target pea
\end{lstlisting}

To change to build location substitute your preferred destination for \$PWD , e.g /usr/local/bin

Alternatively to the command above you can make the code in parallel using:
\begin{lstlisting}[language=bash]
$ make -j 5 all
\end{lstlisting}

where the -j flag controls how many jobs can be run at the same time.

Check to see if you can execute the pea:
\begin{lstlisting}[language=bash]
$ ./pea    
\end{lstlisting}

and you should see something similar to:
\begin{lstlisting}[language=bash]
PEA starting...
Options:
  --help                      Help
  --verbose                   More output
  --quiet                     Less output
  --config arg                Configuration file
  --trace_level arg           Trace level
  --antenna arg               ANTEX file
  --navigation arg            Navigation file
  --sinex arg                 SINEX file
  --sp3file arg               Orbit (SP3) file
  --clkfile arg               Clock (CLK) file
  --dcbfile arg               Code Bias (DCB) file
  --ionfile arg               Ionosphere (IONEX) file
  --podfile arg               Orbits (POD) file
  --blqfile arg               BLQ (Ocean loading) file
  --erpfile arg               ERP file
  --elevation_mask arg        Elevation Mask
  --max_epochs arg            Maximum Epochs
  --epoch_interval arg        Epoch Interval
  --rnx arg                   RINEX station file
  --root_input_dir arg        Directory containg the input data
  --root_output_directory arg Output directory
  --start_epoch arg           Start date/time
  --end_epoch arg             Stop date/time
  --dump-config-only          Dump the configuration and exit
PEA finished
\end{lstlisting}

\newthought{The documentation} for the pea can be generated similarly using doxygen if it is installed.

\begin{lstlisting}[language=bash]
$ sudo apt-get install doxygen
$ cd pea/cpp/build
$ make doc_doxygen
\end{lstlisting}
The docs can then be found at doc\_doxygen/html/index.html

\subsection{POD from source}

%The ACS Version 0.0.1 beta release supports:

%The POD
%Directory Structure
%pod/
%├── LICENSE.md
%├── INSTALL.md
%├── README.md
%├── src/
%├── bin/  (created)
%├── lib/  (created)
%├── config/
%├── tables/
%├── scripts/

Dependencies

The open basic linear algebra library (Openblas.x86\_64,liblas-libs.x86\_64) (You may need to run the command ln -s /usr/lib64/libopenblas.so.3 /usr/lib64/libopenblas.so)
A working C compiler (gcc will do), a working C++ compiler (gcc-g++ will do) and a fortran compiler (we have used gfortran)
Cmake (from cmake.org) at least version 2.8
If the flags set in CMakeLists.txt do not work with your compiler please remove/replace the ones that don't

Build
To build the POD ...
\begin{lstlisting}[language=bash]
$ cd pod
$ mkdir build
$ cd build
$ cmake3 .. 
$ make >make.out 2>make.err
$ less make.err (to verify everything was built correctly)
\end{lstlisting}
You should now have the executables in the bin directory: pod crs2trs brdc2ecef

Test
To test your build of the POD ... - You may not need the ulimit command but we found it necessary

\begin{lstlisting}[language=bash]
$ cd ../pod/test
$ ulimit -s unlimited
$ ./sh_test_pod
\end{lstlisting}

At the completion of the test run, the sh\_test\_pod script will return any differences to the standard test resuts
    


Configuration File
The POD Precise Orbit Determination (./bin/pod) uses the configuration file: 
%├── EQM.in (Full force model equation of motion) ├── VEQ.in (For variational equations) ├── POD.in (For all other config)


%\section{From precompiled binaries}
%
%To do..
%
%\include{implementation overview of software}
% block diagram from aaron
%
%\chapter{PEA examples}
\label{ch:pea_examples}

In this section we go through a number of different ways that the pea can be used to process GNSS data.
\begin{enumerate}
	\item Precise Point Positioning (PPP) processing - In this section we will demonstrate how to processing in PPP mode using the Ionosphere free combination, we will provide an example on how to use IGS products to obtian a float solution, and then an example on how to obtain an ambguity fixed solution. We will also cover how to process gnss streams in realtime.
	\item Obtain an orbit solution from a global tracking network
	\item Obtain an orbit and clock solution from a global tracking network
	\item How to process a Global solution in real-time
	\item How to obtain an ionosphere model 
\end{enumerate}
%


%\chapter{Processing in end user mode}
When set to end user mode, the PEA component of Ginan will process each station separately. This mode will allow the estimation of parameters avaialable to users with single receivers. Expected application of this processing mode are
\begin{itemize}
	\item Estimate the position of a static point
	\item Track the path of a moving vehicle
	\item Monitor atmospheric conditions around the receiver
	\item Sychronise a locak clock
\end{itemize}
The end user  mode of the PEA is based on precise point positioning technique

\textit{Precise Point Positioning}(PPP) is a GNSS positioning method, originally developed, for calculating location of autonomous receivers with high levels of accuracy and precision. PPP aims to calculate the end user position by rigorously modelling and/or estimating error sources in GNSS measurements. 


Parameters that can potentially be estimated in end user mode are:
\begin{itemize}
	\item Antenna position
	\item Receiver clock offset
	\item Tropospheric delay at receiver location
	\item Ionospheric delay at the receiver location (not yet available)
	\item Carrier phase ambiguities
\end{itemize}

\section{PEA PPP Processing examples}
Examples of end user processing using the PEA follows. GINAN applications use YAML format to define configuration files. After installing the dependencies and compiling building the PEA application, the PEA processing can be started by typing the command 
\begin{verbatim}
$ ./pea --config <path_to_config_file>
\end{verbatim}

Details on the configuration parameters included in YAML files can be found in chapter \ref{pea_yaml_configuration}. The processing results for each receiver can be found on the trace files. The name of these output files can be set by the user, in the examples below they will typically be *.TRACE files.

\textbf{Receiver position}  results are preceded by the "\$POS" label and thus, in Linux, can be extracted using the command:
 \begin{verbatim}
$ grep "$POS" <path_to_trace_file>
\end{verbatim}
the output line for the for receiver position will have 10 comma separated fields with the following format:
\begin{verbatim}
$POS,2166,278015.000,6,-4052053.0060,4212836.8682,-2545105.0796,0.0245227,0.0231919,0.0163678
\end{verbatim}
the fields represent, from left to right:
\begin{enumerate}
	\item  "\$POS" label
	\item  GPS week
	\item  GPS TOW in seconds
	\item  Solution type (6 for float PPP, 1 for ambiguity fixed PPP)
	\item  Receiver ECEF X position in meters
	\item  Receiver ECEF Y position in meters
	\item  Receiver ECEF Z position in meters
	\item  Standard deviation of ECEF X positions in meters
	\item  Standard deviation of ECEF X positions in meters
	\item  Standard deviation of ECEF X positions in meters
\end{enumerate}

\textbf{Receiver clock} offset results are preceded by the "\$CLK" label and thus, in Linux, can be extracted using the command:
 \begin{verbatim}
$ grep "$CLK" <path_to_trace_file>
\end{verbatim}
the output line for the for receiver position will have 13 comma separated fields with the following format:
\begin{verbatim}
$CLK,2166,278015.000,6,14,3.1902,0.0000,1.1924,0.0000,0.0860,0.0000,0.0953,0.0000
\end{verbatim}
the fields represent, from left to right:
\begin{enumerate}
	\item  "\$CLK" label
	\item  GPS week
	\item  GPS TOW in seconds
	\item  Solution type (6 for float PPP, 1 for ambiguity fixed PPP)
	\item  Number of satellites used in the solution
	\item  Receiver clock offset for with respect to GPS clock, in nanoseconds
	\item  Receiver clock offset for with respect to GLONASS clock, in nanoseconds
	\item  Receiver clock offset for with respect to Galileo clock, in nanoseconds
	\item  Receiver clock offset for with respect to Beidou clock, in nanoseconds	
	\item  Standard deviation of clock offset wrt. GPS, in nanoseconds
	\item  Standard deviation of clock offset wrt. GLONASS, in nanoseconds
	\item  Standard deviation of clock offset wrt. Galileo, in nanoseconds
	\item  Standard deviation of clock offset wrt. Beidou, in nanoseconds	
\end{enumerate}
If clock offsets for a particular constellation are not available both the offset and its variance will be set to 0.


\textbf{Tropospheric delays} at the receiver position are preceded by the "\$TROP" label and thus, in Linux, can be extracted using the command:
 \begin{verbatim}
$ grep "$TROP" <path_to_trace_file>
\end{verbatim}
the tropospheric delay solutions will be represented to either a single line, with the "\$TROP" or three lines, as follows:
\begin{verbatim}
$TROP,2166,278015.000,6,14,2.294950,0.0030977
$TROP_N,2166,278015.000,6,14,-0.174797,0.0181385
$TROP_E,2166,278015.000,6,14,-0.223868,0.0250276
\end{verbatim}
each of the troposphere output line will contain 7 comma separated field, of which the first five are:
\begin{enumerate}
	\item  Label, "\$TROP", "\$TROP\_N" or "\$TROP\_E"
	\item  GPS week
	\item  GPS TOW in seconds
	\item  Solution type (6 for float PPP, 1 for ambiguity fixed PPP)
	\item  Number of satellites used in the solution
\end{enumerate}
The line starting with "\$TROP" contain the Zenith Tropospheric Delay (ZTD) and its standards deviation, both in meters, as their last two fields.  The line starting with "\$TROP\_N" contains the tropospheric delay gradient in north-south direction, and  the line starting with "\$TROP\_E" contains the tropospheric delay gradient in east-west direction.

Configuration files for specific examples have been added to the \textit{examples} folder in the repository. Examples corresponding to end user processing are explained bellow.

\subsection{Post processed PPP solutions with floating ambiguities using IF combination}
The configuration file named \textit{examples/ex11_pea_pp_user_gps.yaml} sets the PEA to calculate a post-process end user solution for a static receiver. 
 
In this example we will process 24 hours of data from a permanent reference frame station. The algorithm will use an L1+L2 and L1+L5 ionosphere-free combination.
The log files and processing results can be found in `<path to pea>/output/exs/EX01\_IF/`.
\begin{lstlisting}
$ ./pea --config ../../config/EX01-IF-PPP.yaml
\end{lstlisting}
The pea will then have the following output in <path to pea>/output/exs/EX01\_IF/ :

EX01\_IF20624.snx              - contains the station position estimates in SINEX format
EX01\_IF-ALIC2019199900.TRACE  - contains the logging information from the processing run

\begin{lstlisting}[language=bash]
$ grep "REC_POS" /data/acs/pea/output/exs/EX01_IF/EX01_IF-ALIC201919900.TRACE > ALIC_201919900.PPP
\end{lstlisting}
This will pipe all of the receiver position results reported in the station trace file to a seperate file for plotting.
\begin{lstlisting}[language=bash]
$ python3 /data/acs/pea/python/source/pppPlot.py --ppp /data/acs/pea/output/exs/EX01_IF/ALIC_201919900.PPP
\end{lstlisting}
This will then create the plots alic\_pos.png, a time series of the difference between the estimated receiver position and the median estimated position.
And the plot alic\_snx\_pos.png, a time series of the difference between the estimated receiver position and the IGS SINEX solution for Alic Springs on this day.

\subsection{Single Frequency Processing} 
\begin{lstlisting}[language=bash]
$ ./pea --config ../../config/Ex01-SF-PPP.yaml

$ grep "\$POS" /data/acs/pea/output/exs/EX01_SF/EX01_SF-ALBY202011500.TRACE
\end{lstlisting}
And you should see something similar to the following:
\begin{lstlisting}[language=bash]
$POS,2062,431940.000,0,-4052052.7956,4212836.0144,-2545104.6423,0.00000043966020,...
$POS,2062,431970.000,0,-4052052.7956,4212836.0144,-2545104.6423,0.00000043965772,...
\end{lstlisting}



\subsection{Processing realtime}
To process a continuous GPS station in real-time you will need to access the data stream from a NTRIP stream and the correction products from a NTRIPCaster.
Geoscience Australia is running a caster that provides global data stream, a dense network of stream covering the Australian region, and correction procdust provide by IGS Analysis Centres.
You will need to apply for an AUSCORS account, or use the new NTRIPCaster that streams using https.

You can apply for an account at the following link : \href{https://gnss-users-prod.auth.ap-southeast-2.amazoncognito.com/login?response_type=code&client_id=11njl767q0tl1faf9qna469vl1&redirect_uri=https://search-gnss-elasticsearch-prod-5omhch5quzlu5dcpbct4ev5qz4.ap-southeast-2.es.amazonaws.com/_plugin/kibana/app/kibana&state=e36b2054-7ace-4931-91a3-5ba6de893917}{New GA Account link}

Once your have your account and password you can test that you can successfuly connect to the NTRIPCaster by using the following curl command:
\begin{lstlisting}[language=bash]
$ curl https://ntrip.data.gnss.ga.gov.au/MOBS00AUS0 --http0.9 -i -u'<username>:<password> --output -'
\end{lstlisting}


%
\section{Processing a Global Network to obtain satellite clock products}
In this example 24 hours of data from a small global network of 87 stations is processed to obtain the clock products needed for high precision positioning.

Check that the paths in the configuration file for the products and RINEX files are correct for your system. If you have followed the convention layed out in the INSTALL.md document you should not need to amend anything.

To start the processing use the command:
\begin{lstlisting}
$ ./pea --config ../../config/Ex02-Network.yaml
\end{lstlisting}
The process will take approximatelly 2-3 hours to complete depending on CPU performance. The log files and processing results can be found in /data/acs/pea/output/examples/Ex02, or the alternative directory you have specified in the configuration file.

Change into your output directory. You should find a .TRACE file for each station processed, and a PEA.SUM file.
\begin{lstlisting}
$ cd /data/acs/pea/output/examples/Ex02
\end{lstlisting}
To verify your solution, first grep for the xp values:
\begin{lstlisting}
$ grep 'network xp' netprocessing.out > xp_test.txt
\end{lstlisting}
and then run:
\begin{lstlisting}
$ python3 /data/acs/pea/python/src/comprun.py --test /data/acs/output/xp_test.txt --standard /data/acs/pea/example/EX02/standard/xp_standard.txt
\end{lstlisting}
This will produce the plots Posdiff.png, recclk.png, satclk.png, and zwd.png.

\section{Example 03 Processing a Global Network to obtain the orbit and clock products}
In this example 24 hours of data from a small global network of 87 stations is processed to obtain the orbit and clock products needed for high precision positioning.

Check that the paths in the configuration file for the products and RINEX files are correct for your system. If you have followed the convention layed out in the INSTALL.md document you should not need to amend anything.

To start the processing use the command:
\begin{lstlisting}
$ ./pea --config ../../config/Ex03-Network_Orbits.yaml
\end{lstlisting}
The process will take approximatelly 2-3 hours to complete depending on CPU performance. The log files and processing results can be found in /data/acs/pea/output/examples/Ex03, or the alternative directory you have specified in the configuration file.

Change into your output directory. You should find a .TRACE file for each station processed, and a PEA.SUM file.
\begin{lstlisting}
$ cd /data/acs/pea/output/examples/Ex03
\end{lstlisting}
To verify your solution, first grep for the xp values:
\begin{lstlisting}
$ grep 'network xp' netprocessing.out > xp_test.txt
\end{lstlisting}
and then run:
\begin{lstlisting}
$ python3 /data/acs/pea/python/src/comprun.py --test /data/acs/output/xp_test --standard /data/acs/pea/example/EX03/standard/xp_standard.txt
\end{lstlisting}
This will produce the plots Posdiff.png, recclk.png, satclk.png, and zwd.png.

To compare the satellite clocks run:
\begin{lstlisting}
$ python3 /data/acs/pea/python/src/compareclk.py --standard /data/acs/pea/example/EX03/standard/aus20624.clk  --test ./aus20624.clk
\end{lstlisting}
This will produce plots for the differences in satellite clocks G02 through to G32 as well as calculating the RMS and standard deviation with respect to the standard. G01.png does not exists at this has been used as the pivot satellite clock that we use to remove the bias from.

\section{Example 04 Processing a Global Network to obtain Ionospheric Vertical Total Electron Content (VTEC) Maps}
In this example 24 hours of data from a small global network of 87 stations is processed to obtain the IONEX formatted Ionosphere VTEC maps and SINEX formatted satellite Diferential Signal Biases (DSB). 
Ionospheric VTEC maps follows the IONEX 1.1 format, which can be found in: https://gssc.esa.int/wp-content/uploads/2018/07/ionex11.pdf 
The satellite bias follows the bias-SINEX format, which can be found in: http://ftp.aiub.unibe.ch/bcwg/format/draft/sinex\_bias\_100\_dec07.pdf

Check that the paths in the configuration file for the products and RINEX files are correct for your system. 
If you have followed the convention layed out in the INSTALL.md document you should not need to amend anything.

To start the processing use the command:
\begin{lstlisting}
$ ./pea --config <installation directory>/pea/config/Ex04-Ionosphere.yaml
\end{lstlisting}

The process will take approximatelly 1-2 hours to complete depending on CPU performance and setting specifications.

The IONEX and bias-SINEX files can then be used to obtain SPP and PPP positioning solutions as specified in PPPExamples.md

A utility to plot the TEC values in an IONEX file have can be found at:

<installation directory>/pea/python/source/plotIONEX.py
This utility will ask for the path/filename of the IONEX file and a start and stop time in hhmmss format. The Python code will then generate a figure in IONEX\_yyyy-mm-dd\_hh:mm:ss.png format for each entry between the start and stop time.

\section{Network Post-Processing - Ultra-rapid product example}

\newthought{We will} run the pea in a post-processing mode, along side the POD, to produce orbits and clock in near-realtime, as would be run to produce an IGS ultra-rapid product

% Should move this to main?
% taken directly from https://www.overleaf.com/learn/latex/LaTeX_Graphics_using_TikZ:_A_Tutorial_for_Beginners_(Part_3)%E2%80%94Creating_Flowcharts
%\tikzstyle{startstop} = [rectangle, rounded corners, minimum width=3cm, minimum height=1cm,text centered, draw=black, fill=red!30]
%\tikzstyle{io} = [trapezium, trapezium left angle=70, trapezium right angle=110, minimum width=3cm, minimum height=1cm, text centered, draw=black, fill=blue!30]
%\tikzstyle{process} = [rectangle, minimum width=3cm, minimum height=1cm, text centered, draw=black, fill=orange!30]
%\tikzstyle{decision} = [diamond, minimum width=3cm, minimum height=1cm, text centered, draw=black, fill=green!30]
%\tikzstyle{arrow} = [thick,->,>=stealth]


%\begin{tikzpicture}[node distance=2cm]
%\node (start) [startstop] {POD};
%\node (in1) [io, below of=start] {Orbit Partial file}; 
%\node (in2) [io, right of=in1, xshift=3cm] {RINEX data};
%\node (pro1) [process, below of=in1] {PEA};
%\node (in3) [io, below of=pro1] {Clock File, SINEX, 
%                                 Updated Orbit Partial};
%\node (proc2) [process, below of=in3] {POD}
%\node{in4) [io. below of=proc2]{SP3 file}
% Now draw some linking arrows
%\draw [arrow] (start) -- (in1);
%\draw [arrow] (in1) -- (pro1);
%\draw [arrow] (in2) -- (pro1);
%\draw [arrow] (pro1) -- (in3);
%\end{tikzpicture}


\section{Network Real-time Processing}

\newthought{The PEA} is designed to do some stuff.

% Move this to products and data description page.
ANTEX files can be downloaded from https://files.igs.org/pub/station/general/igs14.atx
DCB files can be downloaded from ftp://ftp.aiub.unibe.ch/CODE/P1C1.DCB
Satellite metadata can be obtained from: https://files.igs.org/pub/station/igs\_satellite\_metadata.snx
%
\chapter{POD Examples}
\label{ch:pod_examples}

\section{Processing Example 1}
In this example the pod will perform a dynamic orbit determination for PRN04 over a 6 hour arc. 
The full gravitational force models are applied, with a cannonball model SRP model.

To run the POD ...
\begin{lstlisting}
$ bin/pod
\end{lstlisting}

This should output the following to stdout...
\begin{lstlisting}
Orbit Determination
Orbit residuals in ICRF : RMS(XYZ)   1.6754034501980351E-002   5.2908718335411935E-002   1.5676115599034774E-002
Orbit Determination: Completed
CPU Time (sec)   298.48134399999998
External Orbit comparison
Orbit comparison: ICRF
RMS RTN   2.8094479714173427E-002   2.4358145601708528E-002   4.4097979280889953E-002
RMS XYZ   1.6754034501980351E-002   5.2908718335411935E-002   1.5676115599034774E-002
Orbit comparison: ITRF
RMS XYZ   3.9069978513805753E-002   3.9343671258381237E-002   1.5660654272651970E-002
Write orbit matrices to output files
CPU Time (sec)   349.19307899999995
The results above show that our orbits arcs, over 6 hours, are currently within 2-5 cm of the final combined IGS orbit.
\end{lstlisting}

The processing also produces the following output files...
\begin{verbatim}
%├── DE.430            planetary ephemris intermediate file
%├── Amatrix.out       design matrix
%├── Wmatrix.out       reduced observation matrix
%├── orbext_ICRF.out   intermediary file for the IGS orbit solution in ICRF for comparison purposes
%├── orbext_ITRF.out   intermediary file for the IGS orbit solution in ITRFfor comparison purposes
%├── dorb_icrf.out     differences in solutions in ICRF
%├── dorb_RTN.out      differences in solutions in orbital frame components radial, tangential and normal (RTN)
%├── dorb_Kepler.out   differences in solutions in keperian elements 
%├── dorb_itrf.out     differences in solutions in ITRF 
%├── orb_icrf.out      the final estimated orbit in ICRF
%├── orb_itrf.out      the final estimated orbit in ITRF
%├── VEQ_Smatrix.out   State transition matrix from the variational equations solution
%├── VEQ_Pmatrix.out   Sensitivity matrix from the variational equations solution
\end{verbatim}

\section{Processing Example 2 - ECOM2 SRP}
In this example we will change the SRP model to use the ECOM2 model.

Edit the EQM.in file so that the Solar Radiation Pressure configuration section now looks:

! Solar Radiation Pressure model: ! 1. Cannonball model ! 2. Box-wing model ! 3. ECOM (D2B1) model SRP\_model 3

Then edit VEQ.in, so that the Non-gravitational forces now looks like:

%% Non-gravitational Effects Solar_radiation 0 Earth_radiation 0 Antenna_thrust 0

! Solar Radiation Pressure model: ! 1. Cannonball model ! 2. Box-wing model ! 3. ECOM (D2B1) model SRP\_model 3

run the POD ...

\begin{lstlisting}
$ bin/pod
\end{lstlisting}
This should output the following to stdout...
\begin{lstlisting}
Orbit Determination
Orbit residuals in ICRF : RMS(XYZ)   2.0336204859568077E-002   8.4715644601919167E-003   3.9687932322714677E-002
Orbit Determination: Completed
CPU Time (sec)   299.68054799999999
External Orbit comparison
Orbit comparison: ICRF
RMS RTN   2.8182836396022540E-002   2.4598832384842121E-002   2.5879201921952168E-002
RMS XYZ   2.0336204859568077E-002   8.4715644601919167E-003   3.9687932322714677E-002
Orbit comparison: ITRF
RMS XYZ   1.8757217704973204E-002   1.1635302426688266E-002   3.9702619816620370E-002
Write orbit matrices to output files
CPU Time (sec)   350.88653299999999\
\end{lstlisting}

\section{Example 3 - (pod/examples/ex3)}
GPS IGS SP3 file orbit fitting, orbit prediction and comparison to next IGS SP3 file

\section{Example 4 - (pod/examples/ex4):}
Integration of POD initial conditions file generated by the PEA

\section{Example 5 - (pod/examples/ex5):}
ECOM1+ECOM2 hybrid SRP model
In each example directory (ex1/ex2/ex3/ex4) there is a sh\_ex? script that when executed will run the example and compare the output with the expected solution.

\textit{The POD} is designed to do some stuff.
%
% Start the main matter (normal chapters)
%\mainmatter
%
%\part{Background Theory}
%\chapter{GNSS Overview}
\label{ch:GNSSOverview}

\section{GPS}

\section{Glonass}

\section{Galileo}

\section{Beidou}

\section{QZSS}

\newthought{The PEA and POD} is designed to do some stuff.
%
%\chapter{Observation Modelling}
\label{ch:observation_modelling}

\section{Ionosphere-Free Observations}

\section{Undifferenced - Uncombined}
To do ..
\newthought{The PEA and POD} is designed to do some stuff.

\section{GPS Quarter Cycle}
\begin{fullwidth}
The $L_2$ Civil code (L2C-code) is shifted by a quarter cycle with respect to the P-code. If a receiver is using either the one or the other code to reconstruct the carrier phase measurements, then they will also be shifted by a quarter of a wavelength relative to each other.\\
The noise of phase measurement based on L2C-code is usually lower than the noise of phase observations reconstructed based on p-code. However this is only possible to obtain from satellites that have launched since GPS Block IIR-M was deployed. Older GPS satellites do not provide the L2C-code signal.\\
In RINEX version 2, in order to prevent potential problems in ambiguity resolution some receiver manufactures do correct the phase measurements by 0.25 cycles to keep the phase observations consistent, while others provide the uncorrected phase measurements, an din other cases the user can decide if the correction is applied. Since RINEX version 3.02 the definition has been defined that the phase measurements need to be corrected to have a consistent set of observable that can be introduce for ambiguity resolution.\\
One option to prevent having the quarter cycle issue is to ensure that ambiguities are not resolved between a Block IIR-M or later  with an older satellite.
\end{fullwidth}

\section{Single-satellite and Single-Receiver Observation Combinations}

Linear combinations of the original observations are often applied to eliminate model parameters (eg ionosphere slant delays) or to transform ambiguity parameters (eg widelane transformation). Some of these transformations maintain the information content of the system as they are invertible, but other transformation that are used are do not, and these should be used with caution as the information that contributes to the parameters of interest are lost.\\
%
A large number of different permutations of linear combinations can be formed, with just the dual-frequency legacy signals, this can increases even more as more additional frequencies are available with modernized GNSS signals are deployed. We will only cover the ones used by the PEA in this manual.\\
%
Combining two or more carrier-phase observations into a new signal leads to a different frequency/wavelength. To generalise in the case of a combination for which $\sigma \alpha = 1$, the combined frequency $f_c$ is:
\begin{equation}
f_c = \sigma_{j=1}^n i_j f_j 
\label{eq:comb_freq}
\end{equation}

Remembering that all frequencies for GNSS are derived from a single frequency $f_0$ by multiplication with an integer $k_j$, the individual frequency is obtained from $f_j = j_jf_0$. substituting this into the above equation

$f_c = (\sigma_{j=1}^n i_j k_j) = kf_0$ \label{eq:}

where $k$ is the \emph{lane number}. The corresponding wavelength is:

$\lambda_c = c / kf_0 = \lambda_0 / k$ 

where $\lambda_0$ is the wavelength of the base frequency $f_0$. Since all $i_j$ and $k_j$ are integers, $k$ is also an integer. This parameter uniquely defines the frequency and wavelength of the new signal combination.

With the help of the lane number $k$, the combinations can be categorized into three groups:
\begin{enumerate}
    \item \emph{wide-lane} combination,where the combined wavelength is larger than the largest individual wavelength in the combination.
    \item \emph{intermediate} combinations, for which $\lambda_0$ leas between the largest and shortest individual wavelength.
    \item \emph{narrow-lane} combinations which have a shorter wavelength than the individual signal with the shortest wavelength in the combination.
\end{enumerate}

\newthought{The zero-differenced}, ionosphere free mathematical model for code and phase measurements using dual-frequency can be described by:\\

$ E(P_{r,IF}^S) = \rho_{r}^s + c(dt_{r} - dt^s) + \tau_r^s + c (d_{r,IF} + d_{r,IF}^s) $\label{eq:code_IF_eq}\\

$ E(L_{r,IF}^S) = \rho_{r}^s + c(dt_{r} - dt^s) + \tau_r^s + \lambda_{IF} (z_{r,IF}^s + \delta_{r,IF} - \delta_{IF}^S$\label{eq:phase_IF_eq}\\

with:

$E()$ the expectation notation
$P_{r,IF}^S$ code measurements (m) between satellite $s$ and receiver $r$ for the ionosphere-free measurements;\\
$L_{r,IF}^S$ phase measurements (m)\\
$\rho$ the geometric distance (m)\\
$c$ speed of light (m/s)\\
$dt_r$ receiver clock error (s)\\
$dt^S$ satellite clock error (s)\\
$\tau_r^s $ slant troposphere delay between satellite s and receiver r (m)\\
$dt_{r,IF}$ receiver code bias (s)\\
$dt_{IF}^S$ satellite code bias (s)\\
$\lambda_{IF}$ ionosphere-free wavelength (m)\\
$z_{r,IF}^S$ ionosphere-free ambiguity vector (cycle)\\
$\delta_{r,IF}^S$ receiver ionosphere-free phase bias (cycle)\\
$\delta_{IF}^S$ satellite ionosphere-free phase bias (cycle)\\
%
The ionosphere-free combination will remove the first-order ionosphere delay, both equations are rand deficient and the ambiguity term in eq (2) is not an integer. In order to resolve the ambiguities the linear dependency among the unknown parameters needs to be resolved and the integer property of the ambiguities needs to be recovered. 

\section{Widelane combinations}
The group of wide-lane combinations is especially useful to help with integer ambiguity resolution due to their longer wavelength.

%\begin{figure*}[h]
%\includegraphics[]{images/widelane.png}
%  \caption{Illustration of the linear widelane combinations. Two original frequencies (top and middle) are subtracted to form a longer wavelength observable (bottom)}%
%  \label{fig:widelane}%
%\end{figure*}



\subsection{Common widelane}
\begin{fullwidth}
To improve ambiguity resolution, often the wide lane linear combination (also called L5 or LW linear combination) is formed from the L1 and L2 carrier phase observables. It is designed to be a geometry-preserving combination using only carrier-phase measurements in the frequencies $f_a$ and $f_b$. By selecting the integer coefficients as $i_A = +1$ and $i_B = -1$, one obtains the WL carrier-phase combination $\phi_{r,WL}^s$ in units of cycles:
\end{fullwidth}
\begin{equation}
\phi_{r,WL}^s = \phi_{r,A}^A - \phi_{r,B}^S = \frac{\psi_{r,A}^s}{\lambda_A} - \frac{\psi{r,B}^s}{\lambda_B} \label{eq:wl_cycles}
\end{equation}

The corresponding wavelength $\lambda_{WL}$ is:
\begin{equation}
\lambda_{WL} = \frac{c}{f_A - f_B} 
    \label{eq:wl_wavelength}
\end{equation}

Multiplication of the two above equations leads to the wide-lane combination $\psi_{r,WL}^s$ in units of meters:
\begin{equation}
\psi_{r,WL}^S = \frac{f_A}{f_A-f_B}\psi_{r,A}^s - \frac{f_B}{f_A-f_B}\psi_{r,B}^s
\end{equation}

\subsection{Melbourne-Wubbena linear combination}
The Melbourne-Wubbena linear combination (Melbourne 1985),(Wubbena 1985) is a linear combination of the L1 and L2 carrier phase plus the P1 and P2 pseudorange. The geometry, troposphere and ionosphere are eliminated by it. The Melbourne-Wubbena linear combination can be represented as:

$
E(L_{r,IF}^S) - \frac{cf_2z_{r,w}^s}{f_1^2 - f_2^2} = \rho_r^s + c(dt_{r,IF} - dt_{IF}^s) + \tau_r^s + \lambda_n z_{r,1}^s + (\lambda_{IF}\delta_{r,IF}$

Since ,,% comprises of both code and phase measurements, it is reasonable to exclude the lower
elevation measurements to avoid the multipath impacts from the code observation. Normally, with
30 degree elevation cut-off, an averaging of 5 minutes of (4) is good enough to fixing the wide-lane
ambiguities [RD 04]. The rests are the wide-lane phase bias, which can be broadcasted to the user for
user side wide-lane ambiguity resolution. Either choosing a pivot receiver bias or a single-differencing
between two satellites can avoid the linear dependency. 


-doesn't need lambda not as correlated
%
%\chapter{Kalman Filtering}
\label{ch:kalman_filter}
%
The Kalman filter is over 50 years old but is still one of the most important and common data fusion algorithms in use today. 
Named after Rudolf E.Kálmán, the great success of the Kalman filter is due to its small computational requirement, elegant recursive properties, and its status as the optimal estimator for one-dimensional linear systems with Gaussian error statistics. 
Typical uses of the Kalman filter include smoothing noisy data and providing estimates of parameters of interest. 
Kalman filtering is used in a wide range of applications include global positioning system receivers, in control systems, through to the smoothing the output from laptop trackpads, and many more.

\section{Overview of Kalman Filtering}

Kalman filter are typically used to estimate parameters which change with time. 
Parameters with no process noise are called deterministic.
A Kalman filter has measurements $y_t$, with noise $y_t$, and a state vector $\hat x_t$ (or a parameter list) which have specified statistical properties.

The observation equation at time t:
\begin{equation}
    y_t = H_t x_t + \epsilon_t	 \label{eq:kfObs}
\end{equation}

The state transition equation:
\begin{equation}
    x_{t+} = F_t x_t + w_t	
\end{equation}

The kalman filter processing is broken up into three main steps.

\textit{Prediction} {uses a process noise model} to 'predict' the parameters at the next data epoch, subscript is time quantity refers to, where as the superscript refers to the time of the data:
\begin{equation}
    \hat{x}_{t+1}^t = F_t \hat{x}_t^t
\end{equation}
where, $F_t$ is the state transition matrix
\begin{equation}
    P_{t+1}^t = F_t P_t^t F_t^\intercal + Q_t
\end{equation}
where, $Q_t$ is the process noise covariance matrix.
The state transition matrix $F$ projects the state vector (parameters) forward to the next epoch.
\begin{itemize}
    \item For random walk $F$ = 1
    \item For rate terms: $F$ is matrix 
    $\begin{bmatrix}
    1 & \delta t\\
    0 & 1
  \end{bmatrix}$
    \item for FOGM: $F$ = $e^{-\delta t \beta}$
    \item For white noise $F$ = 0
\end{itemize}
The second equation projects the covariance matrix of the state vector, $P$, forward in time. Contributions from the state transition and process noise ($Q$ matrix). 
$Q$ elements are 0 for deterministic parameters.
%
\textit{The Kalman gain} {is the matrix} that allocates the differences between the observation at time t+1 and their predicted value at this time based on the current values of the state vector according to the noise in the measurements and the state vector noise.

\section{Comparison between Weighted Least Squares and Kalman Filtering}

\begin{itemize}
    \item In kalman filtering apriori constraints must be give for all parameters. This is not needed in weighted least squares, but can also be done.
    \item Kalman filters can allow for 0 variance parameters, this cannot be done in WLS, as this requires the inversion of the constraint matrix.
    \item Kalman filter can allow for a method of applying absolute constraints, WLS can only tightly constrain parameters.
    \item Kalman filters are more prone to numerical stability problems, and take longer to run (they have more parameters).
    \item Process noise models can be implemented in WLS, but they are computationally slow.
\end{itemize}

\section{Implementation in the PEA}

\subsection{Robust Kalman Filter Philosophy}

It is well known that the Kalman filter is the optimal technique for estimating parameters of interest from sets of noisy data - provided the model is appropriate.

In addition, statistical techniques may be used to detect defects in models or the parameters used to characterise the data, providing opportunities to intervene and make corrections to the model according to the nature of the anomaly.

By incorporating these features into a single generic module, the robustness that was previously available only under certain circumstances may now be automatically applied to all systems to which it is applied. These benefits extend automatically to all related modules (such as RTS), and often perform better than modules designed specifically to address isolated issues.

\subsection{Initialisation}

When parameters' initial values are not known a-priori, it is often possible to determine them using a least-squares approach.

To minimise processing times, the minimal subset of existing states, measurements, and covariances are used in least-squares estimation whenever the initial value and variance of a parameter is unspecified.

For rate parameters, multiple epoch’s worth of data are required for an ab-initio initialisation. This logic is incorporated into the filter and is applied automatically as required.

\subsection{Outlier detection, Iteration, and Hypothesis Testing}

As a statistical machine, the Kalman filter is capable of detecting measurements that do not fit within the system as modelled.

In these cases, the model may be adjusted on-the-fly, to allow all measurements to be continued to be used without contaminating the results in the filter.

A typical example of a modelling error in GNSS processing is a cycle-slip, in which the ambiguity term (which usually modelled with no change over time) has a discontinuity. Other examples may include clock-jumps or satellite burns.

Hypotheses are to be generated for any measurements that are statistical outliers, and the model iterated as required.

\subsection{Performance Optimisation}

The inversion of large matrices as required by the Kalman filter easily dominates the processing time required during operation. Techniques are available to reduce, and distribute this processing burden across multiple processors.

The Eigen library is used for algebraic manipulation which allows for automatic parallelisation of vector algebra, and improves code robustness by checking matrix dimensions while in use.

\subsubsection{Chunking}

By dividing measurements into multiple smaller sub-matrices, the long inversion times may be reduced, as the inversion order is of $O(n^3)$

\subsubsection{Blocking}
By separating the filter covariance matrix into a block-diagonal form, individual blocks of the filter may be processed individually, without degredation in accuracy. This may improve performance, and may also enable blocks that are relatively independent to be processed separately, albeit with some degredation in accuracy.

\section{Configuration} \label{KFConfig}

All elements within the kalman filter are configured using the yaml configuration file, and use a consistant format.

\subsection{default\_filter\_parameters}

\begin{lstlisting}[language=yaml,caption=Filter Parameters:]

default_filter_parameters:

    stations:

        error_model:        elevation_dependent         #uniform elevation_dependent
        code_sigmas:        [0.15]
        phase_sigmas:       [0.0015]

        pos:
            estimated:          true
            sigma:              [0.1]
            proc_noise:         [0] #0.57 mm/sqrt(s), Gipsy default value from slow-moving
            proc_noise_dt:      second
            #apriori:                                   # taken from other source, rinex file etc.
            #frame:              xyz #ned
            #proc_noise_model:   Gaussian

        clk:
            estimated:          true
            sigma:              [0]
            proc_noise:         [10]
            proc_noise_dt:      second
            #proc_noise_model:   Gaussian

        clk_rate:
            estimated:          false
            sigma:              [500]
            proc_noise:         [1e-4]
            proc_noise_dt:      second
            clamp_max:          [+500]
            clamp_min:          [-500]
            
    satellites:

        clk:
            estimated:          true
            sigma:              [1000]
            proc_noise:         [1]
            #proc_noise_dt:      min
            #proc_noise_model:   RandomWalk

        # clk_rate:
        #     estimated:          true
        #     sigma:              [10]
        #     proc_noise:         [1e-5]
        #     # clamp_max:          [+500]
        #     # clamp_min:          [-500]

        orb:
            estimated:          false

    eop:
        estimated:  true
        sigma:      [40]


override_filter_parameters:

    stations:
        #ALIC:
            pos:
                sigma:              [0.001]
                proc_noise:         [0]
\end{lstlisting}


The majority of estimated states are configured in this section. These configurations are applied to all estimates unless another configuration overrides these parameters in the override\_filter\_parameter section.

The parameters that are available for estimation include:
\begin{itemize}
\item stations:
\begin{itemize}
\item pos
\item pos\_rate
\item clk
\item clk\_rate
\item amb
\item trop
\item trop\_grads
\end{itemize}
\item satellites:
\begin{itemize}
\item pos (coming soon)
\item pos\_rate (coming soon)
\item clk
\item clk\_rate
\item orb
\end{itemize}
\end{itemize}


\subsection*{estimated}

Boolean to add the state(s) to the kalman filter for estimation.

\subsection*{sigma}

List of a-priori sigma values for each of the components of the state.

If the sigma value is left as zero (or not initialised), then the initial variance and value of the state will be estimated by using a least-squares approach.
In this case, the user must ensure that the solution is likely rank-sufficient, else the least-squares initialisation will fail.

For states with multiple elements (eg, X,Y,Z positions), multiple sigma values may be added to the list. However, if insufficient values are added to the list, the intialiser will use the last value in the list for any extra elements.
ie. Setting \lstinline{sigma: [10]} is sufficient to set all x,y,z components of the apriori standard deviation to 10.

\subsection*{proc\_noise}

List of process noises to be added to the state during state transitions. These are typically in m/sqrt(s), but different times may be assigned separately.
As for the sigma list, the last value will be used for any elements exceeding the list length.

\subsection*{proc\_noise\_dt}

Unit of measure for process noise. 
May be left undefined for seconds, or using sqrt\_second, sqrt\_seconds, sqrt\_minutes, sqrt\_hours, sqrt\_days, sqrt\_weeks, sqrt\_years.

\subsection{override\_filter\_parameters:}

In the case that a specific station or satellite requires an alternate configuration, or to exclude estimates entirely, the override\_filter\_parameters section may be used to overwrite selected components of the configuration.


\subsection{user\_filter\_parameters, network\_filter\_parameters:}

The internal operation of the kalman filter is specified in this section. It has a large impact on the robustness, and associated processing time that the filter will achieve.

\begin{lstlisting}[language=yaml,caption=Filter Operating Parameters:]

user_filter_parameters:

    max_filter_iterations:      5 #5
    max_prefit_removals:        3 #5

    rts_lag:                    -1      #-ve for full reverse, +ve for limited epochs
    rts_directory:              ./
    rts_filename:               PPP-<CONFIG>-<STATION>.rts

    inverter:                   LLT         #LLT LDLT INV

\end{lstlisting}


\subsection*{max\_prefit\_removals:}

Maximum number of pre-fit residuals to reject from the filter.

After the vector of residuals has been generated and before the filter update stage is computed, the residuals are compared with the expected values given the existing states and design matrix.
If the values are deemed to be unreasonable - because the variances of the transformed states and measurements do not overlap to with a 4-sigma level of confidence - then these measurements are deweighted by deweight\_factor, to prevent the bad values from contaminating the filter.

These measurements are recorded as being rejected, and may have additional consequences according to other configurations such as phase\_reject\_limit.

\subsection*{max\_filter\_iterations:}

Maximum number of times to compute the full update stage due to rejections.

This is similar to the max\_filter\_rejections parameter, but the 4-sigma check is performed with post-fit residuals, which are much more precise.

Rejections that occur in this stage require the entire filter inversion to be repeated, and has an associated performance hit when used excessively.


\subsection*{inverter}

There are multiple inverters that may be used within the kalman filter update stage, which may provide different performance outcomes in terms of processing time and accuracty and stability.

The inverter may be selected from:
\begin{itemize}
\item llt
\item ldlt
\item inv
\end {itemize}



\subsection{outage\_reset\_limit:}
Maximum number of epochs with missed phase measurements before the ambiguity associated with the measurement is reset.

\subsection{phase\_reject\_limit:}
Maximum number of phase measurements to reject before the ambiguity associated with the measurement is reset.


\subsection*{rts\_X}

For details about rts configuration, see section \ref{ch:RTS}







\subsection{Process Noise Guidelines}

Currently in the PEA we have random walk process noise models implemented.

The units are typically in meters, and they are given as $\sigma$ = sqrt(variance)

For a random walk process noise, the process noise is incremented at each epoch as $\sqrt{process\_noise}*dt$ where dt is the time step between filter updates.

If you want to allow kinematic processing, then you can increase the process noise e.g.
proc\_noise [0.003]
proc\_noise\_dt: second 

equates to $\frac{0.003}/\sqrt{s}$
\\ 
Or if you wanted highway sppeds 100km/hr = 28 m/s
proc\_noise [28]
proc\_noise\_dt: second
proc\_noise\_model:   Gaussian
or
         proc\_noise\_model:   RandomWalk
or
         proc\_noise\_model:   FOGM 1.0
Where the number after FOGM gives the correlation time B
That way we could keep the same:
            proc\_noise:         [0.01]
            proc\_noise\_dt:      hour

A nice value for using VMF as an apriori value is 0.1mm /sqrt(s)
%
\begin{lstlisting}
trop:
    estimated:          true
    sigma:              [0.1]
    proc_noise:         [0.01]
    proc_noise_dt:      hour
\end{lstlisting}



\section{Recommended Reading}

\begin{enumerate}
    \item https://ocw.mit.edu/courses/earth-atmospheric-and-planetary-sciences/12-540-principles-of-the-global-positioning-system-spring-2012/lecture-notes/MIT12\_540S12\_lec13.pdf
\end{enumerate}







%
%\chapter{Orbit Modelling}
\label{ch:orbit_modelling}


\textit{POD} is designed to do some stuff.

\section{Gravitational Force Models}

\section{Non-Gravitational Force Models}

\subsection{Solar Radiation Force Models}
The magnitude of the SRP acting on the satellite depends on a wide range of parameters. 
The distance to the Sun and the position of the satellite with respect to Earth and Sun (regarding possible eclipses) define the intensity of the incoming radiation.
 The geometry of the satellite, the optical properties of the external surfaces, and the actual orientation with respect to the Sun largely influence the orientation and magnitude of the evolving SRP.
 %Therefore any SRP model depends on an accurate implementation of the satellite orbit, the attitude, and the geometric/physical properties of the satellite structure.
 
\subsection{Cannonball}
\label{sec:cannonball_srp}
The most basic approach with regard to its analytic development is referred to as the cannonball
model.The cannonball model provides a useful, first-order approximation, however, due to its homogeneous material properties and symmetrical shape approximation. We recommend its use as an apriori model, before estimation.

\subsection{ECOM I}

\subsection{ECOM II}

\subsection{ECOM C}

\subsection{Box Wing}

\subsection{Antenna Thrust}

\subsection{Albedo}

\section{Transformtation between Celestial and Terrestrial Reference Systems}

The variational equations obtained from the POD need to be transformed into the terrestrial reference frame so that the adjustments can be made in the ECEF frame that the PEA operates in.

\begin{equation}
    [CRS] = Q(t)R(t)W(t)[TRS]
\end{equation}

Where,
$CRS$ is the Celestial Reference System
$TRS$ is the Terrestrial Reference Systems
$Q(t)$ is the Celestial Pole motion (Precession-Nutation) matrix
$R(t)$ is the Earth Rotation matrix
$W(t)$ is the Polar motion matrix
\\
\begin{equation}
Q(t) = 
\begin{bmatrix} 
1-aX^2  & -aXY     & X \\
 -aXY   & 1 - aY^2 & Y \\
 -X     & -Y       & 1-a(X^2+Y^2) 
\end{bmatrix}
\end{equation}
\\
\begin{equation}
R(t) = R_2(-\theta) = 
\begin{bmatrix}
cos \theta & -sin \theta & 0 \\
sin \Theta & cos \theta  & 0 \\
0 & 0 1
\end{bmatrix}
\end{equation}
%\\
%\begin{equation}
%    W(t) = R_z (-s^') R_y(x_p) R_x(y_p)
%\end{equation}

%\chapter{Ionosphere Modelling}
\label{ch:ionosphere_modelling}
%
%\begin{fullwidth}
\newthought{Ionospheric delay} is the most important nuisance parameter on GNSS processing. 
GNSS processing algorithm needs to account for it by estimating, correcting or cancelling its effects. 
The objective of the project is to develop Ionospheric modelling modules to aid in GNSS data processing. 
he modules will allow end user position algorithms correct the effect of ionosphere delay. 
The modules also allow the network processing more effectively estimate the ionospheric delays.
\\
%
Two potential ionosphere models have been implemented to date. 
Both are dual-layer models, each layer containing an ionosphere density map. 
The maps are represented by spherical cap harmonics in one case and 3rd order b-splines in another. 
The effectiveness of the maps has been tested using measurements from 60 Australian stations distributed. 
The geometry free combination of carrier smoothed pseudorange have been used as ionosphere delay measurements. 
The spherical cap harmonics tested so far can represent the ionospheric delays with an accuracy of about 2-3 TECu, slightly better than the model used for IONEX (single layer, grid based map), but not enough for the 0.2 TECu accuracy needed for PPP. 
The specific parameters (number of layers, map resolution, layer heights and order of the base function) of each model will be refined to improve model accuracy.
\\
%
%\end{fullwidth}

%\section{Ionosphere-Free Observations}

%\chapter{Ambiguity Resolution}
\label{ch:Ambiguity Resolution}
An advantage of the ambiguity fixed solutions is the significantly reduced number of parameters which have to be solved for. 
A reduction of the normal equation to be inverted is important because usually a duplication of its size leads to over four times longer computing time for the inversion. 
If many parameters are estimated (orbits, Earth orientation parameters etc.) ambiguity resolution improves also the results of much longer sessions than the traditional daily solution.  
Due to the FDMA technology the ambiguity resolution for GLONASS is not a straight forward task, and we have not attempted to implement this into the pea yet.
%
There are many methods that can be used to resolve ambiguities, but they mainly consist of two steps:
%
\begin{enumerate}
    \item The ambiguities are estimates as real numbers (with the other parameters).
    \item Integer values of the ambiguities are resolved using the results of step 1 (the real-value ambiguities and the VCV matrix) employing a number of statistical tests to ensure a reliable estimate.
\end{enumerate}
%
\textit{Determining a reference or a pivot} is normally done by selecting a station with the most number of observations, however this is not possible to no apriori for a systems that is designed to run in real-time.
%
In the pea we have implemented a number of different ambiguity resolution strategies:
\begin{itemize}
    \item rounding and weighted rounding
    \item bootstrapping
    \item lambda decorelation
    \item BIE
\end{itemize}
%
\section{rounding algorithm}
The simplest strategy to apply is to round the real-values estimates to the nearest integers, without using any variance co-variance information.
%
\section{bootstrapping}
%
The bootstrapping algorithm takes the first ambiguity and rounds its value to the nearest integer. Having obtained the integer value of this first ambiguity, the real-valued estimates of all remaining ambiguities are then corrected by virtue of their correlation with the first ambiguity. 
Then the second, but now corrected, real-valued ambiguity estimate is rounded to its nearest integer, and the process is then repeated again with both ambiguities held fixed, and the process is continued until all ambiguities are accommodated. 
Thus the bootstrapped estimator reduces to ’integer rounding’ in case correlations are absent.
%
\section{lambda}
\section{BIE}
\subsection{Melbourne-Wubbena linear combination}
The Melbourne-Wubbena linear combination (Melbourne 1985),(Wubbena 1985) is a linear combination of the L1 and L2 carrier phase plus the P1 and P2 pseudorange. The geometry, troposphere and ionosphere are eliminated by it. The Melbourne-Wubbena linear combination can be represented as:
%
\begin{equation}
E(L_{r,IF}^S) - \frac{cf_2z_{r,w}^s}{f_1^2 - f_2^2} = \rho_r^s + c(dt_{r,IF} - dt_{IF}^s) + \tau_r^s + \lambda_n z_{r,1}^s + (\lambda_{IF}\delta_{r,IF}
\end{equation}
%
Since ,,\% comprises of both code and phase measurements, it is reasonable to exclude the lower
elevation measurements to avoid the multipath impacts from the code observation. Normally, with
30 degree elevation cut-off, an averaging of 5 minutes of (4) is good enough to fixing the wide-lane
ambiguities [RD 04]. The rests are the wide-lane phase bias, which can be broadcasted to the user for
user side wide-lane ambiguity resolution. Either choosing a pivot receiver bias or a single-differencing
between two satellites can avoid the linear dependency. 
%
%
doesn't need lambda not as correlated
%
\section{Narrowlane and phase clock estimation}
%
highly correlated need lambda
% \[E(L^S) - \]
%
%
With the fixing of the wide-lane ambiguity, equation (1) and (2) can be further deducted as:
%
The code bias and phase bias in equations \eqref{obsEq} and (6) can be lumped into the corresponding receiver
and satellite clock errors. Then equations (5) and (6) become:
%
with:
%
By such an reformulation, there are two types of satellite clock: 1) IGS type clock, but estimated only
from code measurements; 2) phase clock, estimated using phase measurements and can be used to
support PPP ambiguity resolution on the user side. The drawback of this approach is that there is no
precise IGS compatible clock after the processing and it has to be derived from the existing PEA
processing
%
\section{Narrowlane and phase bias estimation}
%
In this approach, equation (7) holds the same for the code measurements. However, in equation (8),
the phase biases are not lumped into the clocks, but the ambiguities. More specifically, equation (8)
can be written as:
%
In equation (9), the clocks are the same as the clocks provided in equation (7), which means the
precise IGS satellite clock can be estimated.
The narrow-lane integer ambiguities, as shown in equation (9), are linearly dependent on the phase
biases, which means they cannot be estimated simultaneously. However, similar as the wide-lane
ambiguity resolution, the narrow-lane integer ambiguities can be fixed by rounding the float solution
to the nearest integer, and the remaining will be the phase bias, which can be broadcasted to the
user for user side PPP ambiguity resolution.
%
\textit{Bootstrapping} is designed to do some stuff.
%
\textit{Rounding}
%
\textit{Lambda}
%
Decorrelation algorithm
%
%2 nearest
%
\section{References on Ambiguity Resolution}
%
\begin{itemize}
    \item A modified phase clock/bias model to improve PPP ambiguity resolution at Wuhan University Journal of Geodesy - Geng et al. (2019)
    \item On the interoperability of IGS products for precise point positioning with ambiguity resolution Journal of Geodesy - Simon et al. (2020)
    \item Resolution of GPS carrier-phase ambiguities in precise point positioning (PPP) with daily observations Journal of Geodesy - Ge et al. (2008)
    \item Real time zero-difference ambiguities fixing and absolute RTK ION NTM - Laurichesse et al. (2008)
    \item Undifferenced GPS ambiguity resolution using the decoupled clock model and ambiguity datum fixing Navigation – Collins et al. (2010)
    \item Improving the estimation of fractional-cycle biases for ambiguity resolution in precise point positioning Journal of Geodesy – Geng (2012).
    \item Modeling and quality control for reliable precise point positioning integer ambiguity resolution with GNSS modernization
GPS Solutions – Li et al. (2014).
\end{itemize}
%
%
%\include{styleguide}
%%

%\part{Using the software}
%\chapter{Processing in end user mode}
When set to end user mode, the PEA component of Ginan will process each station separately. This mode will allow the estimation of parameters avaialable to users with single receivers. Expected application of this processing mode are
\begin{itemize}
	\item Estimate the position of a static point
	\item Track the path of a moving vehicle
	\item Monitor atmospheric conditions around the receiver
	\item Sychronise a locak clock
\end{itemize}
The end user  mode of the PEA is based on precise point positioning technique

\textit{Precise Point Positioning}(PPP) is a GNSS positioning method, originally developed, for calculating location of autonomous receivers with high levels of accuracy and precision. PPP aims to calculate the end user position by rigorously modelling and/or estimating error sources in GNSS measurements. 


Parameters that can potentially be estimated in end user mode are:
\begin{itemize}
	\item Antenna position
	\item Receiver clock offset
	\item Tropospheric delay at receiver location
	\item Ionospheric delay at the receiver location (not yet available)
	\item Carrier phase ambiguities
\end{itemize}

\section{PEA PPP Processing examples}
Examples of end user processing using the PEA follows. GINAN applications use YAML format to define configuration files. After installing the dependencies and compiling building the PEA application, the PEA processing can be started by typing the command 
\begin{verbatim}
$ ./pea --config <path_to_config_file>
\end{verbatim}

Details on the configuration parameters included in YAML files can be found in chapter \ref{pea_yaml_configuration}. The processing results for each receiver can be found on the trace files. The name of these output files can be set by the user, in the examples below they will typically be *.TRACE files.

\textbf{Receiver position}  results are preceded by the "\$POS" label and thus, in Linux, can be extracted using the command:
 \begin{verbatim}
$ grep "$POS" <path_to_trace_file>
\end{verbatim}
the output line for the for receiver position will have 10 comma separated fields with the following format:
\begin{verbatim}
$POS,2166,278015.000,6,-4052053.0060,4212836.8682,-2545105.0796,0.0245227,0.0231919,0.0163678
\end{verbatim}
the fields represent, from left to right:
\begin{enumerate}
	\item  "\$POS" label
	\item  GPS week
	\item  GPS TOW in seconds
	\item  Solution type (6 for float PPP, 1 for ambiguity fixed PPP)
	\item  Receiver ECEF X position in meters
	\item  Receiver ECEF Y position in meters
	\item  Receiver ECEF Z position in meters
	\item  Standard deviation of ECEF X positions in meters
	\item  Standard deviation of ECEF X positions in meters
	\item  Standard deviation of ECEF X positions in meters
\end{enumerate}

\textbf{Receiver clock} offset results are preceded by the "\$CLK" label and thus, in Linux, can be extracted using the command:
 \begin{verbatim}
$ grep "$CLK" <path_to_trace_file>
\end{verbatim}
the output line for the for receiver position will have 13 comma separated fields with the following format:
\begin{verbatim}
$CLK,2166,278015.000,6,14,3.1902,0.0000,1.1924,0.0000,0.0860,0.0000,0.0953,0.0000
\end{verbatim}
the fields represent, from left to right:
\begin{enumerate}
	\item  "\$CLK" label
	\item  GPS week
	\item  GPS TOW in seconds
	\item  Solution type (6 for float PPP, 1 for ambiguity fixed PPP)
	\item  Number of satellites used in the solution
	\item  Receiver clock offset for with respect to GPS clock, in nanoseconds
	\item  Receiver clock offset for with respect to GLONASS clock, in nanoseconds
	\item  Receiver clock offset for with respect to Galileo clock, in nanoseconds
	\item  Receiver clock offset for with respect to Beidou clock, in nanoseconds	
	\item  Standard deviation of clock offset wrt. GPS, in nanoseconds
	\item  Standard deviation of clock offset wrt. GLONASS, in nanoseconds
	\item  Standard deviation of clock offset wrt. Galileo, in nanoseconds
	\item  Standard deviation of clock offset wrt. Beidou, in nanoseconds	
\end{enumerate}
If clock offsets for a particular constellation are not available both the offset and its variance will be set to 0.


\textbf{Tropospheric delays} at the receiver position are preceded by the "\$TROP" label and thus, in Linux, can be extracted using the command:
 \begin{verbatim}
$ grep "$TROP" <path_to_trace_file>
\end{verbatim}
the tropospheric delay solutions will be represented to either a single line, with the "\$TROP" or three lines, as follows:
\begin{verbatim}
$TROP,2166,278015.000,6,14,2.294950,0.0030977
$TROP_N,2166,278015.000,6,14,-0.174797,0.0181385
$TROP_E,2166,278015.000,6,14,-0.223868,0.0250276
\end{verbatim}
each of the troposphere output line will contain 7 comma separated field, of which the first five are:
\begin{enumerate}
	\item  Label, "\$TROP", "\$TROP\_N" or "\$TROP\_E"
	\item  GPS week
	\item  GPS TOW in seconds
	\item  Solution type (6 for float PPP, 1 for ambiguity fixed PPP)
	\item  Number of satellites used in the solution
\end{enumerate}
The line starting with "\$TROP" contain the Zenith Tropospheric Delay (ZTD) and its standards deviation, both in meters, as their last two fields.  The line starting with "\$TROP\_N" contains the tropospheric delay gradient in north-south direction, and  the line starting with "\$TROP\_E" contains the tropospheric delay gradient in east-west direction.

Configuration files for specific examples have been added to the \textit{examples} folder in the repository. Examples corresponding to end user processing are explained bellow.

\subsection{Post processed PPP solutions with floating ambiguities using IF combination}
The configuration file named \textit{examples/ex11_pea_pp_user_gps.yaml} sets the PEA to calculate a post-process end user solution for a static receiver. 
 
In this example we will process 24 hours of data from a permanent reference frame station. The algorithm will use an L1+L2 and L1+L5 ionosphere-free combination.
The log files and processing results can be found in `<path to pea>/output/exs/EX01\_IF/`.
\begin{lstlisting}
$ ./pea --config ../../config/EX01-IF-PPP.yaml
\end{lstlisting}
The pea will then have the following output in <path to pea>/output/exs/EX01\_IF/ :

EX01\_IF20624.snx              - contains the station position estimates in SINEX format
EX01\_IF-ALIC2019199900.TRACE  - contains the logging information from the processing run

\begin{lstlisting}[language=bash]
$ grep "REC_POS" /data/acs/pea/output/exs/EX01_IF/EX01_IF-ALIC201919900.TRACE > ALIC_201919900.PPP
\end{lstlisting}
This will pipe all of the receiver position results reported in the station trace file to a seperate file for plotting.
\begin{lstlisting}[language=bash]
$ python3 /data/acs/pea/python/source/pppPlot.py --ppp /data/acs/pea/output/exs/EX01_IF/ALIC_201919900.PPP
\end{lstlisting}
This will then create the plots alic\_pos.png, a time series of the difference between the estimated receiver position and the median estimated position.
And the plot alic\_snx\_pos.png, a time series of the difference between the estimated receiver position and the IGS SINEX solution for Alic Springs on this day.

\subsection{Single Frequency Processing} 
\begin{lstlisting}[language=bash]
$ ./pea --config ../../config/Ex01-SF-PPP.yaml

$ grep "\$POS" /data/acs/pea/output/exs/EX01_SF/EX01_SF-ALBY202011500.TRACE
\end{lstlisting}
And you should see something similar to the following:
\begin{lstlisting}[language=bash]
$POS,2062,431940.000,0,-4052052.7956,4212836.0144,-2545104.6423,0.00000043966020,...
$POS,2062,431970.000,0,-4052052.7956,4212836.0144,-2545104.6423,0.00000043965772,...
\end{lstlisting}



\subsection{Processing realtime}
To process a continuous GPS station in real-time you will need to access the data stream from a NTRIP stream and the correction products from a NTRIPCaster.
Geoscience Australia is running a caster that provides global data stream, a dense network of stream covering the Australian region, and correction procdust provide by IGS Analysis Centres.
You will need to apply for an AUSCORS account, or use the new NTRIPCaster that streams using https.

You can apply for an account at the following link : \href{https://gnss-users-prod.auth.ap-southeast-2.amazoncognito.com/login?response_type=code&client_id=11njl767q0tl1faf9qna469vl1&redirect_uri=https://search-gnss-elasticsearch-prod-5omhch5quzlu5dcpbct4ev5qz4.ap-southeast-2.es.amazonaws.com/_plugin/kibana/app/kibana&state=e36b2054-7ace-4931-91a3-5ba6de893917}{New GA Account link}

Once your have your account and password you can test that you can successfuly connect to the NTRIPCaster by using the following curl command:
\begin{lstlisting}[language=bash]
$ curl https://ntrip.data.gnss.ga.gov.au/MOBS00AUS0 --http0.9 -i -u'<username>:<password> --output -'
\end{lstlisting}


%
\section{Processing a Global Network to obtain satellite clock products}
In this example 24 hours of data from a small global network of 87 stations is processed to obtain the clock products needed for high precision positioning.

Check that the paths in the configuration file for the products and RINEX files are correct for your system. If you have followed the convention layed out in the INSTALL.md document you should not need to amend anything.

To start the processing use the command:
\begin{lstlisting}
$ ./pea --config ../../config/Ex02-Network.yaml
\end{lstlisting}
The process will take approximatelly 2-3 hours to complete depending on CPU performance. The log files and processing results can be found in /data/acs/pea/output/examples/Ex02, or the alternative directory you have specified in the configuration file.

Change into your output directory. You should find a .TRACE file for each station processed, and a PEA.SUM file.
\begin{lstlisting}
$ cd /data/acs/pea/output/examples/Ex02
\end{lstlisting}
To verify your solution, first grep for the xp values:
\begin{lstlisting}
$ grep 'network xp' netprocessing.out > xp_test.txt
\end{lstlisting}
and then run:
\begin{lstlisting}
$ python3 /data/acs/pea/python/src/comprun.py --test /data/acs/output/xp_test.txt --standard /data/acs/pea/example/EX02/standard/xp_standard.txt
\end{lstlisting}
This will produce the plots Posdiff.png, recclk.png, satclk.png, and zwd.png.

\section{Example 03 Processing a Global Network to obtain the orbit and clock products}
In this example 24 hours of data from a small global network of 87 stations is processed to obtain the orbit and clock products needed for high precision positioning.

Check that the paths in the configuration file for the products and RINEX files are correct for your system. If you have followed the convention layed out in the INSTALL.md document you should not need to amend anything.

To start the processing use the command:
\begin{lstlisting}
$ ./pea --config ../../config/Ex03-Network_Orbits.yaml
\end{lstlisting}
The process will take approximatelly 2-3 hours to complete depending on CPU performance. The log files and processing results can be found in /data/acs/pea/output/examples/Ex03, or the alternative directory you have specified in the configuration file.

Change into your output directory. You should find a .TRACE file for each station processed, and a PEA.SUM file.
\begin{lstlisting}
$ cd /data/acs/pea/output/examples/Ex03
\end{lstlisting}
To verify your solution, first grep for the xp values:
\begin{lstlisting}
$ grep 'network xp' netprocessing.out > xp_test.txt
\end{lstlisting}
and then run:
\begin{lstlisting}
$ python3 /data/acs/pea/python/src/comprun.py --test /data/acs/output/xp_test --standard /data/acs/pea/example/EX03/standard/xp_standard.txt
\end{lstlisting}
This will produce the plots Posdiff.png, recclk.png, satclk.png, and zwd.png.

To compare the satellite clocks run:
\begin{lstlisting}
$ python3 /data/acs/pea/python/src/compareclk.py --standard /data/acs/pea/example/EX03/standard/aus20624.clk  --test ./aus20624.clk
\end{lstlisting}
This will produce plots for the differences in satellite clocks G02 through to G32 as well as calculating the RMS and standard deviation with respect to the standard. G01.png does not exists at this has been used as the pivot satellite clock that we use to remove the bias from.

\section{Example 04 Processing a Global Network to obtain Ionospheric Vertical Total Electron Content (VTEC) Maps}
In this example 24 hours of data from a small global network of 87 stations is processed to obtain the IONEX formatted Ionosphere VTEC maps and SINEX formatted satellite Diferential Signal Biases (DSB). 
Ionospheric VTEC maps follows the IONEX 1.1 format, which can be found in: https://gssc.esa.int/wp-content/uploads/2018/07/ionex11.pdf 
The satellite bias follows the bias-SINEX format, which can be found in: http://ftp.aiub.unibe.ch/bcwg/format/draft/sinex\_bias\_100\_dec07.pdf

Check that the paths in the configuration file for the products and RINEX files are correct for your system. 
If you have followed the convention layed out in the INSTALL.md document you should not need to amend anything.

To start the processing use the command:
\begin{lstlisting}
$ ./pea --config <installation directory>/pea/config/Ex04-Ionosphere.yaml
\end{lstlisting}

The process will take approximatelly 1-2 hours to complete depending on CPU performance and setting specifications.

The IONEX and bias-SINEX files can then be used to obtain SPP and PPP positioning solutions as specified in PPPExamples.md

A utility to plot the TEC values in an IONEX file have can be found at:

<installation directory>/pea/python/source/plotIONEX.py
This utility will ask for the path/filename of the IONEX file and a start and stop time in hhmmss format. The Python code will then generate a figure in IONEX\_yyyy-mm-dd\_hh:mm:ss.png format for each entry between the start and stop time.

\section{Network Post-Processing - Ultra-rapid product example}

\newthought{We will} run the pea in a post-processing mode, along side the POD, to produce orbits and clock in near-realtime, as would be run to produce an IGS ultra-rapid product

% Should move this to main?
% taken directly from https://www.overleaf.com/learn/latex/LaTeX_Graphics_using_TikZ:_A_Tutorial_for_Beginners_(Part_3)%E2%80%94Creating_Flowcharts
%\tikzstyle{startstop} = [rectangle, rounded corners, minimum width=3cm, minimum height=1cm,text centered, draw=black, fill=red!30]
%\tikzstyle{io} = [trapezium, trapezium left angle=70, trapezium right angle=110, minimum width=3cm, minimum height=1cm, text centered, draw=black, fill=blue!30]
%\tikzstyle{process} = [rectangle, minimum width=3cm, minimum height=1cm, text centered, draw=black, fill=orange!30]
%\tikzstyle{decision} = [diamond, minimum width=3cm, minimum height=1cm, text centered, draw=black, fill=green!30]
%\tikzstyle{arrow} = [thick,->,>=stealth]


%\begin{tikzpicture}[node distance=2cm]
%\node (start) [startstop] {POD};
%\node (in1) [io, below of=start] {Orbit Partial file}; 
%\node (in2) [io, right of=in1, xshift=3cm] {RINEX data};
%\node (pro1) [process, below of=in1] {PEA};
%\node (in3) [io, below of=pro1] {Clock File, SINEX, 
%                                 Updated Orbit Partial};
%\node (proc2) [process, below of=in3] {POD}
%\node{in4) [io. below of=proc2]{SP3 file}
% Now draw some linking arrows
%\draw [arrow] (start) -- (in1);
%\draw [arrow] (in1) -- (pro1);
%\draw [arrow] (in2) -- (pro1);
%\draw [arrow] (pro1) -- (in3);
%\end{tikzpicture}


\section{Network Real-time Processing}

\newthought{The PEA} is designed to do some stuff.

% Move this to products and data description page.
ANTEX files can be downloaded from https://files.igs.org/pub/station/general/igs14.atx
DCB files can be downloaded from ftp://ftp.aiub.unibe.ch/CODE/P1C1.DCB
Satellite metadata can be obtained from: https://files.igs.org/pub/station/igs\_satellite\_metadata.snx
%As explained in previous sections, the Ginan repository consist on two main components. The \textit{precise orbit deterination} (POD) component estimates precise satellite position and orbital parameters, while the \textit{parameter estimation algorithm} (PEA) monitor systematic biases asociated with GNSS signals. The Ginan software follows uses the \textit{precise point positioning}(PPP) philosophy for processing of GNSS signals. PPP was originally developed as a GNSS based positioning method for calculating location of autonomous receivers with high levels of accuracy and precision. PPP aims to calculate the end user position by rigorously modelling and/or estimating error sources in GNSS measurements. 
The systematic of errors in GNSS signals can be summarised as: 
\begin{itemize}
	\item Satellite state estimation errors: position, clock offset, hardware biases, antenna effects
	\item Receiver state estimation errors: position, clock offset, hardware biases, antenna effects
	\item Atmospheric effects: ionospheric propagation delay, tropospheric propagation delay
	\item Other (modellable) enviromental effects: Relativistic corrections, phase windup
\end{itemize}

The various components of the Ginan software package are designed to model or estimate these errors as parameters. A diagram illustrating of the way the Ginan components interact to estimate these parameters can be found in figure \ref{fig:PEAnPOD}. In the example illustrated by the figure:
\begin{enumerate}
	\item  The POD in orbit fitting mode is used to calculate an a-priori position and the linearization partials of orbit parameters  
	\item  The PEA, in network mode, estimate orbital parameters from orbit partials
	\item  The POD, use the orbital parameters to estimate and predict precise satellite positions
	\item  The PEA, in network mode, is used to estimate wide-area parameters: satellite clock offsets, satellite hardware bias and atmospheric delays 
	\item  The PEA, in end-user mode, is used to callculate local parameters like receiver position, receiver clock offset and local atmospheric delays 
\end{enumerate}

Other parameters, such as antenna, phase windup and relativistic effects are calculated from predefined models.

A description of each components and its use is presented below.\\

\chapter{Using the POD Module}
GINAN applications use YAML format to define configuration files. After installing the dependencies and compiling building the POD application, the POD processing can be started by typing the command.
\begin{lstlisting}
$ ./pod -y <path_to_config_file>
\end{lstlisting}
Details on the configuration parameters included in YAML files can be found in chapter \ref{ch:pod_yaml_configuration}. Configuration files corresponding to the examples in this section can be found in the \textit{ginan/examples} directory.\\

The POD module has two main modes of operation, the orbit fitting mode and the orbit integration/prediction mode. In orbit fitting mode, precise orbit parameters are calculated from, potentially inaccurate, satellite position pseudo-observations. In orbit integration mode, precise satellite positions are estimated/predicted from precise orbit parameters.\\

\section{Using the POD for orbit fitting}
The orbit fitting mode can be selected by setting the \textit{ pod\_mode\_fit} to true and \textit{ic\_input\_format : sp3} to true. In this mode, the POD will take satellite position pseudo-measurments from a SP3 formatted file and estimate the orbit state of each satellite contained in the SP3 file. The SP3 file containing a priory satellite position needs to be specified as the \textit{pseudobs\_orbit\_filename} parameter.
The orbit state in POD is represented by a set of parameters consisting of 
\begin{itemize}
	\item Satellite position (in ITRF or TCRF) at the first epoch in the SP3 file
	\item Satellite velocity (in ITRF or TCRF) at the first epoch in the SP3 file
	\item Up to 9 parameters describing the Solar Radiation Pressure over the fitting time
\end{itemize}
These initial conditions, and the models described in chapter \ref{ch:observation_modelling} will allow for the precise determination of satellite positions over the fitting arch (set by the \textit{orbit\_arc\_determination} parameter).\\

The main outputs from this mode of operation are the a-posteriori satellite position in SP3 format, and the orbit partials of satellite positions with respect to the initial conditions. 
The ouput SP3 file which can be found on \textit{output\_directory/gagWWWWD.sp3} where \textit{WWWW} is the GPS week and /textit{D} is the GPS day of the first epoc on the SP3 files.
The orbit partials are written in Ginan's proprietary Initial Conditions File (ICF) format, and can be found in  \textit{output\_directory/gagWWWWD\_orbit\_partials.out}.
Configuration files, \textit{ex21\_pod\_fit\_gps} and \textit{ex21\_pod\_fit\_gnss},  for this mode of operation are included in Ginans \textit{examples} folder.\\

\section{using the POD for orbit integration/prediction}
The orbit fitting mode can be selected by setting the \textit{ pod\_mode\_ic\_int} to true and \textit{ic\_input\_format : icf} to true. 
 In this mode, the POD will take the initial conditions contained in the ICF formatted files and propagaes the satellite positions forward over the time period specified by the sum of \textit{orbit\_arc\_determination} and \textit{orbit\_arc\_prediction} parameters. 
 It also propagates the satellite position backwards by a number of hours specified by the \textit{orbit\_arc\_backwards} parameter.
 The ICF file containing the satellites initial condition and radiation pressure parametes needs to be specified as the \textit{ic\_input\_format : ic\_filename} parameter.\\

It is to note that the orbit fitting mode will also use the orbit integration operation after estimating the initial conditions from pseudo-observations. 
Although the  mode \textit{ pod\_mode\_fit} will only integrated for a number of hours specified  by \textit{orbit\_arc\_determination}.
Selecting the \textit{pod\_mode\_predict} will propagate the initial conditions a number of hours specified by the sum of \textit{orbit\_arc\_determination} and \textit{orbit\_arc\_prediction}
The integrated/predicted satellite position will be output to a SP3 formatted file located in \textit{output\_directory/gagWWWWD.sp3}.\\

The configuration file to perform orbit integration/prediction from SP3 files is \textit{ex23\_pod\_prd\_gps}.  
The configuration file to perform orbit integration/prediction from ICF files is \textit{ex24\_pod\_ic\_gps}. Both located in Ginans \textit{examples} folder.\\


%
%\part is not working?
%\chapter{Using the Source Code}
\chapter{Coding Standards}
\label{ch:coding_standards}

\newthought{Coding Standards} for C++

\section{Code style}
Overall we are aiming for
\begin{itemize}
	\item  Write for clarity
	\item  Write for clarity
	\item  Use short, descriptive variable names
	\item  Use aliases to reduce clutter.
\end{itemize}

\subsection{Bad}
\begin{verbatim}

```cpp

    //check first letter of satellite type against stomething

    if (obs.Sat.id().c_str()[0]) == 'G') 
        doSomething(); 
    else if (obs.Sat.id().c_str()[0]) == 'R')
        doSomething();
    else if (obs.Sat.id().c_str()[0]) == 'E')
        doSomething();
    else if (obs.Sat.id().c_str()[0]) == 'I')
        doSomething();
```
\end{verbatim}

\subsection{Good}

\begin{verbatim}
```cpp

    char& sysChar = obs.Sat.id().c_str()[0];

    switch (sysChar)
    {
        case 'G':   doSomething();   break;
        case 'R':   doSomething();   break;
        case 'E':   doSomething();   break;
        case 'I':   doSomething();   break;
    }
```
\end{verbatim}


\section{Spacing, Indentation, and layout}

\begin{itemize}
	\item  Use tabs, with tab spacing set to 4.	
	\item  Use space or tabs before and after any \begin{verbatim} + - * / = < > == != % etc.\end{verbatim}
	\item  Use space, tab or new line after any , ;
	\item  Use a new line after if statements.
	\item  Use tabs to keep things tidy - If the same function is called multiple times with different parameters, the parameters should line up.
\end{itemize}

\subsection{Bad}

\begin{verbatim}
```cpp

    trySetFromYaml(mongo_metadata,output_files,{"mongo_metadata" });
    trySetFromYaml(mongo_output_measurements,output_files,{"mongo_output_measurements" });
    trySetFromYaml(mongo_states,output_files,{"mongo_states" });
```
\end{verbatim}

\subsection{Good}
\begin{verbatim}
```cpp

    trySetFromYaml(mongo_metadata,             output_files, {"mongo_metadata"              });
    trySetFromYaml(mongo_output_measurements,  output_files, {"mongo_output_measurements"	});
    trySetFromYaml(mongo_states,               output_files, {"mongo_states"		        });
```
\end{verbatim}

\section{Statements}

* One statement per line  - \*unless you have a very good reason

\subsection{Bad}
/begin{verbatim}
```cpp

    z[k]=ROUND(zb[k]); y=zb[k]-z[k]; step[k]=SGN(y);
```
/end{verbatim}

\subsection{Good}

\begin{verbatim}
```cpp

    z[k]    = ROUND(zb[k]);
    y       = zb[k]-z[k]; 
    step[k] = SGN(y);
```
\end{verbatim}

\subsection{Example of a good reason:}

* Multiple statements per line sometimes shows repetetive code more clearly, but put some spaces so the separation is clear.

\subsection{Normal}

\begin{verbatim}
```cpp

    switch (sysChar)
    {
        case ' ':
        case 'G': 
            *sys = E_Sys::GPS; 
            *tsys = TSYS_GPS; 
            break;
        case 'R': 
            *sys = E_Sys::GLO;  
            *tsys = TSYS_UTC; 
            break;
        case 'E': 
            *sys = E_Sys::GAL;  
            *tsys = TSYS_GAL; 
            break;
    //...continues
```
\end{verbatim}

\subsection{Ok}

\begin{verbatim}
```cpp

    if      (sys == SYS_GLO)    fact = EFACT_GLO;
	else if (sys == SYS_CMP)    fact = EFACT_CMP;
	else if (sys == SYS_GAL)    fact = EFACT_GAL;
	else if (sys == SYS_SBS)    fact = EFACT_SBS;
	else                        fact = EFACT_GPS;
```
\end{verbatim}

\subsection{Ok}	

\begin{verbatim}
```cpp

    switch (sysChar)
    {
        case ' ':
        case 'G':   *sys = E_Sys::GPS;      *tsys = TSYS_GPS;    break;
        case 'R':   *sys = E_Sys::GLO;      *tsys = TSYS_UTC;    break;
        case 'E':   *sys = E_Sys::GAL;      *tsys = TSYS_GAL;    break;
        case 'S':   *sys = E_Sys::SBS;      *tsys = TSYS_GPS;    break;
        case 'J':   *sys = E_Sys::QZS;      *tsys = TSYS_QZS;    break;
    //...continues
```
\end{verbatim}

\section{Braces}

New line for braces.
\begin{verbatim}
```cpp

    if (pass)
    {
        doSomething();
    }
```
\end{verbatim}

\section{Comments}

%\begin{itemize}
%\item Prefer `//` for comments within functions
%\item Use `/* */` only for temporary removal of blocks of code.
%\item Use `/** */` and `///<` for automatic documentation
%\end{itemize}

\section{Conditional checks}
%\begin{itemize}
%\item  Put `\&\&` and `||` at the beginning of lines when using multiple conditionals.
%\item  Always use curly braces when using multiple conditionals.
%\end{itemize}
\begin{verbatim}
```cpp

    if  ( ( testA     > 10)
        &&( testB   == false
          ||testC   == false))
    {
        //do something
    }
```
\end{verbatim}

* Use variables to name return values rather than using functions directly

\subsection{Bad}

\begin{verbatim}
```cpp

    if (doSomeParsing(someObject))
    {
        //code contingent on parsing success? failure?
    }
```
\end{verbatim}

\subsection{Good}
\begin{verbatim}
```cpp

    bool fail = doSomeParsing(someObject);
    if (fail)
    {
        //This code is clearly a response to a failure
    }
```
\end{verbatim}

\section{Variable declaration}

\begin{itemize}
\item Declare variables as late as possible - at point of first use.
\item One declaration per line.
\item Declare loop counters in loops where possible.
\item Always initialise variables at declaration.
\end{itemize}

\begin{verbatim}
```cpp

    int  type  = 0;
    bool found = false;         //these have to be declared early so they can be used after the for loop

    for (int i = 0; i < 10; i++)
    {
        bool pass = someTestFunction();    //this pass variable isnt declared until it's used - good
        if (pass)
        {
            type  = typeMap[i];
            found = true;
            break;
        }
    }

    if (found)
    {
        //...
    }    
```
\end{verbatim}

\section{Function parameters}

\begin{itemize}
\item One per line.
\item Add doxygen compatible documentation after parameters in the cpp file.
\item Prefer references rather than pointers unless unavoidable.
\end{itemize}

\begin{verbatim}
```cpp

    void function(
            bool        runTests,           ///< Run unit test while processing
            MyStruct&   myStruct,           ///< Structure to modify
            OtherStr*	otherStr = nullptr)	///< Optional structure object to populate (cant use reference because its optional)
    {
    	//...
    }
```
\end{verbatim}

\section{Naming and Structure}
%\begin{itemize}
%\item For structs/classes, use `CamelCase` with capital start
%\item For member variables, use `camelCase` with lowercase start
%\item For config parameters, use `lowercase_with_underscores`
%\item Use suffixes (`_ptr`, `_arr`, `Map`, `List` etc.) to describe the type of container for complex types
%\item Be sure to provide default values for member variables.
%\item Use heirarchical objects where applicable.
%\end{itemize}
\begin{verbatim}
```cpp

    struct SubStruct
    {
        int    type = 0;
        double val  = 0;
    };

    struct MyStruct
    {
        bool          memberVariable = false;
        double        precision      = 0.1;

        double                     offset_arr[10]  = {};
        OtherStruct*               refStruct_ptr   = nullptr;

        map<string, double>        offsetMap; 
        list<map<string, double>>  variationMapList;
        map<int, SubStruct>        subStructMap;
    };

    //...

    MyStruct myStruct = {};

    if (acsConfig.some_parameter)
    {
        //..
    }
```
\end{verbatim}
\section{Testing}
\begin{itemize}
	\item Use TestStack objects at top of each function that requires automatic unit testing.
	\item Use TestStack objects with descriptive strings in loops that wrap functions that require automatic unit testing.
\end{itemize}
\begin{verbatim}
```cpp

    void function()
    {
        TestStack ts(__FUNCTION__);

        //...

        for (auto& obs : obsList)
        {
            TestStack ts(obs.Sat.id());

            //...
        }
    }
```
\end{verbatim}
\section{Documentation}
\begin{itemize}
\item Use doxygen style documentation for function and struct headers and parameters
\item `/**`  for headers.
\item `///<` for parameters
\end{itemize}
\begin{verbatim}
```cpp

    /** Struct to demonstrate documentation.
    * The first line automatically gets parsed as a brief description, but more detailed descriptions are possible too.
    */
    struct MyStruct
    {
        bool    dummyBool;                  ///< The thing to the left is documented here
    };

    /** Function to demonstrate documentation
    */
    void function(
            bool        runTests,           ///< Run unit test while processing
            MyStruct&   myStruct,           ///< Structure to modify
            OtherStr*	otherStr = nullptr)	///< Optional string to populate
    {
    	//...
    }
```
\end{verbatim}

\section{STL Templates}
\begin{itemize}
\item Prefer maps rather than fixed arrays.
\item Prefer range-based loops rather than iterators or `i` loops, unless unavoidable.
\end{itemize}

\subsection{Bad}
\begin{verbatim}

```cpp

    double double_arr[10] = {};

    //..(Populate array)

    for (int i = 0; i < 10; i++)    //Magic number 10 - bad.
    {

    }
```
\end{verbatim}
%
\begin{verbatim}
```cpp

    map<string, double> doubleMap;

    //..(Populate Map)

    for (auto iter = doubleMap.begin(); iter != doubleMap.end(); iter++)   //long, undescriptive - bad
    {
    	if (iter->first == someVar)     //'first' is undescriptive - bad
    	{
    		//..
    	}
    }
```
\end{verbatim}
\subsection{Good - Iterating Maps}
\begin{verbatim}
```cpp

    map<string, double> offsetMap;

    //..(Populate Map)

    for (auto& [siteName, offset] : doubleMap)	//give readable names to map keys and values
    {
        if (siteName.empty() == false)
        {
        
        }
    }
```
\end{verbatim}
\subsection{ Good - Iterating Lists}
\begin{verbatim}
```cpp

    list<Obs> obsList;

    //..(Populate list)

    for (auto& obs : obsList)         //give readable names to list elements
    {
        doSomethingWithObs(obs);
    }
```
\end{verbatim}

\section{Special Case - Deleting from maps/lists}

Use iterators when you need to delete from STL containers:
\begin{verbatim}
```cpp

    for (auto it = someMap.begin(); it != someMap.end();  )
    {
        KFKey key = it->first;				//give some alias to the key/value so they're readable

        if (measuredStates[key] == false)
        {
	    	it = someMap.erase(it);
    	}
	    else
	    {
	    	++it;
    	}
    }
```
\end{verbatim}
\section{Namespaces}

Commonly used std containers may be included with `using`
\begin{verbatim}
```cpp
    #include <string>
    #include <map>
    #include <list>
    #include <unordered_map>
    
    using std::string;
    using std::map;
    using std::list
    using std::unordered_map;
```
\end{verbatim}
%\include{repositories}
%\chapter{Overview of the PEA}


The software execution will be largely sequential, using threads sparingly to limit the amount of collision avoidance overhead. Where possible though, tasks will be completed in parallel utilising parallelisation libraries to take advantage of all cpu cores in multi-processor systems while still retaining a linear flow through the execution.

Sections of the software that create and modify global objects, such as while reading ephemeris data, will be executed on a single core only. This will ensure that there are no collisions that need to be checked for, and debugging of these functions is deterministic.

For sections of the software that have clear delineation between objects, such as per-receiver calculations, these may be completed in parallel, provided they do not attempt to modify or create objects with more global scope. When globally accessible objects need to be created for individual receivers, they should be pre-initialised before the entry to parallel execution section.


\section{Data Input and Synchronisation}

Before the processing of data from an epoch is initiated, all other relevant data is accumulated. As this code affects global objects that have effects in multiple places, this code is run in a single thread until data processing is ready to begin.


\subsection{Config}

Configurations are defined in .yaml files. At the beginning of each epoch the timestamp of the configuration file is read, and if there has been a modification, new parameters in the configuration will be loaded into memory.


\subsection{Product Input}

Various external products may be required for operation of the software, as defined in the configuration file. At the beginning of each epoch, if any product input files have been added to the config, or if the inputs are detected to have been modified, they will be re-read into memory.


\subsection{Metadata Input}

Metadata such as GNSS ephemerides are available from external sources to augment the capability of the software. This data is ingested at the beginning of each epoch before processing begins.


\subsection{Observation Data Input}

Observation data forms the basis for operation of the software. Observations from various sources are synchronised and collated at the beginning of each epoch before processing begins. The software uses class inheritance and polymorphism such that all data type inputs are retrieved using a single common instruction, with backend functions performing any retrieval and parsing required.

Observation data is syncronised by timestamp - when the main function requests data of a specific timestamp all data until that point is parsed (but may be discarded), before the observation data corresponding to that timestamp being used in processing.


\subsection{Initialisation of Objects}

During the following stages of processing many receiver-specific objects may be created within global objects. To prevent thread collision in the global objects, the receiver-specific objects are created here sequentially.


\section{Preprocessor}

The preprocessor is run on input data to detect the anomolies and other metrics that are available before complete processing of the data is performed. This enables low-latency reporting of issues, and prepares data for more efficient processing in the later stages of operation.

\subsection{Cycle Slip Detection}
\subsection{Low Latency Anomaly Detection}
\subsection{Missing Data}
\subsection{Output / Reporting}

\section{Precise Point Positioning}

The largest component of the software, the PPP module ingests all of the data available, and applies scientific models to estimate and predict current and future parameters of interest.

Version 1 of the GINAN toolkit satisfies many of the requirements for GNSS modelling, but has been achieved by incrementally adding features as they became available and as scientific models have been developed. Many of the components make assumptions about the outputs of previous computations performed in the software, and require care before adding or making changes to the code, or even setting configuration options.

It is intended that the software will be reorganised with the benefit of hindsight, to remove interactions between modules and explicitly execute each processing step in a manner similar to an algebraic formulation used by experts in the field.

It is tempting for researchers to apply heuristics or corrections that may have been historically used to assist in computation, but these must be limited to effects that can be modelled and applied through the kalman filter, in order to maintain the optimality and robustness that it provides.

As the models required for ‘user’, ‘network’, and even ‘ionosphere’ modes are equivalent, the only distinction between the modes is the extent of modelling to be applied, which can be reduced to a simple configuration change. As such the parallel streams within the software will be eliminated and reduced to a single unified model, with example configurations for common use-cases.

\subsection{Force and Dynamic Models}

At the beginning of processing of an epoch, parameters with time-dependent models are updated to reflect the time increment since the previous epoch. Simple models will be well defined when initialised, but more complex models will require updating at every epoch.

Ultimate positioning performance largely depends on accurate dynamic models, with development of these models improving predictive capability, and reducing uncertainty and adjustments at every point in time.

\subsubsection{Gravity}
\subsubsection{Solar Radiation Pressure}
\subsubsection{Other}
\subsubsection{Orbit and State Prediction}

Before the available observations for an epoch are utilised, a prediction is made of the parameters of interest by utilising the previous estimates and applying dynamic models through the Kalman filter’s state transition.


\subsection{Phenomena Modelling + Estimation}

In order to accurately estimate and predict parameters of interest, all phenomena that affect GNSS/SLR observations must be isolated and modelled, as being components comprising the available measurements.

Where the values of parameters are well known they may be used directly to extract other parameters of interest from the data - such as using published corrections to precisely determine a user’s position.

When data is unavailable, or when it is desired to compute these products for subsequent publication and use, estimates of the values are derived from the available data.

It is the sophistication of the models available and applied that determines the ultimate performance of the software.

The software will be developed to allow for all applicable phenomena to be modelled, estimated, such that user’s desired constraints may be applied and parameters of interest extracted.


\subsection{Initialisation of Parameters}

Estimation parameters are initialised on the point of first use, automatically by the kalman filter module. Their initial value may be selected to be user-defined, extracted from a model or input file, or established using a least-squares estimation.


\subsection{Robust Kalman Filter}

It is well known that the Kalman filter is the optimal technique for estimating parameters of interest from sets of noisy data - provided the model is appropriate.

In addition, statistical techniques may be used to detect defects in models or the parameters used to characterise the data, providing opportunities to intervene and make corrections to the model according to the nature of the anomaly.

By incorporating these features into a single generic module, the robustness that was previously available only under certain circumstances may now be automatically applied to all systems to which it is applied. These benefits extend automatically to all related modules (such as RTS), and often perform better than modules designed specifically to address isolated issues.
Initialisation

For further details about the software's robust kalman filter see chapter \ref{ch:kalman_filter}


\subsection{RTS Smoothing}

The intermediate outputs of a kalman filter are of use for other algorithms such as RTS smoothers. All intermediate values required for such algorithms are to be recorded in a consistent manner, suitable for later processing.


\subsection{Integer Ambiguity Resolution}

GNSS phase measurements allow for very precise measurements of biases but require extra processing steps to disambiguate between cycles. Techniques have been demonstrated that perform acceptably under certain conditions and measurement types, but require substantial bookkeeping and may not easily transfer to different measurement applications.
Product calculation

In order for estimated and predicted values to be of use to end-users, they must be prepared and distributed in an appropriate format.

Some parameters of interest are not directly estimated by the filter, but may be derived from estimates by secondary operations, which are performed in this section of the code.

In this section, data is written to files or pushed to NTRIP casters and other data sinks.


\section{Post-processing}


\subsection{Smoothing}

The RTS Smoothing algorithm is capable of using intermediate states, covariances, and state transition matrices stored during the Kalman filter stage to calculate reverse smoothed estimates of parameters.

The intermediate data is stored in binary files with messages that contain tail blocks containing the length of the message. This allows for the file to be efficiently traversed in reverse; seeking to the beginning of each message as defined by the tail block.

For further details about the software's RTS Smoothing algorithm see chapter \ref{ch:RTS}


\subsection{Minimum Constraints}

The minimum constraints algorithm is capable of aligning a network of stations to a reference system without introducing any bias to the positions of the stations.

A subset of stations positions are selected and weighted to create pseudo-observations to determine the optimal rigid transformation between the coordinates and the reference frame. The transformation takes the same algebraic form as a kalman filter stage and is implemented as such in the software.

For further details about the software's RTS Smoothing algorithm see chapter \ref{ch:MinCon}


\subsection{Unit Testing}

The nature of GNSS processing means that well-defined unit tests are difficult to write from first-principles. The software however, is capable of comparing results between runs to determine if the results have changed unexpectedly.

Intermediate variables are tagged throughout the code, and auxiliary files specify which variables should be tested as they are obtained, and the expected values from previous runs.


\subsection{Logging}

Details of processing are logged to trace files according to the processing mode in use.

Per-station files are created with intermediate processing values and information, while a single summary file is generated for the unified filter and combined processing.

Information produced during processing is output to the console, and may also be redirected to other logging sinks such as a database, or json formatted output.

In addition to processing information, inputs may be recorded to file for replaying later.
 
 

 
%\include{Overview_of_the_POD}
\chapter{Equation Conventions}
\label{ch:conventions}

\textit{In this manual we will} be adhering to the following conventions

\section{List of Symbols}

\begin{itemize}
	\item {\boldmath{$i$}} Receiver identification or {\boldmath{$r$}}
	\item {\boldmath{$j$}} Satellite identification or {\boldmath{$s$}}
	\item {\boldmath{$k$}} Epoch number or {\boldmath{$t$}}
	\item {\boldmath{$q$}} GNSS type (GPS,GALILEO,GLONASS,QZSS)
	\item {\boldmath{$c$}} Speed of light [m/s]
	\item {\boldmath{$x$}} Vector of parameters to be estimated
	\item {\boldmath{$y$}} Vector of observations
	\item {\boldmath{$v$}} Vector of residuals
	\item {\boldmath{$H$}} Design matrix
	\item {\boldmath{$\sigma$}} Standard deviation of observable
	\item {\boldmath{$\Delta$}} Increment to a priori values [m]
	\item {\boldmath{$\lambda$}} Wavelength or {\boldmath{$\lambda_1,\lambda_2,\lambda_5$}}
	\item {\boldmath{$f_1,f_2,f_5$}} frequency
	\item {\boldmath{$\alpha$}} Ambguity or {\boldmath{$N$}} Real valued ambguity and {\boldmath{$\bar{N}$}} Integer part of real valued ambguity
	\item {\boldmath{$\alpha$}} level of significance
	
	\item {\boldmath{$\beta$}} Biases
	\item {\boldmath{$\zeta$}} Clock offsets
	\item {\boldmath{$\delta t$}} Clock error [s]
	\item {\boldmath{$\kappa$}} Correction - relativity
	\item {\boldmath{$\iota$}} Ionosphere or  {\boldmath{$I$}}
	\item {\boldmath{$\tau$}} Troposphere or {\boldmath{$T,T_h,T_w$}}
	\item {\boldmath{$M$}} elevation dependent mapping function for the troposphere wet delay
	\item {\boldmath{$\xi$}} Phase wind-up error
	\item {\boldmath{$\epsilon$}} Error in observations and unmodelled effects [m]
	\item {\boldmath{$\phi_i^j$}} Carrier phase observable (times c) [m]
	\item {\boldmath{$P_i^j$}} Pseudo range observable [m]
		
\end{itemize}

Lets try this for example:
For an undifference, uncombined float solution, the linearized observation equations for pseudorange and phase observations from satellite $s$ to receiver $r$ can be described as:

\begin{math}
\Delta P_{r,f}^{q,s} = u_r^{q,s} . \Delta x + c . (\delta t_r^q - \delta t^{q,s}) + M_r^{q,s} . T_r + \gamma_f^q . I_{r,1}^{q,s} + d_{r,f}^q - d_f^{q,s} + \epsilon_{P,f}^q
\end{math}\\
\begin{math}
\Delta\phi_{r,f}^{q,s} = u_r^{q,s} . \Delta x + c . (\delta t_r^q - \delta t^{q,s}) + M_r^{q,s} . T_r - \gamma_f^q . I_{r,1}^{q,s} + \lambda_f^q . N_{r,f}^{q,s} + b_{r,f}^q - b_f^{q,s} + \epsilon_{L,f}^q 
\end{math}

where$\Delta P_{r,f}^{q,s}$ and $\Delta\phi_{r,f}^{q,s}$ are the respective pseudorange and phase measurements on the frequency $f$(f=1,2), from which the computed values are removed;
$u_r^{q,s}$ is the receiver-to-satellite unit vector;
$\Delta x$ is the vector of the receiver position corrections to its preliminary position; 
$\delta t_r^q$ and $\delta t^{q,s}$ are the receiver and satellite clock errors respectively;
$c$ is the speed of light in a vaccum
$M_r^{q,s}$ is the elevation dependent mapping function for the troposphere wet delay from the corresponding zenith one $T_r$;
$I_{r,1}^{q,s}$ is the ionosphere delay along the line-of-sight from a receiver to a satellite at the first frequency and $\gamma_f^q = (\lambda_f^q / \lambda_1^q)^2$;
$\lambda_f^q$ is the wavelength for the frequency $f$ of a GNSS $q$;
$N_{r,f}^{q,s}$is the phase ambguity 
$d_{r,f}^q$ and $b_{r,f}^q$ are the receiver hardware delays of code and phase observations respectively;
$d_f^{q,s}$ and $b_f^{q,s}$ are the satellite hardware delays of code and phase observations, respectively;
$\epsilon_{P,f}$ and $\epsilon_{L,f}$ are the code and phase measurement noises respectively. 


%
%\clearpage
Some random reference \gls{ambguity}.
Some random reference \gls{wavelength}.
Some random reference \gls{gps}.
%\printglossary[title=Terms Used, toctitle=List of Terms]
%\printglossary[type=\acronymtype]
% The back matter contains appendices, bibliographies, indices, glossaries, etc.
\backmatter
\printglossaries
%\printglossaries[type=\acronymtype]
%\printglossaries%[type=\acronymtype]
%\printglossaries
%
%\part{YAML Configuration File Reference}
%
%\include{Examples in Detail}












\chapter{PEA Configuration File - YAML}
	
The PEA processing engine uses a single yaml file for configuration of all processing options.

\section{YAML Syntax}
The yaml format allows for heirarchical, self descriptive configurations of parameters, and has a straightforward syntax.

White-space (indentation) is used to specify heirarchies, with each level typically indented with 4 space characters.

Colons (:) are used to separate configuration keys from their values.

Lists may be created by either appending multiple values on a single line, wrapped in square brackets and separated by commas, or, by adding each value on a separate indented line with a dash before the value.

Adding a hash symbol (\#) to a line will render the remainder of the line as a comment to be ignored by the parser.

Strings with special characters or spaces should be wrapped in quotation marks.

You will see all of these used in the example configuration files, but the files may be re-ordered, or re-formatted to suit your application.

\section{Default Values}

Many processing options have default values associated with them. To prevent repetition, and to ensure that the values are reported correctly, these values may be viewed in the acsConfig.hpp file within the source code directories.


\section{Globbing}
Files may be specified individually, as lists, or by searching available files using a globbed filename using the star character (*)

\section{Wildcard Tags}
Output filenames can include wildcards wrapped in \lstinline{< >} brackets to allow more generic names to be used. While processing, these tags are replaced with details gathered from processing, and allows for automatic generation of, for example, hourly output files.

\subsection*{\textless CONFIG\textgreater}
This is replaced with the 'config\_description' value entered in the yaml file.
\subsection*{\textless STATION\textgreater}
This is replaced with the 4 character station id of each station that generates a trace file.
\subsection*{\textless LOGTIME\textgreater}
This is replaced with the (rounded) time of the epochs within the trace file.

If trace file rotation is configured for 1 hour, the \lstinline{<LOGTIME>} wildcard will be rounded down to the closest hour, and subsequently change value once per hour and generate a separate output file for each hour of processing.

\subsection*
{\textless DDD\textgreater,
\textless D\textgreater, 
\textless WWWW\textgreater, 
\textless YYYY\textgreater, 
\textless YY\textgreater, 
\textless MM\textgreater, 
\textless DD\textgreater, 
\textless HH\textgreater}
These are replaced with the components of time of the start epoch.

















\section{input\_files:}

This section of the yaml file specifies the lists of files to be used for general metadata inputs, and inputs of external product data.

\begin{lstlisting}[language=yaml,caption=A typical input\_files section]
# Example
input_files:

	root_input_directory: /data/acs/pea/proc/exs/products/
	
	atxfiles:   [ igs14_2045_plus.atx                                ]
	snxfiles:   [ igs19P2062.snx                                     ]
	blqfiles:   [ OLOAD_GO.BLQ                                       ]  
	navfiles:   [ brdm1990.19p                                       ]  
	orbfiles:   [ orb_partials/gag20624_orbits_partials_new.out      ]  
	sp3files:   [ "*.sp3"                                            ]
	clkfiles:   [ jpl20624.clk                                       ]  
	erpfiles:   [ igs19P2062.erp                                     ]  
	dcbfiles:   [ CAS0MGXRAP_20191990000_01D_01D_DCB.BSX             ] 
	bsxfiles:   []
	ionfiles:   [] 
\end{lstlisting}

\subsection*{root\_input\_directory:}
This specifies a root directory to be prepended to all other file paths specified in this section. For file paths that are absolute, (ie. starting with a /), this parameter is not applied.

\subsection*{atxfiles:}
A list of ANTEX files to be used in processing. These may supply the antenna parameters to be used by satellites and receivers.

\subsection*{snxfiles:}
A list of SINEX files to be used in processing. These may supply the initial positions and other metadata for receivers.

\subsection*{blqfiles:}
A list of BLQ files to be used in processing. These may supply the ocean tide loading data.

\subsection*{navfiles:}
A list of NAV files to be used in processing. These may supply the basic broadcast ephemerides for satellites.

\subsection*{orbfiles:}
A list of ORB files to be used in processing. These may supply the PEA with orbital and ratiation pressure data from the Ginan's POD module, allowing precise orbit data to be passed between the two pieces of software.

\subsection*{sp3files:}
A list of SP3 files to be used in processing. These may supply the ephemerides for higher precision processing.

\subsection*{clkfiles:}
A list of CLK files to be used in processing. These may supply the clock offsets for satellites and receivers for higher precision processing.

\subsection*{erpfiles:}
A list of ERP files to be used in processing. These may supply the earth rotation parameter information.

\subsection*{dcbfiles:}
A list of DCB files to be used in processing. These may supply the differential code biases to assist with ambiguity resolution.

\subsection*{bsxfiles:}
A list of BSINEX files to be used in processing. These may supply biases to assist with ambiguity resolution.

\subsection*{ionfiles:}
A list of ION files to be used in processing. These may supply the ionospheric modelling parameters for single frequency processing.






\section{station\_data:}
This section specifies the sources of observation data to be used in positioning.


There are numerous ways that the \emph{pea} can access GNSS observations to process. 
You can specify individual files to process, set it up so that it will search a particular directory, or you can use a command line flag \lstinline{--rnx <rnxfilename>} to add an additional file to process. 
The data should be uncompressed rinex (gunzipped, and not in hatanaka format), or RTCM3 formatted binary data.


It may consist of RINEX files, or RTCM streams, which are specified as follows:
\subsubsection{Post processing:}

\begin{lstlisting}[language=yaml,caption=station\_data:]
# post processing example
station_data:
	root_stations_directory: /data/acs/ginan/examples/data
	rnxfiles:
		- "ALIC*.rnx"
		- "BAKO*.rnx"
		
	#obs_rtcmfiles:
	#	- "*-OBS.rtcm3"
		
	#nav_rtcmfiles:
	#	- "*-NAV.rtcm3"
\end{lstlisting}


\subsection*{root\_stations\_directory:}
This specifies a root directory to be prepended to all other file paths specified in this section. For file paths that are absolute, (ie. starting with a /), this parameter is not applied.

\subsection*{rnxfiles:}
This is a list of RINEX files to be used for observation data. The first 4 characters of the filename are used as the receiver ID.

If multiple files are supplied with the same ID, they are all processed in sequence - according to the epoch times specified within the files. In this case, it is advisible to correctly specify the start\_epoch for the filter, or the first epoch in the first file will likely be used.

\subsection*{obs\_rtcmfiles:}
This is a list of RTCM binary files to be used for observation data. The first 4 characters of the filename are used as the receiver ID.

This can be used to read data that has been saved from a stream for later testing.

\subsection*{nav\_rtcmfiles:}
This is a list of RTCM binary files to be used for navigation and correction data. No receiver is to be associated with these files.

\subsubsection{Real-time processing:}

To process data in real-time you will need to set up the location, username annd password for the caster that you will be obtaining the input data streams from in the configuration file.

The pea supports obtaining streams from casters that use NTRIP 2.0 over http and https.


\begin{lstlisting}[language=yaml,caption=station\_data:]
# realtime streaming example
station_data:
	
	stream_root: "http://<username>:<password>@auscors.ga.gov.au:2101/"

	nav_streams:
		- BCEP00BKG0
		- SSRA00CNE0
		
	obs_streams:
		- STR100AUS0
\end{lstlisting} \label{lst:realtimeConfig}

As shown in listing:\ref{lst:realtimeConfig}, the caster url, username and password are specified within double quotes with the \emph{stream\_root} tag. In this example the streams are being obtained from the auscors caster run by Geoscience Australia. 
The broadcast information is being obtained from the stream \emph{BCEP00BKG0} being supplied by BKG, and corrections to the utlra-rapid predicted orbit are being obtained from the stream \emph{SSRA00CNE0}. 
The real-time data being processed is for the continuous GNSS station located at Mount Stromlo obtained from the stream \emph{STR100AUS0}.

You can test your username and password is working correctly by running the curl command:
\begin{lstlisting}[language=bash]
curl https://ntrip.data.gnss.ga.gov.au/ALIC00AUS0 -H "Ntrip-Version: NTRIP/2.0" -i  --output - -u <user>
\end{lstlisting}

\subsection*{stream\_root:}
This specifies a root url to be prepended to all other streams specified in this section. If the streams used have individually specified root urls, usernames, or passwords, this should not be used.

\subsection*{obs\_streams:}
This is a list of RTCM streams to read realtime data from. The first 4 characters of the filename are used as the receiver ID.

In combination with the stream\_root parameter, they may require a username, password, port and mountpoint.

The streams in this section are processed for observations from receivers.

\subsection*{nav\_streams:}
This is a list of RTCM streams to read realtime data from. 

In combination with the stream\_root parameter, they may require a username, password, port and mountpoint.

The streams in this section are processed separately from observations, and will typically be used for receiving SSR messages or other navigational data from an external service.










\section{output\_files:}
This section of the yaml file specifies options to enable outputs and specify file locations.

An example of this section follows:
\begin{lstlisting}[language=yaml,caption=output\_files:]
output_files:

root_output_directory:          /data/acs/ginan/examples/<CONFIG>/

output_trace:                   true
trace_level:                    3
trace_directory:                ./
trace_filename:                 <CONFIG>-<STATION><LOGTIME>.TRACE

output_residuals:               false

output_config:                  true

output_summary:                 false
summary_directory:              ./
summary_filename:               <CONFIG>-<YYYY><DDD><HH>.SUM

output_clocks:                  true
clocks_directory:               ./
clocks_filename:                <CONFIG>.clk
\end{lstlisting}

\subsection*{root\_output\_dir:}
This specifies a root directory to be prepended to all other file paths specified in this section. For file paths that are absolute, (ie. starting with a /), this parameter is not applied.

\subsection*{[X]\_directory:}
Directory to output file [X] to, where [X] are the features below. May contain wildcard tags. May be relative to root\_output\_dir, or absolute. If the directory does not exist, it will be created.

\begin{itemize}
\item trace\_directory
\item summary\_directory
\item clocks\_directory
\item ionex\_directory
\item biasSinex\_directory
\item sinex\_directory
\item persistance\_directory
\item rtcm\_directory
\end{itemize}

\subsection*{[X]\_filename:}
Filename to use for output of [X]. May contain wildcard tags. File will be created or overwritten if it already exists.

\begin{itemize}
\item trace\_filename
\item summary\_filename
\item clocks\_filename
\item ionex\_filename
\item biasSinex\_filename
\item sinex\_filename
\item persistance\_filename
\item obs\_rtcm\_filename
\item nav\_rtcm\_filename
\end{itemize}

\subsection*{trace\_level:}
Integer from 0-5 to specify verbosity of trace outputs. (5 - print everything)

\subsection*{trace\_rotate\_period, trace\_rotate\_period\_units:}
Granularity of length of time used for \textless LOGTIME\textgreater tags. These parameters may be used such that the filename of an output will change intermittently, and thus distribute the output over multiple files.

The \textless LOGTIME\textgreater tag is updated according to the epoch time, not the current clock time.

trace\_rotate\_period must be a numeric value, and trace\_rotate\_period\_units may be one of seconds (default), minutes, hours, days, weeks, years, (with or without plural s).

\subsection*{output\_residuals:}
Boolean to print the residuals from kalman filter operation to relevant trace files.

\subsection*{output\_config:}
Boolean to print a copy of the yaml file to the top of each trace file. This may assist with keeping a record of the parameters used to generate the particular results contained in the file.

\subsection*{output\_trace:}
Boolean to generate per-station trace files.

\subsection*{output\_summary:}
Boolean to generate a network summary file.

\subsection*{output\_clocks:}
Boolean to generate RINEX formatted clock files from processed data.

\subsection*{output\_AR\_clocks:}
Boolean to specify that the ambiguity resolved version of clocks should be output if output\_clocks is enabled.

\subsection*{output\_ionex:}
Boolean to generate an IONEX file from processed ionosphere data.

\subsection*{output\_ionstec:}
Boolean to generate an IONSTEC file from processed ionosphere data.

\subsection*{output\_biasSINEX:}
Boolean to generate a biasSINEX from processed network data.

\subsection*{output\_sinex:}
Boolean to generate a sinex file containing processed solutions, and the metadata used to generate them.

\subsection*{output\_persistance:}
Boolean to save the network filter state, and navigation and ephemerides structure to disk once per epoch. For realtime processing where ephemerides are sourced from a a stream over several minutes, this may enable quicker start-up if the processor is restarted.

\subsection*{input\_persisance:}
Boolean to try to load a saved filter and navigation structure from disk.

\subsection*{output\_mongo\_measurements:}
Boolean to output kalman filter measurements and residuals to a mongo database.

\subsection*{output\_mongo\_states:}
Boolean to output the results of kalman filter processing to a mongo database.

\subsection*{output\_mongo\_logs:}
Boolean to output timestamped log data from the console to a mongo database.

\subsection*{output\_mongo\_metadata:}
Boolean to output timestamped metadata from processing to a mongo database. (unimplemented)

\subsection*{delete\_mongo\_history:}
Boolean to delete a previous database using the same <CONFIG> tag before processing, to prevent collisions.

\subsection*{mongo\_uri:}
The URL to the location of the mongo database server.



















\section{processing\_options:}

This sections specifies the extent of processing that is performed by the engine.

\subsection*{epoch\_interval:}
Increment in nominal epoch time for each processing epoch. This parameter may be used to sub-sample datasets by using an epoch\_interval that is a multiple of the dataset's internal interval between epochs.

\subsection*{start\_epoch}
Nominal time of the first epoch to process. Time is formatted as YYYY-MM-DD HH:MM:SS.
This parameter may be left undefined to use the first available data point.

\subsection*{end\_epoch}
Maximum nominal time of the last epoch to process. This parameter may be left undefined.

\subsection*{max\_epochs:}
Maximum epochs to process before completion. This parameter may be left undefined.

\subsection*{process\_modes:}

\begin{lstlisting}[language=yaml,caption=process\_modes:]
process_modes:
    user:                   true
    network:                false
    minimum_constraints:    false
    rts:                    false
    ionosphere:             false
\end{lstlisting}

\subsection*{user:}
Boolean to process all stations individually. Typically used for determining position of individual receivers.

\subsection*{network:}
Boolean to process all stations in a single filter. May be used for determination of orbits, clocks, etc.

\subsection*{minimum\_constraints:}
Boolean to apply a rigid transformation to the results of the network filter after completion.

\subsection*{ionosphere:}
Boolean to compute an ionosphere model from observations.

\subsection*{unit\_tests:}
Boolean to run tests to compare intermediate values during processing to stored results.


\subsection*{process\_sys:}
\begin{lstlisting}[language=yaml,caption=process\_sys:]
process_sys:
    gps:            true
    glo:            false
    gal:            false
    bds:            false
\end{lstlisting}

Booleans to enable the various GNSS satellite systems.

\begin{itemize}
\item gps
\item glo
\item gal
\item bds
\end{itemize}

\subsection*{elevation\_mask:}
Minimum elevation required for observations to be used, measured in degrees.

\subsection*{ppp\_ephemeris:}
Option to specify source of satellite ephemeris used in PPP processing. Sources are:
\begin{itemize}
\item broadcast
\item precise
\item precise\_com
\item sbas
\item ssr\_apc
\item ssr\_com
\end{itemize}

\subsection*{tide\_solid:}
Boolean to apply solid tide model to station positions.

\subsection*{tide\_otl:}
Boolean to apply ocean tide loading model to station positions.

\subsection*{tide\_pole:}
Boolean to apply pole tide model to station positions.

\subsection*{phase\_windup}
Boolean to apply phase windup model to satellite phase measurements.

\subsection*{reject\_eclipse}
Boolean to exclude eclipsed satellites from processing.

\subsection*{raim}
Boolean to perform 'Receiver autonomous integrity monitoring' to detect and exclude observations that result in SPP failures.

\subsection*{cycle\_slip: thres\_slip:}
Threshold to apply to geometry free phase values to determine if an observation should be rejected due to a slip.

\subsection*{max\_inno:}
Maximum innovation in PPP measurement before both phase and code measurements are excluded.

\subsection*{deweight\_factor:}
Factor by which measurement variances are increased upon detection of a bad measurement.

\subsection*{max\_gdop:}
Maximum 'geometric dilution of precision' allowed for an SPP result to be valid.

\subsection*{antexacs:}
Internal processing option. Bad things will likely happen if this is set to false.

\subsection*{sat\_pcv:}
Boolean to model satellite phase center variations.

\subsection*{pivot\_station:}
Station specified as origin for receiver clocks. Clocks for this station will be constrained to zero. May be set to \textless AUTO \textgreater or undefined to use first available station.

\subsection*{pivot\_satellite}:
Unused.

\subsection*{wait\_next\_epoch:}
Expected time interval between successive epochs data arriving. For real-time this should be set equal to epoch\_interval.

\subsection*{wait\_all\_stations:}
Window of delay to allow observation data to be received for processing.
Processing will begin at the earliest of:
\begin{itemize}
\item Observations received for all stations
\item wait\_all\_stations has elapsed since any station has received observations
\item wait\_all\_stations has elapsed since wait\_next\_epoch expired.
\end{itemize}

\subsection*{code\_priorities:}
List of observation codes that may be used in processing, and the order of priority for use. (Currently only a single code is used per frequency)

\subsection*{joseph\_stabilisation:}
Boolean to apply additional calculations in filter to ensure numerical stability.










\section{troposphere:}

\subsection*{model:}

\subsection*{vmf3dir:}

\subsection*{orography:}

\subsection*{gpt2grid:}




\section{ionosphere:}

\subsection*{corr\_mode:}

\subsection*{iflc\_freqs:}





\section{unit\_test\_options:}

\subsection*{output\_pass:}

Boolean to print pass messages in output file when tests pass. This may produce very long test files for not much benefit if we are just looking for failures.

\subsection*{stop\_on\_fail:}

Boolean to halt processing as soon as an error is located. This may allow testing and reporting to complete far sooner if there is a failure.

\subsection*{stop\_on\_done:}

Boolean to halt further processing if all required tests have been completed.

\subsection*{output\_errors:}

Boolean to print debug information about the error to the output file.

\subsection*{absorb\_errors:}

Boolean to replace incorrect values found in processing with the correct test values and continue processing as if the test had passed.
This may be useful for preventing a single bad test from causing cascading test failures as the values diverge from the original result.

\subsection*{directory, filename:}

File and directory to store and open test files.





\section{ionosphere\_filter\_parameters:}

\subsection*{model:}

Model to use in ionosphere routines. May be one of:
\begin{itemize}
\item meas\_out
\item bspline
\item spherical\_caps
\item spherical\_harmonics
\end{itemize}

\subsection*{model\_noise:}

Process noise to be applied to ionosphere kalman filter. (deprecated)

\subsection*{lat\_center, lon\_center:}

Longitude and latitude of center of ionosphere map in degrees.

\subsection*{lat\_width, lon\_width:}

Width of ionosphere maps in degrees.

\subsection*{lat\_res, lon\_res:}

Resolution of ionosphere maps in degrees.

\subsection*{time\_res:}

Resolution of ionosphere maps in time.

\subsection*{func\_order:}

Order of Legendre function used in spherical caps ionosphere.

\subsection*{layer\_heights:}

List of heights of modelled ionosphere layers.






\section{ambiguity\_resolution\_options:}

\subsection*{Min\_elev\_for\_AR:}

Minimum elevation to perform ambiguity resolution (degrees)

\subsection*{Set\_size\_for\_lambda:}

\subsection*{Max\_round\_iterat:}

\subsection*{GPS\_amb\_resol:}

\subsection*{WL\_mode:}

\subsection*{WL\_succ\_rate\_thres:}

\subsection*{WL\_sol\_ratio\_thres:}

\subsection*{WL\_procs\_noise\_sat:}

\subsection*{WL\_procs\_noise\_sta:}

\subsection*{NL\_mode:}

\subsection*{NL\_succ\_rate\_thres:}

\subsection*{NL\_proc\_start:}

\subsection*{read\_OSB, read\_DSB, read\_SSR, read\_satellite\_bias, read\_station\_bias, read\_GLONASS\_IFB:}

Booleans to enable reading of bias types from file.

\subsection*{write\_OSB, write\_DSB, write\_SSR\_bias, write\_satellite\_bias, write\_station\_bias:}

Booleans to enable writing of bias types to file.

\subsection*{bias\_output\_rate:}










\section{output\_options:}

This section specifies values to be used in the generation of output files.

\begin{lstlisting}[language=yaml,caption=output\_options:]
output_options:

    config_description:             ex11
    analysis_agency:                GAA
    analysis_center:                Geoscience Australia
    analysis_program:               AUSACS
    rinex_comment:                  AUSNETWORK1
\end{lstlisting}

\subsection*{config\_description:}

The value entered here is used to complete the \verb|<CONFIG>| wildcard.
This may enable a single change in the yaml file to make changes to many options, including output folders and filenames.

\subsection*{analysis\_agency, analysis\_center, analysis\_program, rinex\_comment:}

String to be written within files during various output files' generation.








\section{user\_filter\_parameters, network\_filter\_parameters, ionosphere\_filter\_parameters:}


\begin{lstlisting}[language=yaml,caption=Kalman Filter Configuration]
user_filter_parameters:

    max_filter_iterations:      5
    max_prefit_removals:        3

    rts_lag:                    -1      #-ve for full reverse, +ve for limited epochs
    rts_directory:              ./
    rts_filename:               PPP-<CONFIG>-<STATION>.rts

    inverter:                   LLT         #LLT LDLT INV

\end{lstlisting}

For details on the configuration of kalman filters refer to \ref{KFConfig}, and \ref{ch:RTS}










\section{default\_filter\_parameters:}



\begin{lstlisting}[language=yaml,caption=Default\_filter\_parameters]
default_filter_parameters:

    stations:

        error_model:        elevation_dependent         #uniform elevation_dependent
        code_sigmas:        [0.15]
        phase_sigmas:       [0.0015]

        pos:
            estimated:          true
            sigma:              [0.1]
            proc_noise:         [0.00057] #0.57 mm/sqrt(s), Gipsy default value from slow-moving
            proc_noise_dt:      second
\end{lstlisting}


\subsection*{error\_model:}

The GNSS observations can be weighted in three different ways in the PEA:
\begin{itemize}
    \item unigorm - all observations are assigned the same variance
    \item elevation\_dependent - an elevation dependent function is used to scale the observations, those at higher elevation are given more weight (a smaller standard deviation) than those observed at lower elevation
    \item SNR observations (coming soon) - the Carrier to Noise observations supplied by the receiver are used to determine the observation weight. Generally speaking this is very similar to elevation weighting, but is useful when use observations obtained from a non-geodetic grade receiver/antenna.
\end{itemize}

\subsection*{code\_sigmas, phase\_sigmas:}

Lists of the default sigma values for GNSS measurements, measured in meters.
Separate values may be entered for L1, L2 frequencies if desired, or the last value will be used for any undefined values in the list.

\subsection*{pos, clk, amb, trop...:}

For details on the configuration of estimated elements refer to \ref{KFConfig}

\section{minimum\_constraints:}

For details on the configuration of minimum constraints refer to \ref{MinConConfig}

\chapter{POD YAML Configuration}
\label{ch:pod_yaml_configuration}
\section{POD processing options (pod\_options)}
%

These options will control how the pod will process the input files, with four different options available. Only one of the options listed below can be set to true, the remainder must be set to false.\\
\begin{table}[h!]
\begin{tabular}{|p{4.5cm}|p{2cm}|p{3.5cm}|}
	\hline
    Option & Values & Comments \\
    \hline
    %\multicolumn{3}{|p{13cm}|}{In this section only one of the following options listed below can be set to true, the remainder must be set to false.}\\
    %\hline
	\textbf{pod\_mode\_fit} & true or false & Orbit Determination (pseudo-observations; orbit fitting \ ) \\
	\textbf{pod\_mode\_predict} & true or false & Orbit Determination and Prediction \\
	\textbf{pod\_mode\_eqm\_int} & true or false & Orbit Integration (Equation of Motion only) \\
	\textbf{pod\_mode\_ic\_int} & true or false & Orbit Integration and Partials (Equation of Motion and Variational Equations) initial condition integration \\
	\hline
\end{tabular}
\caption{caption for this table}
\label{table:label_name}
\end{table}
%
\begin{lstlisting}[language=xml,caption=pod\_options yaml configuration example]
pod_options:
# Example YAML showing different processing options
#--------------------------------------------------
pod_mode_fit:     true   
pod_mode_predict: false  
pod_mode_eqm_int: false  
pod_mode_ic_int:  false  
\end{lstlisting}
%
\begin{enumerate}
	\item \textbf{pod\_mode\_fit} - this is used to fit an existing sp3 file (this is sometimes referred to as pseudo observations) with the parameters that are set later on. See pod example 1
	\item \textbf{pod\_mode\_predict} - determine an orbit from observations and then predict the orbits path
	\item \textbf{pod\_mode\_eqm\_int} - determine the equations of motion only
	\item \textbf{pod\_mode\_ic\_int} -set up the initial conditions
\end{enumerate}

\section{Time scale(time\_scale)}
%
\begin{table}[h!]
\begin{tabular}{|p{4.5cm}|p{2cm}|p{3.5cm}|}
\hline
Option & Values & Comments \\
\hline
\textbf{TT\_time} & true or false & Terrestrial (TT)\\
\textbf{UTC\_time} & true or false & Universal (UTC)\\
\textbf{GPS\_time} & true or false & Satellite (GPS)\\
\textbf{TAI\_time} & true or false & Atomic (TAI)\\
\hline
\end{tabular}
\caption{caption for this table}
\label{table:label_name}
\end{table}
%
\begin{lstlisting}[language=xml,caption=time\_scale yaml configuration example]
	time_scale:
		TT_time:  false
		UTC_time: false
		GPS_time: true
		TAI_time: false
\end{lstlisting}
%
\begin{enumerate}
	\item \textbf{TT\_time} - 
	\item \textbf{UTC\_time} - 
	\item \textbf{GPS\_time} - 
	\item \textbf{TAI\_time} -
\end{enumerate}
%
\section{Initial Conditions (IC)}
\subsection{IC input format (ic\_input\_filename)}
 one is true, the other is false. If icf selected the ic\_filename value specifies the path to the file.
\begin{table}[h!]
\begin{tabular}{|p{4.5cm}|p{2cm}|p{3.5cm}|}
	\hline
	Option & Values & Comments \\
	\hline
   sp3 & true or false & sp3 format file\\
   icf & true or false & initial conditions file\\
   \hline
\end{tabular}
\caption{Initial Conditions input format options}
\label{table:yaml}
\end{table}
%
\begin{lstlisting}[language=xml,caption=ic\_input\_format yaml configuration example]
 	ic_input_format:
		sp3:  true   # Input a-priori orbit in sp3 format
		icf:  false  # Input a-priori orbit in POD Initial Conditions File (ICF) format
		ic_filename: some_file
\end{lstlisting}
%
\subsection{IC input reference system (ic\_input\_refsys)}
reference system for the initial conditions one is true, the other is false. 
\begin{table}[h!]
	\begin{tabular}{|p{4.5cm}|p{2cm}|p{3.5cm}|}
		\hline
		Option & Values & Comments \\
		\hline
		\textbf{itrf} & true or false & terestrial\\
		\textbf{icrf} & true or false & celestial \\
		\textbf{kepler} & true or false & polar form of celestial\\
		\hline
	\end{tabular}
	\caption{Initial Conditions reference system}
	\label{table:yaml}
\end{table}
%
{\small
\begin{lstlisting}[language=xml,caption=ic\_input\_refsys yaml configuration example]
   ic_input_refsys:
		itrf: true   # Initial Conditions Reference Frame: ITRF, ICRF
		icrf: false  # Initial Conditions Reference Frame: ITRF, ICRF
		kepler: false
\end{lstlisting}
}
%
\section{Using Pseudo observartions}
These options are used to control how pseudo observations are used by the POD.
 
\begin{table}[h!]
	\begin{tabular}{|p{4.5cm}|p{2cm}|p{3.5cm}|}
		\hline
		Option & Values & Comments \\
		\hline
		\textbf{pseudobs\_orbit\_filename} & filename & \emph{path to the observations file}\\
		\textbf{pseudobs\_interp\_step} & int &  Interval (sec) of the interpolated orbit\\
		\textbf{pseudobs\_interp\_points} & int & Number of data points used in Lagrange interpolation (at least 2)\\
		\hline
	\end{tabular}
	\caption{Initial Conditions reference system}
	\label{table:yaml}
\end{table}
%
{\small
\begin{lstlisting}[language=xml,caption=pseudo observation model yaml configuration example]
	pseudobs_orbit_filename: igs19424.sp3 # Pseudo observations orbit filename
	pseudobs_interp_step:    900          # Interval (sec) of the interpolated orbit
	pseudobs_interp_points:  12           # Number of data points used in Lagrange interpolation
\end{lstlisting}
}
%
\section{Orbit arc length}
\begin{table}[h!]
	\begin{tabular}{|p{4.5cm}|p{2cm}|p{3.5cm}|}
		\hline
		Option & Values & Comments \\
		\hline
		\textbf{orbit\_arc\_determination} & int &\emph{number of hours to integrate}\\
		\textbf{orbit\_arc\_prediction} & int &  \emph{number of hours to predict at end of orbit arc}\\
		\textbf{orbit\_arc\_backwards} & int & \emph{number of hours to check before start of orbit arc}\\
		\hline
	\end{tabular}
	\caption{Orbit arc options}
	\label{table:yaml}
\end{table}
%
{\small
\begin{lstlisting}[language=xml,caption=orbit arc length yaml configuration example]
# Orbit arc length (in hours)
	orbit_arc_determination: 24  # Orbit Estimation arc
	orbit_arc_prediction:    12  # Orbit Prediction arc
	orbit_arc_backwards:     2   # Orbit Propagation backwards arc
\end{lstlisting}
}
%
\section{External Orbit Comparison}
In this section only one of the following options listed below can be set to true, the remainder must be set to false.\\
\begin{table}[h!]
\begin{tabular}{|p{4.5cm}|p{2cm}|p{3.5cm}|}
	\hline
	Option & Values & Comments \\
	\hline
	\textbf{ext\_orbit\_enabled}  & true or false & \\
	\textbf{ext\_orbit\_type\_sp3}  & true or false & \\
	\textbf{ext\_orbit\_type\_interp} & true or false & \\
	\textbf{ext\_orbit\_type\_kepler} & true or false & \\
	\textbf{ext\_orbit\_type\_lagrange} & true or false & \\
	\textbf{ext\_orbit\_type\_position\_sp3} & true or false & \\
	\textbf{ext\_orbit\_filename} & filename &  \emph{path to the orbit file}\\
	\textbf{ext\_orbit\_interp\_step} & int & Interval (sec) of the interpolated\/Kepler orbit\\
	\textbf{ext\_orbit\_interp\_points} & int & Number of data points used in Lagrange interpolation (at least 2)\\
    \hline
\end{tabular}
	\caption{External orbit options}
\label{table:yaml}
\end{table}
%
{\small
\begin{lstlisting}[language=xml,caption=orbit arc length yaml configuration example]

# External Orbit Comparison
	ext_orbit_enabled: true
	ext_orbit_type_sp3:       false        # Orbit data in sp3 format 
	                                       # (including position and velocity vectors)
	ext_orbit_type_interp:    true         # Interpolated orbit based on Lagrange 
	                                       # interpolation of sp3 file
	ext_orbit_type_kepler:    false        # Keplerian orbit
	ext_orbit_type_lagrange:  false        # 3-day Lagrange interpolation
	ext_orbit_type_position_sp3:    false  # Position and SP3 file
	ext_orbit_filename:       igs19424.sp3 # External (comparison) orbit filename
	ext_orbit_interp_step:    900          # Interval (sec) of the interpolated/Kepler orbit
	ext_orbit_interp_points:  12           # Number of data points used 
	                                       # in Lagrange interpolation
\end{lstlisting}
}
%
\subsection{External orbit reference frame (ext\_orbit\_frame)}
\begin{table}[h!]
	\begin{tabular}{|p{4.5cm}|p{2cm}|p{3.5cm}|}
		\hline
		Option & Values & Comments \\
		\hline
		\textbf{itrf}  & true or false & terrestrial\\
		\textbf{icrf}  & true or false & celestial\\
		\textbf{kepler} & true or false & kepler orbital elements\\
        \hline
	\end{tabular}
	\caption{External orbit options}
	\label{table:yaml}
\end{table}
%
{\small
\begin{lstlisting}[language=xml,caption=external orbit reference frame yaml configuration example]
   ext_orbit_frame:
		itrf: true            # External orbit reference frame - ITRF
		icrf: false           # External orbit reference frame - ICRF
		kepler: false	
\end{lstlisting}
}
%        
\section{Earth Orientation Parameters}
In this section only one of the following options listed below can be set to true, the remainder must be set to false.\\
\subsection{EOP type}
\begin{table}[h!]
	\begin{tabular}{|p{4.5cm}|p{2cm}|p{3.5cm}|}
		\hline
		Option & Values & Comments \\
		\hline
		\textbf{EOP\_soln\_c04} & true or false & C04 is the IERS solution\\
		\textbf{EOP\_soln\_rapid} & true or false & Rapid is the rapid\/prediction center solution\\
		\textbf{EOP\_soln\_igs} & true or false & igs is the ultra-rapid solution using partials. To use this you need both the rapid file and partials file.\\
		\textbf{EOP\_soln\_c04\_file} & filename & \\
		\textbf{EOP\_soln\_rapid\_file} & filename & \\
		\textbf{ERP\_soln\_igs\_file} & filename & \\
		\textbf{EOP\_soln\_interp\_points} & int & \\
		\hline
	\end{tabular}
	\caption{Earth Orientation Parameter solution options}
	\label{table:yaml_eop_sol}
\end{table}
%
{\small
\begin{lstlisting}[language=xml,caption=eop estimation options]
   	EOP_soln_c04:   true         # IERS C04 solution : EOP_sol = 1
	EOP_soln_rapid: false        # IERS rapid service/prediction center (RS/PC) Daily : EOP_sol = 2
	EOP_soln_igs:   false        # IGS ultra-rapid ERP + IERS RS/PC Daily (dX,dY) : EOP_sol = 3. Need both rapid_file AND igs_file
	EOP_soln_c04_file: eopc04_14_IAU2000.62-now
	EOP_soln_rapid_file: finals2000A.daily
	ERP_soln_igs_file: igu18543_12.erp
	EOP_soln_interp_points: 4    # EOP solution interpolation points
\end{lstlisting}
}
%
\subsection{IAU Precession-Nutation model}
\begin{table}[h!]
	\begin{tabular}{|p{4.5cm}|p{2cm}|p{3.5cm}|}
		\hline
		Option & Values & Comments \\
		\hline
        \textbf{eop\_soln\_interp\_points} & int & number of data points to be used in an eop interpolation (at least 2!)\\
        \textbf{iau\_model\_2000} & true or false & \\
        \textbf{iau\_model\_2006} & true or false & \\
		\hline
	\end{tabular}
	\caption{EOP model options}
	\label{table:yaml}
\end{table}
%
{\small
	\begin{lstlisting}[language=xml,caption=eop model]
	# IAU Precession-Nutation model:
	iau_model_2000: true         # IAU2000A: iau_pn_model = 2000
	iau_model_2006: false        # IAU2006/2000A: iau_pn_model = 2006
	\end{lstlisting}
}
%
\section{Input files}
\begin{table}[h!]
	\begin{tabular}{|p{4.5cm}|p{2cm}|p{3.5cm}|}
		\hline
		Option & Values & Comments \\
		\hline
        \textbf{gravity\_model\_file} & filename & \\
        \textbf{DE\_fname\_header} & filename &  Emphemeris header file \\
        \textbf{DE\_fanme\_data} & filename & Emphemeris data file \\
        \textbf{ocean\_tides\_model\_file} & filename & \\
        \textbf{leapsec\_filename} & filename & leapseconds to be added\\
        \textbf{satsinex\_filename} & filename & sinex file with satellite meta-data\\       
		\hline
	\end{tabular}
	\caption{Input files}
	\label{table:label_name}
\end{table}
%
{\small
	\begin{lstlisting}[language=xml,caption=yaml example for general input files]
	# Gravity model file
	gravity_model_file: goco05s.gfc  
	# goco05s.gfc, eigen-6s2.gfc, ITSG-Grace2014k.gfc
	
	# Planetary/Lunar ephemeris - JPL DE Ephemeris
	DE_fname_header: header.430_229
	DE_fname_data:   ascp1950.430
	
	# Ocean tide model file
	ocean_tides_model_file: fes2004_Cnm-Snm.dat 
	# FES2004 ocean tide model
	
	# Leap second filename
	leapsec_filename: leap.second
	
	# Satellite metadata SINEX
	satsinex_filename: igs_metadata_2063.snx	
	\end{lstlisting}
}
%

\section{Output options}
\begin{table}[h!]
	\begin{tabular}{|p{4.5cm}|p{2cm}|p{3.5cm}|}
		\hline
		Option & Values & Comments \\
		\hline
        \textbf{sp3\_velocity} & true or false & if you wish to write out the velocities for comparison \\
        \textbf{partials\_velocity}: true or false & if you wish to write velocity vector partials to the output file\\
		\hline
	\end{tabular}
	\caption{Input files}
	\label{table:label_name}
\end{table}

{\small
	\begin{lstlisting}[language=xml,caption=yaml example for output file optionss]
	# Write to sp3 orbit format: Option for write Satellite Velocity vector
	sp3_velocity: false        #  Write Velocity vector to sp3 orbit
	
	#----------------------------------------------------------------------
	# Write partials of the velocity vector w.r.t. parameters into the orbits_partials output file:
	partials_velocity: false   # Write out velocity vector partials wrt orbital state vector elements
	\end{lstlisting}
}
\section{Variational Equation Options}
\begin{table}[h!]
	\begin{tabular}{|p{4.5cm}|p{2cm}|p{3.5cm}|}
		\hline
		Option & Values & Comments \\
		\hline
		\textbf{veq\_integration}& true or false &   pod mode overides it anyway. Ignore.\\
		\hline
		\textbf{ITRF} & true or false & reference\_frame\\
		\textbf{ICRF} & true or false &  one must be true\\   
		\textbf{kepler} & true or false & \\    
		\hline
	\end{tabular}
	\caption{general options}
	\label{table:label_name}
\end{table}
%
{\small
	\begin{lstlisting}[language=xml,caption=yaml example for variational equation options]
# Variational Equations
veq_integration: false

#----------------------------------------------------------------------
# Reference System for Variational Equations'  - Partials & Parameter Estimation
veq_refsys:
itrs: true      # ITRS: Terrestrial Reference System
icrs: false     # ICRS: Celestial Reference System
kepler: false
	\end{lstlisting}
}

\section{General Options}
\begin{table}[h!]
	\begin{tabular}{|p{4.5cm}|p{2cm}|p{3.5cm}|}
		\hline
		Option & Values & Comments \\
		\hline
        \textbf{estimator\_iterations} & int & integrate this number of times, using the generated initial conditions from the previous run as a start point\\   
        \hline
	\end{tabular}
	\caption{general options}
	\label{table:label_name}
\end{table}
%
{\small
\begin{lstlisting}[language=xml,caption=yaml example for output file options]
	# Parameter Estimation
	estimator_iterations: 2
\end{lstlisting}
}
%
%
\section{Apriori solar radiation models}   
%
\begin{table}[h!]
	\begin{tabular}{|p{4.5cm}|p{2cm}|p{3.5cm}|}
		\hline
		Option & Values & Comments \\
		\hline
		\textbf{no\_model}  & true or false & \\
		\textbf{cannon\_ball\_model} & true or false & see \nameref{sec:cannonball_srp}\\
		\textbf{simple\_boxwing\_model} & true or false & \\
		\textbf{full\_boxwing\_model} & true or false & \\
		\hline
	\end{tabular}
	\caption{srp\_apriori\_model; Exactly one option must be true}
	\label{table:label_name}
\end{table}
%
{\small
\begin{lstlisting}[language=xml,caption=yaml example for apriori srp model options]
srp_apriori_model:
	no_model:   false
	cannon_ball_model:   true
	simple_boxwing_model:   false
	full_boxwing_model:     false
\end{lstlisting}
}
%
\subsection{Estimated Solar radiation models}
\begin{table}[h!]
	\begin{tabular}{|p{4.5cm}|p{2cm}|p{3.5cm}|}
		\hline
		Option & Values & Comments \\
		\hline
		\textbf{ECOM1}  & true or false & \\ 
		\textbf{ECOM2}  & true or false & \\
		\textbf{hybrid} & true or false &  mix of ECOM1 and ECOM2\\
		\textbf{SBOXW}  & true or  false & Simple box wing model\\
		\textbf{EMPirical models} & true or false & Empirical is independent of the other four\\
		\hline
	\end{tabular}
	\caption{srp\_modes}
	\label{table:label_name}
\end{table}
%
%
{\small
\begin{lstlisting}[language=xml,caption=yaml example for apriori srp model options]
srp_apriori_model:
	no_model:               false
	cannon_ball_model:      true
	simple_boxwing_model:   false
	full_boxwing_model:     false
\end{lstlisting}
}
%
\subsection{gravity\_model }
Type of gravity model to apply, only one option can be true.
\begin{table}[h!]
	\begin{tabular}{|p{4.5cm}|p{2cm}|p{3.5cm}|}
		\hline
		Option & Values & Comments \\
		\hline
		central\_force            & true or false &  \\ 
		static\_gravity\_model    & true or false &  \\
		time\_variable\_model     & true or false &  \\
		iers\_geopotential\_model & true or false &  \\
		\hline
	\end{tabular}
	\caption{caption for this table}
	\label{table:label_name}
\end{table}
{\small
	\begin{lstlisting}[language=xml,caption=yaml example for gravitational force model options]
gravity_model:
	central_force:           false   # Central force gravity field              : gravity_model = 0
	static_gravity_model:    false   # Static global gravity field model        : gravity_model = 1
	time_variable_model:     true    # Time-variable global gravity field model : gravity_model = 2
	iers_geopotential_model: false   # IERS conventional geopotential model     : gravity_model = 3
    \end{lstlisting}
}
\subsection{stochastic pulse (pulse)}

Do not mix pulses in R/T/N (terrestrial) with pulses in (X/X/Z)


\begin{table}[h!]
	\begin{tabular}{|p{4.5cm}|p{2cm}|p{3.5cm}|}
		\hline
		Option & Values & Comments \\
		\hline
		\textbf{enabled} & true or false & then if true: \\
		\hline
		\textbf{epoch\_number} & int & number of epochs to apply pulses each day \\
		\textbf{offset} & int & seconds until the first pulse of the day \\
		\textbf{interval} & int & seconds between each pulse (after the first)\\
		\hline
		directions & & \\
		\textbf{x\_direction} &  true or false &  \\
		\textbf{y\_direction} &  true or false &  \\
		\textbf{z\_direction} &  true or false &  \\
		\textbf{r\_direction} &  true or false &  \\
		\textbf{t\_direction} &  true or false &  \\
		\textbf{n\_direction} &  true or false &  \\
\hline
	\end{tabular}
	\caption{caption for this table}
	\label{table:label_name}
\end{table}
%
{\small
	\begin{lstlisting}[language=xml,caption=yaml example for gravitational force model options]
   pulse:
	enabled:      false
	epoch_number:     1    # number of epochs to apply pulses
	offset:       43200    # since the start of day
	interval:     43200    # repeat every N seconds
	directions:
		x_direction:   true
		y_direction:   true
		z_direction:   true
		r_direction:   false
		t_direction:   false
		n_direction:   false
	reference_frame:
		icrf:          true
		orbital:       false
	\end{lstlisting}
}
%
\section{EQM options}
%
\subsection{Integration Step}
\begin{table}[h!]
	\begin{tabular}{|p{4.5cm}|p{2cm}|p{3.5cm}|}
		\hline
		Option & Values & Comments \\
		\hline
		RK4\_integrator\_method & true or false &  Do not use RK4 for veq as it is not implemented\\ 
		RKN7\_integrator\_method & true or false & only one can be true\\ 
		RK8\_integrator\_method  & true or false & \\
		\hline
		\textbf{integrator\_step} & int & step size in seconds \\
		\hline
	\end{tabular}
	\caption{caption for this table}
	\label{table:label_name}
\end{table}
%
{\small
	\begin{lstlisting}[language=xml,caption=yaml example for gravitational force model options]
# Numerical integration method
# Runge-Kutta-Nystrom 7th order RKN7(6): RKN7, Runge-Kutta 4th order: RK4, Runge-Kutta 8th order RK8(7)13: RK8
integration_options:
	RK4_integrator_method: false
	RKN7_integrator_method: true
	RK8_integrator_method: false
	integrator_step: 900         # Integrator stepsize in seconds	
	\end{lstlisting}
	}

\subsection{Gravity Field}
\begin{table}[h!]
	\begin{tabular}{|p{4.5cm}|p{2cm}|p{3.5cm}|}
		\hline
		Option & Values & Comments \\
		\hline
		\textbf{enabled} & true or false & and if true:\\
		\textbf{gravity\_degree\_max} &  & maximum model terms in spherical harmonic expansion \\ 
		\textbf{timevar\_degree\_max} &  & maximum time variable model terms in spherical harmonic expansion \\
		\hline
	\end{tabular}
	\caption{caption for this table}
	\label{table:label_name}
\end{table}
%
{\small
	\begin{lstlisting}[language=xml,caption=yaml example for gravitational force model options]
# Gravitational Forces
gravity_field:
	enabled: true
	gravity_degree_max:      15      # Gravity model maximum degree/order (d/o)
	timevar_degree_max:      15      # Time-variable coefficients maximum d/o
	\end{lstlisting}
}
%
\subsection{planetary\_perturbations:}
\begin{table}[h!]
	\begin{tabular}{|p{4.5cm}|p{2cm}|p{3.5cm}|}
		\hline
		Option & Values & Comments \\
		\hline
		\textbf{enabled} & true or false & Uses the emphemeris \\ %TODO specify the file
		\hline
	\end{tabular}
	\caption{planetary\_perturbations}
	\label{table:label_name}
\end{table}  
%
{\small
	\begin{lstlisting}[language=xml,caption=yaml example for planetary pertubations]
# Planetary Gravitational Forces
	planetary_perturbations:
		enabled: true	
	\end{lstlisting}
}	
\subsection{tidal\_effects:}
\begin{table}[h!]
	\begin{tabular}{|p{4.5cm}|p{2cm}|p{3.5cm}|}
		\hline
		Option & Values & Comments \\
		\hline
		solid\_tides\_nonfreq & True or False & frequency independent Solid Earth Tides \\
		solid\_tides\_freq & True or False & frequency dependent Solid Earth Tides \\
		ocean\_tides & True or False & uses the ocean tides file \\
		solid\_earth\_pole\_tides & True or False & tide induced earth spin rotation not about the centre of the ellipsoid \\
		ocean\_pole\_tide & True or False & ocean response to the above \\
		ocean\_tides\_degree\_max & True or False & maximum model term in spherical harmonic expansion \\
		\hline
	\end{tabular}
	\caption{caption for this table}
	\label{table:label_name}
\end{table}
%
{\small
	\begin{lstlisting}[language=xml,caption=yaml example for tidal effects]
   	tidal_effects:
		enabled: true
		solid_tides_nonfreq:    true   # Solid Earth Tides frequency-independent terms
		solid_tides_freq:       true   # Solid Earth Tides frequency-dependent terms
		ocean_tides:            true   # Ocean Tides
		solid_earth_pole_tides: true   # Solid Earth Pole Tide
		ocean_pole_tide:        true   # Ocean Pole Tide
		ocean_tides_degree_max: 15     # Ocean Tides model maximum degree/order	
	\end{lstlisting}
}	
\section{relativistic\_effects:}
\begin{table}[h!]
	\begin{tabular}{|p{4.5cm}|p{2cm}|p{3.5cm}|}
		\hline
		Option & Values & Comments \\
		\hline
		\textbf{enabled} & true or false & Lens\_Thirring, SchwarzChild and deSitter effects, there are no means to separate these effects currently. The Lens Thirring effect is calculated but subsequently ignored in the POD.\\
		\hline
	\end{tabular}
	\caption{relativistic\_effects}
	\label{table:label_name}
\end{table}
%
{\small
	\begin{lstlisting}[language=xml,caption=yaml example for relativistic effects]
# Relativistic effects
	relativistic_effects:
		enabled: true
	\end{lstlisting}
}
\section{non\_gravitational\_effects:}
\subsection{Models to be applied:}
\begin{table}[h!]
	\begin{tabular}{|p{4.5cm}|p{2cm}|p{3.5cm}|}
		\hline
		Option & Values & Comments \\
		\hline
		solar\_radiation: & true or false & radiation push from the sun \\
		earth\_radiation: & true or false & radiation push from the earth \\
		antenna\_thrust:  & true or false & reverse thrust from antenna radiation \\
		\hline
	\end{tabular}
	\caption{caption for this table}
	\label{table:label_name}
\end{table}
%
{\small
	\begin{lstlisting}[language=xml,caption=yaml example for non gravitational effects]
	# Non-gravitational Effects
	non_gravitational_effects:
		enabled: true
		solar_radiation: true
		earth_radiation: true
		antenna_thrust:  true
	\end{lstlisting}
}
\subsection{Empirical parameters}
\begin{table}[h!]
	\begin{tabular}{|p{4.5cm}|p{2cm}|p{3.5cm}|}
		\hline
		Option & Values & Comments \\
		\hline
		ecom\_d\_bias & true or false & \\ 
		ecom\_y\_bias & true or false & \\ 
		ecom\_b\_bias & true or false & \\ 
		ecom\_d\_cpr & true or false & (only ECOM1\/hybrid)\\ 
		ecom\_y\_cpr & true or false & (only ECOM1\/hybrid)\\ 
		ecom\_b\_cpr & true or false & \\
		ecom\_d\_2\_cpr & true or false &  (only ECOM2\/hybrid)\\ 
		ecom\_d\_4\_cpr & true or false & (only ECOM2\/hybrid) \\
		emp\_r\_bias & true or false & \\
		emp\_t\_bias & true or false & \\ 
		emp\_n\_bias  & true or false & \\
		emp\_r\_cpr  & true or false & \\
		emp\_t\_cpr  & true or false & \\
		emp\_n\_cpr & true or false & \\
		cpr\_count  & int & empirical cpr count \\
		\hline
	\end{tabular}
	\caption{Configuration options for solar radiation pressure models}
	\label{table:label_name}
\end{table}
%
{\small
	\begin{lstlisting}[language=xml,caption=yaml example for srp parameters]
	# Non-gravitational Effects
   srp_parameters:
		ECOM_D_bias:  true
		ECOM_Y_bias:  true
		ECOM_B_bias:  true
		EMP_R_bias:   true
		EMP_T_bias:   true
		EMP_N_bias:   true
		ECOM_D_cpr:   true
		ECOM_Y_cpr:   true
		ECOM_B_cpr:   true
		ECOM_D_2_cpr: false
		ECOM_D_4_cpr: false
		EMP_R_cpr:    true
		EMP_T_cpr:    true
		EMP_N_cpr:    true
		cpr_count:    1
	\end{lstlisting}
}


NB EQM and VEQ srp parameters MUST be identical. May move into pod\_options in future.
overrides are not implemented yet. Ignore for now. We imagine overrides at the system, block (sat type) and individual satellite level


\section{overides}
In this section put any system, block or PRN overrides that are different to the ones chosen before
{\small
	\begin{lstlisting}[language=xml,caption=yaml example for override]
overrides:
	system:
	GPS:
	srp_apriori_model:
	no_model:   false
	cannon_ball_model:   true
	simple_boxwing_model:   false
	full_boxwing_model:     false
	GAL:
	srp_apriori_model:
	no_model:   false
	cannon_ball_model:   false
	simple_boxwing_model:   false
	full_boxwing_model:     true
	GLO:
	srp_apriori_model:
	no_model:   false
	cannon_ball_model:   false
	simple_boxwing_model:   true
	full_boxwing_model:     false
	BDS:
	srp_apriori_model:
	no_model:   false
	cannon_ball_model:   false
	simple_boxwing_model:   true
	full_boxwing_model:     false
block:
	GPS-IIF:
		srp_apriori_model:
		no_model:   false
		cannon_ball_model:   false
		simple_boxwing_model:   true
		full_boxwing_model:     false
	# GPS BLK IIF use ECOM2 parameters
	srp_parameters:
		ECOM_D_bias:   true
		ECOM_Y_bias:   true
		ECOM_B_bias:   true
		ECOM_D_2_cpr: false
		ECOM_D_4_cpr: false
		ECOM_B_cpr:   true
prn:
	G01:
		srp_apriori_model:
			no_model:   false
			cannon_ball_model:   false
			simple_boxwing_model:   false
			full_boxwing_model:     true
	

	\end{lstlisting}
}

%\section{Example 1 configuration file}
%\lstinputlisting[language=xml, caption=pod example1 configuration file]{../../test/pod/ex1.yaml}
%\section{Example 2 configuration file}
%\lstinputlisting[language=xml, caption=pod example2 configuration file]{../../test/pod/ex2.yaml}
%\section{Example 3 configuration file}
%\lstinputlisting[language=xml, caption=pod example3 configuration file]{../../test/pod/ex3.yaml}
%\section{Example 4 configuration file}
%\lstinputlisting[language=xml, caption=pod example4 configuration file]{../../test/pod/ex4.yaml}
%\section{Example 5 configuration file}
%\lstinputlisting[language=xml, caption=pod example5 configuration file]{../../test/pod/ex5.yaml}
% \bibliography{sample-handout}
\printindex
\end{document}
