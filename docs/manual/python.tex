
\chapter{Python}
\section{Python Installation for Plotting, Processing,
etc.}\label{python-installation-for-plotting-processing-etc.}

Lastly, to run many of the included scripts for fast parsing of
.trace/.snx files, plotting of results, automatic running of the PEA
based on input date/times and stations, etc. then a number of python
dependencies are needed.

The file \texttt{python/ginan\_py37.yaml} has a list of the necessary
python dependencies.\\
The best way to take advantage of this is to install the
\emph{\emph{Miniconda}} virtual environment manager.\\
This will allow you to pass the \texttt{.yaml} file into the
\texttt{conda} command and automatically set up a new python
environment.

\paragraph{To install miniconda,}enter the following commands:

\begin{lstlisting}[language=bash,caption=Installing conda]
$ wget https://repo.anaconda.com/miniconda/Miniconda3-latest-Linux-x86_64.sh
$ bash Miniconda3-latest-Linux-x86_64.sh
\end{lstlisting}

And follow the installation instructions on screen (choosing all
defaults is fine).

\textit{Create virtual environment}, after installation you can create the \emph{ginan37} python environment. First open a new terminal session and enter:

\begin{lstlisting}[language=bash,caption=Example showing how to create a conda environment]
$ conda env create -f <dir_to_pea>/python/ginan_py37.yaml
\end{lstlisting}

You have now created the virtual python environment \emph{ginan37}
with all necessary dependencies. Anytime you wish you run python
scripts, ensure you are in the virtual environment by activating:

\begin{lstlisting}[language=bash,caption=Activating a conda environment]
$ conda activate gin37
\end{lstlisting}

And then run your desired script.

\section{Examples to run the PEA from a python script}\label{examples-to-run-the-pea-from-a-python-script}

The following examples will show how to use the python scripts in the
\texttt{source} directory to automatically gather files and run the pea
to produce results (as .TRACE files and plots).

\section{Run the PEA}\label{run-the-pea}

The easiest way to run the is to choose a reference station, a start
time, end time and directories to store the gathered files and resultant
outputs.\\
Doing this for station \emph{HOB200AUS} and processing 2 days of data,
starting 20 Dec 2020 00:00 and ending 21 Dec 2020 23:59:30, we can run
the following code (assuming the \texttt{pwd} is your pea directory):

\texttt{python3\ python/source/run\_pea\_PPP.py\ HOB200AUS\ 2020-12-20\_00:00:00\ 2020-12-21\_23:59:30\ \textless{}directory\_to\_download\_files\_to\textgreater{}\ \textless{}results\_output\_directory\textgreater{}\ -md\ -del\_yaml}

This will produce a .TRACE file that contains XYZ position information
for that station and estaimes for the Zenith Total Delay (ZTD). The
flags are as follows:\\
- \texttt{-md} is used to allow dates to be input in \emph{YYYY-MM-DD}
format as above. If the flag is not selected, the format must be
\emph{YYYY-DOY} (where \emph{DOY} is day-of-year).\\
- \texttt{-del\_yaml} is used to delete the \texttt{.yaml} config file
after the pea run. Without this flag, the \texttt{.yaml} file will be
kept (in \texttt{pwd})

\section{Near-Real-Time (NRT) Daily
Run}\label{near-real-time-nrt-daily-run}

In this example, we will see how to run the pea to produce ZTD and
station coordinate results for reference station \emph{HOB200AUS}, using
the most rececnt data available on GA and CDDIS servers.\\
Assuming the pwd is your \texttt{pea} directory, then the following code
should run to produce a .TRACE file and plots of ZTD and XYZ station
coordinates for the most recent day:

\texttt{python3\ python/source/run\_pea\_PPP.py\ HOB200AUS\ \_\ \_\ \textless{}directory\_to\_download\_files\_to\textgreater{}\ \textless{}results\_output\_directory\textgreater{}\ -rapid\ -m\_r\ -plt\_m\_r}

This differs to the code above in that the flags \texttt{-rapid},
\texttt{-m\_r} and \texttt{plt\_m\_r} have been selected.\\
- \texttt{-rapid} will find the rapid versions of files - these are less
accurate than final versions but are available sooner at CDDIS\\
- \texttt{-m\_r} will get the code to search the CDDIS database and find
the most recent rapid files available for \texttt{clk} and \texttt{sp3}
files. The date and time is then selected based on this, allowing us to
\_ \_ for the start and end times\\
- \texttt{-plt\_m\_r} will plot the ZTD and XYZ results as \texttt{.png}
files and place in a \texttt{plots} directory


\section{Site metadata utility}\label{log2snx}

Proper geodetic processing requires knowing of the site's receiver and antenna information. PEA extracts this information from the input sinex file or a set of sinex files. In addition, user can download \href{ftp://igs.ensg.ign.fr/pub/igs/igscb/station/general/igs.snx}{igs.snx} file that contains historical records of $\sim$500 sites and is regenerated daily from the logfiles available at \url{ftp://ftp.igs.org/pub/station/log/}. We developed a similar tool, log2snx.py, that does parsing of available igs logfiles and produces a sinex file as needed by PEA.

\begin{lstlisting}[caption=log2snx.py help message]
$ python3 scripts/log2snx.py  -h
  usage: log2snx.py [-h] [-l LOGGLOB] [-r RNXGLOB [RNXGLOB ...]] [-o OUTFILE]
  [-fs FRAME_SNX] [-fd FRAME_DIS] [-fp FRAME_PSD]
  [-d DATETIME] [-n NUM_THREADS]

IGS log files parsing utility. Globs over log files using LOGGLOB expression
and outputs SINEX metadata file. If provided with frame and frame
discontinuity files (soln), will project the selected stations present in the
frame to the datetime specified. How to get the logfiles: rclone sync
igs:pub/sitelogs/ /data/station_logs/station_logs_IGS -vv How to get the frame
files: rclone sync itrf:pub/itrf/itrf2014 /data/ITRF/itrf2014/ -vv --include
"*{gnss,IGS-TRF}*" --transfers=10 rclone sync igs:pub/ /data/TRF/ -vv
--include "{IGS14,IGb14,IGb08,IGS08}/*" see rclone config options inside this
script file Alternatively, use s3 bucket link to download all the files needed
s3://peanpod/aux/

optional arguments:
-h, --help            show this help message and exit
-l LOGGLOB, --logglob LOGGLOB
        logs glob path (required)
-r RNXGLOB [RNXGLOB ...], --rnxglob RNXGLOB [RNXGLOB ...]
        rinex glob path (optional)
-o OUTFILE, --outfile OUTFILE
        output file path (optional)
-fs FRAME_SNX, --frame_snx FRAME_SNX
        frame sinex file path (optional)
-fd FRAME_DIS, --frame_dis FRAME_DIS
        frame discontinuities file path (required with
        --frame_snx)
-fp FRAME_PSD, --frame_psd FRAME_PSD
        frame psd file path (optional)
-d DATETIME, --datetime DATETIME
        date to which project frame coordinates, default is
        today (optional)
-n NUM_THREADS, --num_threads NUM_THREADS
        number of threads to run in parallel
\end{lstlisting}

\section{Sinex preview utility}\label{snx2map}

To quickly visualise and compare networks, a fast sinex preview utility has been developed - snx2map.py. It accepts any number of sinex files, extract estimated coordinates of the sites present and outputs a map in the form of html page.

\begin{lstlisting}[caption=snx2map.py help message]
$ python3 scripts/snx2map.py  -h
usage: snx2map.py [-h] [-i SNXPATH [SNXPATH ...]] [-o OUTDIR]

Parse sinex SITE/ID block and create html map.

optional arguments:
  -h, --help            show this help message and exit
  -i SNXPATH [SNXPATH ...], --snxpath SNXPATH [SNXPATH ...]
                        path to sinex file (.snx/.ssc). Can be compressed with
                        LZW (.Z)
  -o OUTDIR, --outdir OUTDIR
                        path to output dir
\end{lstlisting}



% \section{Latency Tool}\label{latency-tool}

% To run currently do something like\ldots{}

% \begin{verbatim}
% nohup python source/main/python/npi/latency.py 2>&1 & 
% \end{verbatim}

% A simple configuration file \texttt{config/config.yaml} looks
% like\ldots{}

% \begin{verbatim}
% streams:
% - type: NTRIP
%   protocol: http
%   host: auscors.ga.gov.au
%   port: 2101
%   username: a_username
%   password: a_password
%   count: 10
%   stations:
%   - {format: RTCM3, id: COCO7}
 
% \end{verbatim}

% and again with more streams looks something like\ldots{}

% \begin{verbatim}
% streams:
% - type: NTRIP
%   protocol: http
%   host: auscors.ga.gov.au
%   port: 2101
%   username: a_username
%   password: a_password
% #  count: 10
%   stations:
%   - {id: TID17}
%   - {id: 00NA7}
%   - {id: 01NA7}
%   - {id: ABMF7}

% \end{verbatim}


% \section{Messages}\label{messages}

% %\subsection{COCO7}\label{coco7}
% %
% %From 10,000 messages\ldots{} - 1006 - 1008 - 1013 - 1019 - 1020 - 1033 -
% %1044 - 1077 - 1087 - 1097 - 1117 - 1127
% %
% %From 1,000 messages\ldots{} - 1006 Station co-ordinates - 1008 Antenna
% %Description - 1013 Auxiliary Operation Information - System Parameters -
% %1019 Auxiliary Operation Information - Satellite Ephemeris Data - 1020
% %Auxiliary Operation Information - Satellite Ephemeris Data - 1033
% %Receiver and Antenna Description - 1044 Auxiliary Operation Information
% %- Satellite Ephemeris Data - 1077 Observations - GPS MSMs (MSM7) - 1087
% %Observations - GLONASS MSMs (MSM7) - 1097 Observations - Galileo MSMs
% %(MSM7) - 1117 Observations - QZSS MSMs (MSM7) - 1127 Observations - BDS
% %MSMs (MSM7)
% %
% %2018-07-20 12:05:51,887 INFO Connected - starting to read
% %messages\ldots{} 2018-07-20 12:05:51,894 INFO frame length={[}None{]}
% %message number={[}None{]} 2018-07-20 12:05:51,895 INFO frame
% %length={[}56{]} message number={[}1117{]} 2018-07-20 12:05:51,896 INFO
% %frame length={[}294{]} message number={[}1127{]} 2018-07-20 12:05:51,898
% %INFO frame length={[}422{]} message number={[}1077{]} 2018-07-20
% %12:05:51,899 INFO frame length={[}261{]} message number={[}1087{]}
% %2018-07-20 12:05:51,899 INFO frame length={[}22{]} message
% %number={[}1097{]} 2018-07-20 12:05:51,900 INFO frame length={[}56{]}
% %message number={[}1117{]} 2018-07-20 12:05:51,901 INFO frame
% %length={[}294{]} message number={[}1127{]} 2018-07-20 12:05:51,902 INFO
% %frame length={[}422{]} message number={[}1077{]} 2018-07-20 12:05:51,903
% %INFO frame length={[}261{]} message number={[}1087{]} 2018-07-20
% %12:05:51,904 INFO frame length={[}22{]} message number={[}1097{]}
% %2018-07-20 12:05:51,904 INFO frame length={[}56{]} message
% %number={[}1117{]} 2018-07-20 12:05:51,906 INFO frame length={[}294{]}
% %message number={[}1127{]} 2018-07-20 12:05:51,906 INFO frame
% %length={[}56{]} message number={[}1033{]} 2018-07-20 12:05:51,908 INFO
% %frame length={[}422{]} message number={[}1077{]} 2018-07-20 12:05:51,909
% %INFO frame length={[}261{]} message number={[}1087{]} \ldots{}
% %2018-07-20 12:08:27,591 INFO frame length={[}22{]} message
% %number={[}1097{]} 2018-07-20 12:08:27,591 INFO frame length={[}56{]}
% %message number={[}1117{]} 2018-07-20 12:08:27,593 INFO frame
% %length={[}294{]} message number={[}1127{]} 2018-07-20 12:08:28,718 INFO
% %frame length={[}21{]} message number=\protect\hyperlink{section-1}{1006}
% %2018-07-20 12:08:28,718 INFO frame length={[}29{]} message
% %number={[}1008{]} 2018-07-20 12:08:28,720 INFO frame length={[}422{]}
% %message number={[}1077{]} 2018-07-20 12:08:28,720 INFO frame
% %length={[}256{]} message number={[}1087{]} 2018-07-20 12:08:28,720 INFO
% %frame length={[}22{]} message number={[}1097{]} 2018-07-20 12:08:28,721
% %INFO frame length={[}56{]} message number={[}1117{]} 2018-07-20
% %12:08:28,721 INFO frame length={[}294{]} message number={[}1127{]}
% %2018-07-20 12:08:29,566 INFO frame length={[}422{]} message
% %number={[}1077{]} 2018-07-20 12:08:29,567 INFO count={[}1000{]}
% %skipped={[}1{]}
% %
% %9 secs -\textgreater{} 06:00 then 2:29 to 08:29 so 2:38 secs to get
% %1,000 messages = 158 secs to get 1,000 messages = 0.158 secs / message
% %or 6.3 messages / sec

% \begin{itemize}
% \item
%   1033 Received and Antenna Description
% \item
%   1077 GPS MSMs
% \item
%   1087 GLONASS MSMs
% \item
%   1097 Galileo MSMs
% \item
%   1117 QZSS MSMs
% \item
%   1127 BDS MSMs
% \end{itemize}

% \subsection{MSM (1,2,3,4,5,6,7)}\label{msm-1234567}

% MSM message contains - message header - satellite data - signal data

% The message header contains - message number DF002 uint12 12-bits -
% reference station id DF003 uint12 12-bits - GNSS epoch time ??? uint30
% 30-bits - multiple message bit DF393 bit(1) 1-bit

% GNSS Epoch Time - GPS is --TOW DF004 DF004 GPS Epoch Time (TOW)
% 0-604,799,999ms 1ms uint30 milliseconds from beginning of GPS week
% (midnight GMT on Saturday night/Sunday morning measured in GPS time) -
% GLONASS is -- GLONASS DAY OF WEEK DF416 int3 3-bits DF416 GLONASS Day of
% Week 0-7 1 uint3 0=sunday, 1=monday, \ldots{}, 6=saturday, 7=not known
% -- GLONASS EPOCH TIME DF034 uint27 27-bits DF034 GLONASS Epoch Time
% 0-86,400,999ms 1ms uint27\\
% - Galileo is -- TOW DF248 DF248 GALILEO Epoch Time 0-604,799,999ms 1ms
% uint30 - QZSS is -- TOW DF428 DF428 QZSS Epoch Time 0-604,799,999ms 1ms
% uint30\\
% - BeiDou is -- TOW DF+002 DF427 BeiDou Epoch Time 0-604,799,999ms 1ms
% uint30

% \section{RTCM v3}\label{rtcm-v3}

% The following is a Hex-ASCII example of a message type 1005 (Stationary
% Antenna Reference Point, No Height Information).

% D3 00 13 3E D7 D3 02 02 98 0E DE EF 34 B4 BD 62 AC 09 41 98 6F 33 36 0B
% 98

% The parameters for this message are: 
% % 
% Reference Station Id = 2003 
% % 
% GPS
% Service supported, but not GLONASS or Galileo 
% % 
% ARP ECEF-X =
% 1114104.5999 meters 
% % 
% ARP ECEF-Y = -4850729.7108 meters 
% % 
% ARP ECEF-Z =
% 3975521.4643 meters

% \section{ACS Metrics}\label{acs-metrics}

% \subsection{RTCM v3}\label{rtcm-v3-1}

% An \texttt{RTCM\ v3} stream of data consists of a set of
% \texttt{RTCM\ v3} packets.

% An \texttt{RTCM\ v3} packet consists of:

% \begin{itemize}
% \item
%   8-bit preamble
% \item
%   16-bits ** 6-bit reserved (should be \texttt{0}s) ** 10-bit length
% \item
%   variable length message ** 12-bit message type
% \item
%   24-bit message CRC
% \end{itemize}

% \begin{longtable}[]{@{}lll@{}}
% \toprule
% x & x & x\tabularnewline
% \midrule
% \endhead
% Preamble & 8-bits &\tabularnewline
% \tabularnewline
% \bottomrule
% \end{longtable}

% \begin{itemize}
% \item
%   8-bit pre-amble character
% \item
%   The start or an \texttt{RTCM\ v3} packet is marked by the
%   \texttt{0xD3} pre-amble character
% \end{itemize}

% Following this is

% \subsection{Reading RTCM3 Data\ldots{}}\label{reading-rtcm3-data}

% \begin{itemize}
% \item
%   RTCM3 packet starts with 0xd3 byte so first skip to start of packet
% \end{itemize}

% See
% {[}https://pdfs.semanticscholar.org/bf93/21e569cc0f009982af3158a6489accc8f3e5.pdf{]}

% RTCM v3 frame looks like: - preamble 8-bits (0xD3) - reserved 6-bits -
% message length 10-bits - message variable length - CRC 24-bits

% So to parse it we do 1. find the start of packet byte - i.e.~the
% preamble byte - i.e.~0xd3\\
% 2. read the next 2 bytes (aka 16-bits) and extract the message length
% from the right-hand 10-bits

% \section{RTCM v3}\label{rtcm-v3-2}

% d3 74 63

% \begin{verbatim}
%                09   8765 4321
% \end{verbatim}

% 1101 0011 \textbar{} 0111 0100 \textbar{} 0110 0011 \textbar{}

% so the length is 63 (hex which is 99 decimal)

% d3 74 63 18 d1 94 65 0d 43 4c 53 14 b1 2c 4f

% 1101 0011 \textbar{} 0111 0100 \textbar{} 0110 0011 \textbar{} 1101 0011
% \textbar{} 0111 0100 \textbar{} 0110 0011

% 00011000 00011000

% byte 0 = preamble = 0xD3

% bytes 1 \& 2 = reserved (6-bits) + message length (10-bits)

% \begin{verbatim}
% byte1 & 0x03 << 8 | byte2

% byte1 = 00011000
% byte2 = 00011000

% 00011000 & 0x03 = 00000000
% \end{verbatim}

% bytes 3\ldots{}n-1 = variable length message

% bytes n\ldots{}n+2 = 24-bit CRC (CRC24)

% \subsection{RTCM v3 Message}\label{rtcm-v3-message}

% reserved : 1-bit message number : 12-bits uint12

% 18 d1 is\ldots{}

% 123456789012 0001100011010001

% \section{RTCM v3}\label{rtcm-v3-3}

% \subsection{GPS RTK Observable
% Messages}\label{gps-rtk-observable-messages}

% message header (aka number aka type): - 1001 - 1002 - 1003 - 1004

% \begin{longtable}[]{@{}llll@{}}
% \toprule
% DATA FIELD & DF NUMBER & DATA TYPE & NO. OF BITS\tabularnewline
% \midrule
% \endhead
% Message Number (e.g.,``1001''= 0011 1110 1001) & DF002 & uint12 &
% 12\tabularnewline
% Reference Station ID & DF003 & uint12 & 12\tabularnewline
% GPS Epoch Time (TOW) & DF004 & uint30 & 30\tabularnewline
% Synchronous GNSS Flag & DF005 & bit(1) & 1\tabularnewline
% No. of GPS Satellite Signals Processed & DF006 & uint5 &
% 5\tabularnewline
% GPS Divergence-free Smoothing Indicator & DF007 & bit(1) &
% 1\tabularnewline
% GPS Smoothing Interval & DF008 & bit(3) & 3\tabularnewline
% TOTAL & & & 64\tabularnewline
% \bottomrule
% \end{longtable}

% \subsection{Station Antenna Reference Point
% Messages}\label{station-antenna-reference-point-messages}

% \begin{itemize}
% \item
%   1005
% \item
%   1006
% \end{itemize}

% {\emph{1005}}\label{section}

% \begin{longtable}[]{@{}llll@{}}
% \toprule
% DATA FIELD & DF NUMBER & DATA TYPE & NO. OF BITS\tabularnewline
% \midrule
% \endhead
% Message Number (e.g. ``1005''= 0011 1110 1101) & DF002 & uint12 &
% 12\tabularnewline
% Reference Station ID & DF003 & uint12 & 12\tabularnewline
% \ldots{} & & &\tabularnewline
% TOTAL & & & 152\tabularnewline
% \bottomrule
% \end{longtable}

% \hypertarget{section-1}{{\emph{1006}}\label{section-1}}

% \begin{longtable}[]{@{}llll@{}}
% \toprule
% DATA FIELD & DF NUMBER & DATA TYPE & NO. OF BITS\tabularnewline
% \midrule
% \endhead
% Message Number (e.g. ``1006''= 0011 1110 1110) & DF002 & uint12 &
% 12\tabularnewline
% Reference Station ID & DF003 & uint12 & 12\tabularnewline
% \ldots{} & & &\tabularnewline
% TOTAL & & & 168\tabularnewline
% \bottomrule
% \end{longtable}

% \subsection{Antenna Description
% Messages}\label{antenna-description-messages}

% \begin{itemize}
% \item
%   1007
% \item
%   1008
% \end{itemize}

% {\emph{1007}}\label{section-2}

% \begin{longtable}[]{@{}llll@{}}
% \toprule
% DATA FIELD & DF NUMBER & DATA TYPE & NO. OF BITS\tabularnewline
% \midrule
% \endhead
% Message Number (e.g. ``1007''= 0011 1110 1111) & DF002 & uint12 &
% 12\tabularnewline
% Reference Station ID & DF003 & uint12 & 12\tabularnewline
% \ldots{} & & &\tabularnewline
% TOTAL & & & 40+8*N\tabularnewline
% \bottomrule
% \end{longtable}

% The Python RTCM3 library knows about the following messages: - 1001 -
% 1002 - 1003 - 1004 - 1005 - 1006 - 1008 - 1009 - 1010 - 1011 - 1012 -
% 1033

