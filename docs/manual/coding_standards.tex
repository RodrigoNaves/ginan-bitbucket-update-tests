\chapter{Coding Standards}
\label{ch:coding_standards}

\textit{Coding Standards} for C++

\section{Code style}
Decades of experience has shown that codebases that are built with concise, clean code have fewer issues and are easier to maintain. If submitting a pull request for a patch to the software, please ensure your code meets the following standards.

Overall we are aiming to
\begin{itemize}
	\item  Write for clarity
	\item  Write for clarity
	\item  Use short, descriptive variable names
	\item  Use aliases to reduce clutter.
\end{itemize}

\subsubsection{Unconcise code - Not recommended}
\begin{lstlisting}[language=C++]
//check first letter of satellite type against something

if (obs.Sat.id().c_str()[0]) == 'G') 
    doSomething(); 
else if (obs.Sat.id().c_str()[0]) == 'R')
    doSomething();
else if (obs.Sat.id().c_str()[0]) == 'E')
    doSomething();
else if (obs.Sat.id().c_str()[0]) == 'I')
    doSomething();
    
\end{lstlisting}

\subsubsection{Clear Code - Good}

\begin{lstlisting}[language=c++]
char& sysChar = obs.Sat.id().c_str()[0];

switch (sysChar)
{
    case 'G':   doSomething();   break;
    case 'R':   doSomething();   break;
    case 'E':   doSomething();   break;
    case 'I':   doSomething();   break;
}
\end{lstlisting}


\section{Spacing, Indentation, and layout}

\begin{itemize}
	\item  Use tabs, with tab spacing set to 4.	
	\item  Use space or tabs before and after any \begin{verbatim} + - * / = < > == != % etc.\end{verbatim}
	\item  Use space, tab or new line after any , ;
	\item  Use a new line after if statements.
	\item  Use tabs to keep things tidy - If the same function is called multiple times with different parameters, the parameters should line up.
\end{itemize}

\subsubsection{Scattered Parameters - Bad}

\begin{lstlisting}[language=c++]
trySetFromYaml(mongo_metadata,output_files,{"mongo_metadata" });
trySetFromYaml(mongo_output_measurements,output_files,{"mongo_output_measurements" });
trySetFromYaml(mongo_states,output_files,{"mongo_states" });
\end{lstlisting}

\subsubsection{Aligned Parameters - Good}
\begin{lstlisting}[language=c++]
trySetFromYaml(mongo_metadata,             output_files, {"mongo_metadata"              });
trySetFromYaml(mongo_output_measurements,  output_files, {"mongo_output_measurements"	});
trySetFromYaml(mongo_states,               output_files, {"mongo_states"		        });
\end{lstlisting}

\section{Statements}

One statement per line 
- * unless you have a very good reason

\subsubsection{Multiple Statements per Line - Bad}
\begin{lstlisting}[language=c++]
z[k]=ROUND(zb[k]); y=zb[k]-z[k]; step[k]=SGN(y);
\end{lstlisting}

\subsubsection{Single Statement per Line - Good}

\begin{lstlisting}[language=c++]
z[k]    = ROUND(zb[k]);
y       = zb[k]-z[k]; 
step[k] = SGN(y);
\end{lstlisting}

\subsubsection{Example of a good reason:}

* Multiple statements per line sometimes shows repetetive code more clearly, but put some spaces so the separation is clear.

\subsubsection{Normal}

\begin{lstlisting}[language=c++]
switch (sysChar)
{
    case ' ':
    case 'G': 
        *sys = E_Sys::GPS; 
        *tsys = TSYS_GPS; 
        break;
    case 'R': 
        *sys = E_Sys::GLO;  
        *tsys = TSYS_UTC; 
        break;
    case 'E': 
        *sys = E_Sys::GAL;  
        *tsys = TSYS_GAL; 
        break;
    //...continues
}
\end{lstlisting}

\subsubsection{Ok}

\begin{lstlisting}[language=c++]
if      (sys == SYS_GLO)    fact = EFACT_GLO;
else if (sys == SYS_CMP)    fact = EFACT_CMP;
else if (sys == SYS_GAL)    fact = EFACT_GAL;
else if (sys == SYS_SBS)    fact = EFACT_SBS;
else                        fact = EFACT_GPS;
\end{lstlisting}

\subsubsection{Ok}	

\begin{lstlisting}[language=c++]
switch (sysChar)
{
    case ' ':
    case 'G':   *sys = E_Sys::GPS;      *tsys = TSYS_GPS;    break;
    case 'R':   *sys = E_Sys::GLO;      *tsys = TSYS_UTC;    break;
    case 'E':   *sys = E_Sys::GAL;      *tsys = TSYS_GAL;    break;
    case 'S':   *sys = E_Sys::SBS;      *tsys = TSYS_GPS;    break;
    case 'J':   *sys = E_Sys::QZS;      *tsys = TSYS_QZS;    break;
//...continues
}
\end{lstlisting}

\section{Braces}

New line for braces.
\begin{lstlisting}[language=c++]
if (pass)
{
    doSomething();
}
\end{lstlisting}

\section{Comments}

\begin{itemize}
\item Prefer `//` for comments within functions
\item Use `/* */` only for temporary removal of blocks of code.
\item Use `/** */` and `///<` for automatic documentation
\end{itemize}

\section{Conditional checks}
%
\begin{itemize}
\item  Put `\&\&` and `||` at the beginning of lines when using multiple conditionals.
\item  Always use curly braces when using multiple conditionals.
\end{itemize}
%
\begin{lstlisting}[language=c++]
if  ( ( testA     > 10)
    &&( testB   == false
      ||testC   == false))
{
    //do something
}
\end{lstlisting}

* Use variables to name return values rather than using functions directly

\subsubsection{Bad}

\begin{lstlisting}[language=c++]
if (doSomeParsing(someObject))
{
    //code contingent on parsing success? failure?
}
\end{lstlisting}

\subsubsection{Good}
\begin{lstlisting}[language=c++]
bool fail = doSomeParsing(someObject);
if (fail)
{
    //This code is clearly a response to a failure
}
\end{lstlisting}

\section{Variable declaration}

\begin{itemize}
\item Declare variables as late as possible - at point of first use.
\item One declaration per line.
\item Declare loop counters in loops where possible.
\item Always initialise variables at declaration.
\end{itemize}

\begin{lstlisting}[language=c++]
int  type  = 0;
bool found = false;         //these have to be declared early so they can be used after the for loop

for (int i = 0; i < 10; i++)
{
    bool pass = someTestFunction();    //this pass variable isnt declared until it's used - good
    if (pass)
    {
        type  = typeMap[i];
        found = true;
        break;
    }
}

if (found)
{
    //...
}    
\end{lstlisting}

\section{Function parameters}

\begin{itemize}
\item One per line.
\item Add doxygen compatible documentation after parameters in the cpp file.
\item Prefer references rather than pointers unless unavoidable.
\end{itemize}

\begin{lstlisting}[language=c++]
void function(
        bool        runTests,           ///< Run unit test while processing
        MyStruct&   myStruct,           ///< Structure to modify
        OtherStr*	otherStr = nullptr)	///< Optional structure object to populate (cant use reference because its optional)
{
   	//...
}
\end{lstlisting}

\section{Naming and Structure}
%
\begin{itemize}
\item For structs or classes, use `CamelCase` with capital start
\item For member variables, use `camelCase` with lowercase start
\item For config parameters, use `lowercase\_with\_underscores`
\item Use suffixes (`\_ptr`, `\_arr`, `Map`, `List` etc.) to describe the type of container for complex types.
\item Be sure to provide default values for member variables.
\item Use heirarchical objects where applicable.
\end{itemize}
%
\begin{lstlisting}[language=c++]
struct SubStruct
{
    int    type = 0;
    double val  = 0;
};

struct MyStruct
{
    bool          memberVariable = false;
    double        precision      = 0.1;

    double                     offset_arr[10]  = {};
    OtherStruct*               refStruct_ptr   = nullptr;

    map<string, double>        offsetMap; 
    list<map<string, double>>  variationMapList;
    map<int, SubStruct>        subStructMap;
};

//...

MyStruct myStruct = {};

if (acsConfig.some_parameter)
{
    //..
}
\end{lstlisting}
\section{Testing}
\begin{itemize}
	\item Use TestStack objects at top of each function that requires automatic unit testing.
	\item Use TestStack objects with descriptive strings in loops that wrap functions that require automatic unit testing.
\end{itemize}
\begin{lstlisting}[language=c++]
void function()
{
    TestStack ts(__FUNCTION__);

    //...

    for (auto& obs : obsList)
    {
        TestStack ts(obs.Sat.id());

        //...
    }
}
\end{lstlisting}
\section{Documentation}
\begin{itemize}
\item Use doxygen style documentation for function and struct headers and parameters
\item `/**`  for headers.
\item `///<` for parameters
\end{itemize}
\begin{lstlisting}[language=c++]
/** Struct to demonstrate documentation.
* The first line automatically gets parsed as a brief description, but more detailed descriptions are possible too.
*/
struct MyStruct
{
    bool    dummyBool;                  ///< The thing to the left is documented here
};

/** Function to demonstrate documentation
*/
void function(
        bool        runTests,           ///< Run unit test while processing
        MyStruct&   myStruct,           ///< Structure to modify
        OtherStr*	otherStr = nullptr)	///< Optional string to populate
{
   	//...
}
\end{lstlisting}

\section{STL Templates}
\begin{itemize}
\item Prefer maps rather than fixed arrays.
\item Prefer range-based loops rather than iterators or `i` loops, unless unavoidable.
\end{itemize}

\subsubsection{Bad}
\begin{lstlisting}[language=c++]
double double_arr[10] = {};

//..(Populate array)

for (int i = 0; i < 10; i++)    //Magic number 10 - bad.
{

}
\end{lstlisting}
%
\begin{lstlisting}[language=c++]
map<string, double> doubleMap;

//..(Populate Map)

for (auto iter = doubleMap.begin(); iter != doubleMap.end(); iter++)   //long, undescriptive - bad
{
   	if (iter->first == someVar)     //'first' is undescriptive - bad
   	{
   		//..
   	}
}
\end{lstlisting}
\subsubsection{Good - Iterating Maps}
\begin{lstlisting}[language=c++]
map<string, double> offsetMap;

//..(Populate Map)

for (auto& [siteName, offset] : doubleMap)	//give readable names to map keys and values
{
    if (siteName.empty() == false)
    {
    
    }
}
\end{lstlisting}
\subsubsection{Good - Iterating Lists}
\begin{lstlisting}[language=c++]
list<Obs> obsList;

//..(Populate list)

for (auto& obs : obsList)         //give readable names to list elements
{
    doSomethingWithObs(obs);
}
\end{lstlisting}

\subsubsection{Special Case - Deleting from maps/lists}

Use iterators when you need to delete from STL containers:
\begin{lstlisting}[language=c++]
for (auto it = someMap.begin(); it != someMap.end();  )
{
    KFKey key = it->first;				//give some alias to the key/value so they're readable

    if (measuredStates[key] == false)
    {
    	it = someMap.erase(it);
   	}
 else
 {
    	++it;
   	}
}
\end{lstlisting}
\section{Namespaces}

Commonly used std containers may be included with `using`
\begin{lstlisting}[language=c++]
#include <string>
#include <map>
#include <list>
#include <unordered_map>

using std::string;
using std::map;
using std::list
using std::unordered_map;
\end{lstlisting}
