\chapter{Coding Standards}
\label{ch:coding_standards}

\newthought{Coding Standards} for C++

\section{Code style}
Overall we are aiming for
\begin{itemize}
	\item  Write for clarity
	\item  Write for clarity
	\item  Use short, descriptive variable names
	\item  Use aliases to reduce clutter.
\end{itemize}

\subsection{Bad}
\begin{verbatim}

```cpp

    //check first letter of satellite type against stomething

    if (obs.Sat.id().c_str()[0]) == 'G') 
        doSomething(); 
    else if (obs.Sat.id().c_str()[0]) == 'R')
        doSomething();
    else if (obs.Sat.id().c_str()[0]) == 'E')
        doSomething();
    else if (obs.Sat.id().c_str()[0]) == 'I')
        doSomething();
```
\end{verbatim}

\subsection{Good}

\begin{verbatim}
```cpp

    char& sysChar = obs.Sat.id().c_str()[0];

    switch (sysChar)
    {
        case 'G':   doSomething();   break;
        case 'R':   doSomething();   break;
        case 'E':   doSomething();   break;
        case 'I':   doSomething();   break;
    }
```
\end{verbatim}


\section{Spacing, Indentation, and layout}

\begin{itemize}
	\item  Use tabs, with tab spacing set to 4.	
	\item  Use space or tabs before and after any \begin{verbatim} + - * / = < > == != % etc.\end{verbatim}
	\item  Use space, tab or new line after any , ;
	\item  Use a new line after if statements.
	\item  Use tabs to keep things tidy - If the same function is called multiple times with different parameters, the parameters should line up.
\end{itemize}

\subsection{Bad}

\begin{verbatim}
```cpp

    trySetFromYaml(mongo_metadata,output_files,{"mongo_metadata" });
    trySetFromYaml(mongo_output_measurements,output_files,{"mongo_output_measurements" });
    trySetFromYaml(mongo_states,output_files,{"mongo_states" });
```
\end{verbatim}

\subsection{Good}
\begin{verbatim}
```cpp

    trySetFromYaml(mongo_metadata,             output_files, {"mongo_metadata"              });
    trySetFromYaml(mongo_output_measurements,  output_files, {"mongo_output_measurements"	});
    trySetFromYaml(mongo_states,               output_files, {"mongo_states"		        });
```
\end{verbatim}

\section{Statements}

* One statement per line  - \*unless you have a very good reason

\subsection{Bad}
/begin{verbatim}
```cpp

    z[k]=ROUND(zb[k]); y=zb[k]-z[k]; step[k]=SGN(y);
```
/end{verbatim}

\subsection{Good}

\begin{verbatim}
```cpp

    z[k]    = ROUND(zb[k]);
    y       = zb[k]-z[k]; 
    step[k] = SGN(y);
```
\end{verbatim}

\subsection{Example of a good reason:}

* Multiple statements per line sometimes shows repetetive code more clearly, but put some spaces so the separation is clear.

\subsection{Normal}

\begin{verbatim}
```cpp

    switch (sysChar)
    {
        case ' ':
        case 'G': 
            *sys = E_Sys::GPS; 
            *tsys = TSYS_GPS; 
            break;
        case 'R': 
            *sys = E_Sys::GLO;  
            *tsys = TSYS_UTC; 
            break;
        case 'E': 
            *sys = E_Sys::GAL;  
            *tsys = TSYS_GAL; 
            break;
    //...continues
```
\end{verbatim}

\subsection{Ok}

\begin{verbatim}
```cpp

    if      (sys == SYS_GLO)    fact = EFACT_GLO;
	else if (sys == SYS_CMP)    fact = EFACT_CMP;
	else if (sys == SYS_GAL)    fact = EFACT_GAL;
	else if (sys == SYS_SBS)    fact = EFACT_SBS;
	else                        fact = EFACT_GPS;
```
\end{verbatim}

\subsection{Ok}	

\begin{verbatim}
```cpp

    switch (sysChar)
    {
        case ' ':
        case 'G':   *sys = E_Sys::GPS;      *tsys = TSYS_GPS;    break;
        case 'R':   *sys = E_Sys::GLO;      *tsys = TSYS_UTC;    break;
        case 'E':   *sys = E_Sys::GAL;      *tsys = TSYS_GAL;    break;
        case 'S':   *sys = E_Sys::SBS;      *tsys = TSYS_GPS;    break;
        case 'J':   *sys = E_Sys::QZS;      *tsys = TSYS_QZS;    break;
    //...continues
```
\end{verbatim}

\section{Braces}

New line for braces.
\begin{verbatim}
```cpp

    if (pass)
    {
        doSomething();
    }
```
\end{verbatim}

\section{Comments}

%\begin{itemize}
%\item Prefer `//` for comments within functions
%\item Use `/* */` only for temporary removal of blocks of code.
%\item Use `/** */` and `///<` for automatic documentation
%\end{itemize}

\section{Conditional checks}
%\begin{itemize}
%\item  Put `\&\&` and `||` at the beginning of lines when using multiple conditionals.
%\item  Always use curly braces when using multiple conditionals.
%\end{itemize}
\begin{verbatim}
```cpp

    if  ( ( testA     > 10)
        &&( testB   == false
          ||testC   == false))
    {
        //do something
    }
```
\end{verbatim}

* Use variables to name return values rather than using functions directly

\subsection{Bad}

\begin{verbatim}
```cpp

    if (doSomeParsing(someObject))
    {
        //code contingent on parsing success? failure?
    }
```
\end{verbatim}

\subsection{Good}
\begin{verbatim}
```cpp

    bool fail = doSomeParsing(someObject);
    if (fail)
    {
        //This code is clearly a response to a failure
    }
```
\end{verbatim}

\section{Variable declaration}

\begin{itemize}
\item Declare variables as late as possible - at point of first use.
\item One declaration per line.
\item Declare loop counters in loops where possible.
\item Always initialise variables at declaration.
\end{itemize}

\begin{verbatim}
```cpp

    int  type  = 0;
    bool found = false;         //these have to be declared early so they can be used after the for loop

    for (int i = 0; i < 10; i++)
    {
        bool pass = someTestFunction();    //this pass variable isnt declared until it's used - good
        if (pass)
        {
            type  = typeMap[i];
            found = true;
            break;
        }
    }

    if (found)
    {
        //...
    }    
```
\end{verbatim}

\section{Function parameters}

\begin{itemize}
\item One per line.
\item Add doxygen compatible documentation after parameters in the cpp file.
\item Prefer references rather than pointers unless unavoidable.
\end{itemize}

\begin{verbatim}
```cpp

    void function(
            bool        runTests,           ///< Run unit test while processing
            MyStruct&   myStruct,           ///< Structure to modify
            OtherStr*	otherStr = nullptr)	///< Optional structure object to populate (cant use reference because its optional)
    {
    	//...
    }
```
\end{verbatim}

\section{Naming and Structure}
%\begin{itemize}
%\item For structs/classes, use `CamelCase` with capital start
%\item For member variables, use `camelCase` with lowercase start
%\item For config parameters, use `lowercase_with_underscores`
%\item Use suffixes (`_ptr`, `_arr`, `Map`, `List` etc.) to describe the type of container for complex types
%\item Be sure to provide default values for member variables.
%\item Use heirarchical objects where applicable.
%\end{itemize}
\begin{verbatim}
```cpp

    struct SubStruct
    {
        int    type = 0;
        double val  = 0;
    };

    struct MyStruct
    {
        bool          memberVariable = false;
        double        precision      = 0.1;

        double                     offset_arr[10]  = {};
        OtherStruct*               refStruct_ptr   = nullptr;

        map<string, double>        offsetMap; 
        list<map<string, double>>  variationMapList;
        map<int, SubStruct>        subStructMap;
    };

    //...

    MyStruct myStruct = {};

    if (acsConfig.some_parameter)
    {
        //..
    }
```
\end{verbatim}
\section{Testing}
\begin{itemize}
	\item Use TestStack objects at top of each function that requires automatic unit testing.
	\item Use TestStack objects with descriptive strings in loops that wrap functions that require automatic unit testing.
\end{itemize}
\begin{verbatim}
```cpp

    void function()
    {
        TestStack ts(__FUNCTION__);

        //...

        for (auto& obs : obsList)
        {
            TestStack ts(obs.Sat.id());

            //...
        }
    }
```
\end{verbatim}
\section{Documentation}
\begin{itemize}
\item Use doxygen style documentation for function and struct headers and parameters
\item `/**`  for headers.
\item `///<` for parameters
\end{itemize}
\begin{verbatim}
```cpp

    /** Struct to demonstrate documentation.
    * The first line automatically gets parsed as a brief description, but more detailed descriptions are possible too.
    */
    struct MyStruct
    {
        bool    dummyBool;                  ///< The thing to the left is documented here
    };

    /** Function to demonstrate documentation
    */
    void function(
            bool        runTests,           ///< Run unit test while processing
            MyStruct&   myStruct,           ///< Structure to modify
            OtherStr*	otherStr = nullptr)	///< Optional string to populate
    {
    	//...
    }
```
\end{verbatim}

\section{STL Templates}
\begin{itemize}
\item Prefer maps rather than fixed arrays.
\item Prefer range-based loops rather than iterators or `i` loops, unless unavoidable.
\end{itemize}

\subsection{Bad}
\begin{verbatim}

```cpp

    double double_arr[10] = {};

    //..(Populate array)

    for (int i = 0; i < 10; i++)    //Magic number 10 - bad.
    {

    }
```
\end{verbatim}
%
\begin{verbatim}
```cpp

    map<string, double> doubleMap;

    //..(Populate Map)

    for (auto iter = doubleMap.begin(); iter != doubleMap.end(); iter++)   //long, undescriptive - bad
    {
    	if (iter->first == someVar)     //'first' is undescriptive - bad
    	{
    		//..
    	}
    }
```
\end{verbatim}
\subsection{Good - Iterating Maps}
\begin{verbatim}
```cpp

    map<string, double> offsetMap;

    //..(Populate Map)

    for (auto& [siteName, offset] : doubleMap)	//give readable names to map keys and values
    {
        if (siteName.empty() == false)
        {
        
        }
    }
```
\end{verbatim}
\subsection{ Good - Iterating Lists}
\begin{verbatim}
```cpp

    list<Obs> obsList;

    //..(Populate list)

    for (auto& obs : obsList)         //give readable names to list elements
    {
        doSomethingWithObs(obs);
    }
```
\end{verbatim}

\section{Special Case - Deleting from maps/lists}

Use iterators when you need to delete from STL containers:
\begin{verbatim}
```cpp

    for (auto it = someMap.begin(); it != someMap.end();  )
    {
        KFKey key = it->first;				//give some alias to the key/value so they're readable

        if (measuredStates[key] == false)
        {
	    	it = someMap.erase(it);
    	}
	    else
	    {
	    	++it;
    	}
    }
```
\end{verbatim}
\section{Namespaces}

Commonly used std containers may be included with `using`
\begin{verbatim}
```cpp
    #include <string>
    #include <map>
    #include <list>
    #include <unordered_map>
    
    using std::string;
    using std::map;
    using std::list
    using std::unordered_map;
```
\end{verbatim}