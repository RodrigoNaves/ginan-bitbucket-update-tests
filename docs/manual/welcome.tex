\chapter{Welcome}
\label{ch:Ginan}

% We would like to thank the Wardaman people for permission in the use of their word Ginan. 

Ginan is the fifth-brightest star in the Southern Cross (Epsilon Crucis) . It represents a red dilly-bag filled with special songs of knowledge.
Indigenous Australians often used songs to convey and to pass on knowledge to others, song were also often used as a way to navigate the country.

We hope that you find this software tool kit will convey our understanding on how to process GNSS signals and will also help you to navigate the country!

the story Ginan was found by Mulugurnden (the crayfish), who brought the red flying foxes from the underworld to the sky. The bats flew up the track of the Milky Way and traded the spiritual song to Guyaru, the Night Owl (the star Sirius). The bats fly through the constellation Scorpius on their way to the Southern Cross, trading songs as they go.

The song informs the people about initiation, which is managed by the stars in Scorpius and related to Larawag (who ensures the appropriate personnel are present for the final stages of the ceremony).

The brownish-red colour of the dilly bag is represented by the colour of Epsilon Crucis, which is an orange giant that lies 228 light years away.