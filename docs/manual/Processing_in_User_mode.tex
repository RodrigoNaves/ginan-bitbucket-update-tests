\chapter{Processing in Precise Point Positioning mode}

\textit{Precise Point Positioning}(PPP)

\section{Pea PPP Processing examples}
In this example we will process 24 hours of data from a permanent reference frame station. The algorithm that will be use an L1+L2 and L1+L5 ionosphere free combination. The log files and processing results can be found in /data/acs/pea/output/exs/EX01\_IF/.

\begin{lstlisting}[language=bash]
$ ./pea --config ../../config/Ex01-IF-PPP.yaml
$ grep "\$POS" /data/acs/pea/output/exs/EX01_IF/EX01_IF-ALIC.TRACE
\end{lstlisting}

And you should see the following: 
\begin{lstlisting}[language=bash] 
$POS,2062,431940.000,0,-4052052.7956,4212836.0144,-2545104.6423,...
$POS,2062,431970.000,0,-4052052.7956,4212836.0144,-2545104.6423,...
\end{lstlisting}


\subsection{Using the Ionosphere-free observable to process a static data set - the float solution}

In this example we will process 24 hours of data from a permanent reference frame station. The algorithm will use an L1+L2 and L1+L5 ionosphere-free combination.
The log files and processing results can be found in `<path to pea>/output/exs/EX01\_IF/`.
\begin{lstlisting}
$ ./pea --config ../../config/EX01-IF-PPP.yaml
\end{lstlisting}
The pea will then have the following output in <path to pea>/output/exs/EX01\_IF/ :

EX01\_IF20624.snx              - contains the station position estimates in SINEX format
EX01\_IF-ALIC2019199900.TRACE  - contains the logging information from the processing run

\begin{lstlisting}[language=bash]
$ grep "REC_POS" /data/acs/pea/output/exs/EX01_IF/EX01_IF-ALIC201919900.TRACE > ALIC_201919900.PPP
\end{lstlisting}
This will pipe all of the receiver position results reported in the station trace file to a seperate file for plotting.
\begin{lstlisting}[language=bash]
$ python3 /data/acs/pea/python/source/pppPlot.py --ppp /data/acs/pea/output/exs/EX01_IF/ALIC_201919900.PPP
\end{lstlisting}
This will then create the plots alic\_pos.png, a time series of the difference between the estimated receiver position and the median estimated position.
And the plot alic\_snx\_pos.png, a time series of the difference between the estimated receiver position and the IGS SINEX solution for Alic Springs on this day.

\subsection{Single Frequency Processing} 
\begin{lstlisting}[language=bash]
$ ./pea --config ../../config/Ex01-SF-PPP.yaml

$ grep "\$POS" /data/acs/pea/output/exs/EX01_SF/EX01_SF-ALBY202011500.TRACE
\end{lstlisting}
And you should see something similar to the following:
\begin{lstlisting}[language=bash]
$POS,2062,431940.000,0,-4052052.7956,4212836.0144,-2545104.6423,0.00000043966020,...
$POS,2062,431970.000,0,-4052052.7956,4212836.0144,-2545104.6423,0.00000043965772,...
\end{lstlisting}



\subsection{Processing realtime}
To process a continuous GPS station in real-time you will need to access the data stream from a NTRIP stream and the correction products from a NTRIPCaster.
Geoscience Australia is running a caster that provides global data stream, a dense network of stream covering the Australian region, and correction procdust provide by IGS Analysis Centres.
You will need to apply for an AUSCORS account, or use the new NTRIPCaster that streams using https.

You can apply for an account at the following link : \href{https://gnss-users-prod.auth.ap-southeast-2.amazoncognito.com/login?response_type=code&client_id=11njl767q0tl1faf9qna469vl1&redirect_uri=https://search-gnss-elasticsearch-prod-5omhch5quzlu5dcpbct4ev5qz4.ap-southeast-2.es.amazonaws.com/_plugin/kibana/app/kibana&state=e36b2054-7ace-4931-91a3-5ba6de893917}{New GA Account link}

Once your have your account and password you can test that you can successfuly connect to the NTRIPCaster by using the following curl command:
\begin{lstlisting}[language=bash]
$ curl https://ntrip.data.gnss.ga.gov.au/MOBS00AUS0 --http0.9 -i -u'<username>:<password> --output -'
\end{lstlisting}

