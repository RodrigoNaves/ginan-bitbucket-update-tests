\chapter{Using PEA in network mode}
\label{ch:pea_network_mode}
The Parameter Estimation Algorithm (PEA) is designed estimate the GNSS error parameters that cannot be precisely estimated. The PEA will estimate the following parameters:
\begin{itemize}
	\item Correction to satellite initial conditions estimated by the POD component
	\item Satellite clock offset and drift
	\item Satellite hardware bias for two signal carriers
	\item Satellite differential bias for two signal pseudoranges 
	\item Ionospheric propagation delay
	\item Tropospheric propagation delay
	\item Receiver/station position and velocity
	\item Receiver/station clock offset and drift
	\item Receiver/station hardware biase for signal carriers
	\item Receiver/station differential bias for two signal pseudoranges 
	\item Relative carrier phase ambiguities
\end{itemize}  
In order to estimate the full range of parameters, the PEA will need to ingest GNSS observation data form a Global network of proper density. Receiver position and clock, Tropospheric delays can also be estimated from a single receiver and thus will be addressed on chapter \ref{ch:pea_user_mode}.\\

As is the case for the POD, the PEA use YAML formatted configuration files to set the processing options, and is run using the command:
\begin{lstlisting}
$ ./pea --config <path_to_config_file>
\end{lstlisting}
Details on the configuration parameters included in YAML files can be found in chapter \ref{ch:pea_yaml_configuration}.
With one exception (Ionosphere delay modelling using smoothed pseudorange), processing of network data is activated by setting the \textit{ processing\_options : process\_modes : network} to \textit{true}.
Configuration files corresponding to the examples in this section can be found in the \textit{ginan/examples} directory. A few of these examples are exaplained bellow.\\

\section{Processing a Global Network to Adjust Satellite Positions}
The is designed to use integrated/predicted satellite positions for their real-time network  processing mode, thus all satellite position correction can olny be made in post process mode.
A basic example for PEA network processing is provided in  \textit{ex17\_pea\_pp\_netw\_gnss\_ar.yaml}. 
In order to activate estimation of satellite position corrections the entry /textit{default\_filter\_parameters : satellites : orb : estimated} neets to be set to \textit{true}.
It will also require an ICF file, generated by POD in orbit fitting mode, as input. The path to such file should be given in \textit{input\_files : orbfiles} entry.
ANTEX files, with antenna information for both stations and satellites should be provided in  \textit{input\_files : atxfiles} 
SINEX files, with station antennas should be provided in  \textit{input\_files : snxfiles}
RINEX 3.XX navigation files, with broadcast clocks should be provided in  \textit{input\_files : navfiles}
BLQ formatted ocean tide  loading parameters for each station   should be provided in \textit{input\_files : blqfiles} if available to correct for OTL, otherwise \textit{processing\_options : tide\_otl} should be set to \textit{false}.
RINEX 3.XX observation files for the network stations should be provided in \textit{station\_data : rnxfiles} (* can be used as a wildcard).\\

The main output of this processing mode would be an ICF formatted file containing the corrected initial conditions for each of the processed satellites. 
The file will be created in the same path as the input ICF file with the \textit{\_pea} suffix attached. 
This file can be used by the POD to integrate/predict the satellite positions over the required time window.\\

Receiver/station positions can also be estimated as part of the network processing. 
In order to estimate station positions, the parameter \textit{default\_filter\_parameters : stations : pos : estimated} and \textit{output\_files : output\_sinex} needs to be set to \textit{true}
A SINEX formatted file with the estimated station position will be generated in thepath specified as \textit{root\_output\_directory}.\\

\section{Post process estimation of Satellite clocks and biases}
The example configuration file \textit{ex17\_pea\_pp\_netw\_gnss\_ar.yaml} corresponds to a post-mission network processing mode for GPS satellite clocks.
The required input files are similar to the satellite position estimation example.
The main difference will be that SP3 formatted files can be used as input for satellite position (POD generated ICF files can also be used).
The frequency in which the GNSS error parameters, including satellite and receiver clocks, are estimated should be set by the parameter \textit{processing\_options : epoch\_interval}. 
In order to output the estimated clocks, the parameter  \textit{output\_files : output\_clocks} needs to be set to \textit{true}.
The satellite and receiver will then be output to a RINEX clock formatted file specified in \textit{output\_files : clock\_filename}.\\
 
In order to estimate the satellite and receiver hardware biases, the ambiguities need to be separated from the biases and resolved to integer values. 
In order to perform ambiguity resolution, the proper parameters needs to be set on the \textit{ambiguity\_resolution\_options} fields. 
The target constellations need to be selected by setting  \textit{GPS\_amb\_resol} and/or \textit{GAL\_amb\_resol} to true (only GPS and GAL constellation are supported in the current version). 
Both \textit{WL\_mode} and \textit{NL\_mode} needs to be set to something different than \textit{off}. 
It is advised that \textit{round} or \textit{iter\_rnd} be used for network processing. 
In order to output the estimated biases the parameter  \textit{output\_files : output\_biasSINEX} needs to be to \textit{true} and \textit{ambiguity\_resolution\_options : bias\_output\_rate} set to a number (of seconds) different than zero.
The satellite biases will then be output to a bias SINEX formatted file specified in \textit{output\_files : biasSINEX\_filename}.
Receiver biases are also estimated by the process, however they are not reflected on the output files.\\

\section{Real-time estimation of Satellite clocks and biases}
An example of using PEA for real-time estimation of GPS satellite clocks is provided in \textit{ex17\_pea\_rt\_netw\_gnss\_ar.yaml}.
The configuration file is similar to that used for post-mission processing.
The main difference is that the input data specified in the \textit{station\_data} field will correspond to RTCM formatted streams instead of files.
Currently the PEA can only get real-time data by connecting to an NTRIP caster.
The host name, user name and password corresponding to the NTRIP can shold be specified under \textit{station\_data : stream\_root} using the format \textit{http(s)://user:password@hostname/}.
The mountpoint corresponding to station observables need to be listed under \textit{station\_data : obs\_streams}.
Ephemeris streams (broadcast ephemeris and SSR corrections) shold be listed under \textit{station\_data : nav\_streams}.
Alternatively the mountpoints can be specified using the full path \textit{http(s)://user:password@hostname/mountpoint}, leaving the  \textit{station\_data : stream\_root} field empty,  this allows to use streams from multiple NTRIP casters.
Satellite positions needs to be provided using (predicted) SP3 files or using real-time streams (broadcast + SSR corrections).\\

In the \textit{processing\_options} field, the \textit{ppp\_ephemeris} parameter needs to be set to \textit{precise} is using SP3 files and \textit{ssr\_apc} or \textit{ssr\_com} if using SSR correction streams.
In the same field, the \textit{epoch\_interval} is used to set the update interval of the network solutions, and the \textit{wait\_next\_epoch} and \textit{wait\_all\_stations} to help syncronise station streams.
The PEA will wait for  \textit{wait\_next\_epoch} seconds from the start of the previous epoc for the first observation to arrive (and skip the current epoc if no observations arrive).
The PEA will wait for \textit{wait\_all\_stations} seconds from the first observation for data from other stations before processing.\\

The estimated clock and bias can be output to an NTRIP caster (as well as to local files). 
The output NTRIP caster streams need to be specified using the \textit{output\_streams} field.
The host name, user name and password can be set in the \textit{output\_streams : stream\_root} parameter.
The names for the output streams should be listed under \textit{output\_streams : stream\_label}
Once the label is created, the mountpoint and RTCM messages to encode can be specified in the \textit{output\_streams : label} field.
Currently the PEA supports output for GPS and Galileo orbits and clock messages (1060 and 1243), code bias (1059 and 1243) and phase bias (1265 and 1267).\\ 

\section{Post process estimation of atmospheric delays}
The PEA is capable of estimation both Tropospheric and Ionospheric delays on GNSS signals.
Tropospheric delays can be estimated both in network (\textit{processing\_options : process\_modes : network = true}) and end-user (\textit{processing\_options : process\_modes : user = true}) processing mode and will be addressed in the next chapter.
Ionosphere delay estimation and mapping is activated by setting \textit{processing\_options : process\_modes : ionosphere} to \textit{true}.
Two types of Ionosphere delay estimates are supported by PEA. 
If all other parameters in the \textit{processing\_options : process\_modes} field are set to \textit{false} then the Ionospheric delay are estimated based on carrier smoothed pseudoranges.
If in addition to \textit{processing\_options : process\_modes : ionosphere}, \textit{processing\_options : process\_modes : network} is set to true, the PEA will attempt to calculate Ionospheric delay measurements from ambiguity resolved carrier phase measurements. For this mode to work, the parameters in the \textit{ambiguity\_resolution\_options} field needs to be set properly.
Ionospheric delay estimate are availabe for GPS signals only.\\

The Ionosphere slant delay measurements delay measurments can be outputted into STEC files if \textit{output\_files : output\_ionstec} parameter is set to \textit{true} (the output file name can be set using \textit{ionstec\_filename}).
The format of STEC files have one of two forms. If the vector in \textit{ionosphere\_filter\_parameters : layer\_heights} is not empty, each line on the STEC file will contain the slant delay measurements and the piercing points at each layer height:
 \begin{lstlisting}
#IONO_MEA, 2102,171000.000, AGGO, G05, -1.6030, 1.9699e-04, 2, 1, 350, -33.662, -61.955, 1.443
\end{lstlisting}
the fields representing, from left to right:
\begin{enumerate}
	\item  "IONO\_MEA" label
	\item  GPS week
	\item  GPS TOW in seconds
	\item  Receiver/station name
	\item  Satellite ID
	\item  Slant delay in meters
	\item  Slant delay variance in meters$^2$
	\item  Number of Ionosphere layers N
	\item  N fields containing: 
	\begin{enumerate}
		\item Height of layer in Km
		\item Latitude of piercing point (in degrees)
		\item Longitude of piercing moint (in degrees)
		\item Slant to vertical mapping function
	\end {enumerate}  
\end{enumerate}
If the layer heights field is empty the "Number of Ionosphere layers" field will be 0 and folloed by the receiver position in ECEF, and the satellite position in ECEF.\\

Vertial TEC (VTEC) maps can be estimated from the slant delay measurments and output as IONEX formatted maps ( \textit{output\_files : output\_ionex = true}) and its corresponding DCB ( \textit{output\_files : output\_biasSINEX = true}).
Ionosphere mapping and output are controlled by parameters in the \textit{ionosphere\_filter\_parameters} field.
Currently only spherical harmonics based mapping is supported by the PEA \textit{model = spherical\_harmonic}, setting  \textit{ionosphere\_filter\_parameters : model} to \textit{meas\_out} will output the ionosphere measurments but will not perform mapping. 
If spherical harmonics is selected as mapping method, the ionospheric delays will be mapped into myltiple thin layer shells.
The heigh of the shells can be set in the \textit{ionosphere\_filter\_parameters : layer\_heights} vector
The VTEC at each layer will be fit to spherical harmonic comonents, with a maximum order and degree of  \textit{ionosphere\_filter\_parameters : func\_order}.
If  \textit{output\_files : output\_ionex} is set to \textit{true} the Ionosphere map will be outputtted in IONEX 1.11 format.
The area of the IONEX map can be set using the \textit{lat\_center}, \textit{lon\_center},  \textit{lat\_width}, \textit{lon\_width} parameters.
The horizontal resolution of the IONEX map can be set using the \textit{lat\_res}, \textit{lon\_res} parameters.
The temporal resolution of the IONEX file is defined by the \textit{time\_res} parameter.\\  

A configuration file, \textit{ex16\_pea\_pp\_ionosphere.yaml} can be used to generate a single layer IONEX map and acompanning biasSINEX file from smoothed pseudorange observations.\\
