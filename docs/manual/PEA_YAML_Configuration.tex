\chapter{PEA YAML Configuration}
\label{ch:pea_yaml_configuration}
\section{PEA processing options (pod\_options)}
%
%
\section{Input File Options}
\begin{lstlisting}[language=xml,caption=yaml input files configuration example]
input_files:

root_input_directory: /data/acs/pea/proc/exs/products

atxfiles:      [  igs14_2045_plus.atx                       ]  # Antenna models
snxfiles:      [  igs19P2062.snx                            ]  # meta data and apriori coords
blqfiles:      [  OLOAD_GO.BLQ                              ]  # ocean loading is applied
navfiles:      [  brdm1990.19p                              ]  # gnss broadcast file
sp3files:      [  gag20624.sp3                              ]  # precise orbit data 
erpfiles:      [  igs19P2062.erp                            ]  # earth orintation parameters
#dcbfiles:      [  CAS0MGXRAP_20191990000_01D_01D_DCB.BSX   ]  # monthly DCB file
#clkfiles:     [  jpl20624.clk                              ]  # satellie and receiver clock 
orbfiles:      [  gag20624_orbits_partials.out              ]  # need this when estimating orbits (overrides .sp3 file)
\end{lstlisting}

\section{RINEX station data}

%RINEX2 and RINEX3 data
There are numerous ways that the \emph{pea} can access GNSS RINEX observations to process. You can specify individual rinex files to process, set it up so that it will search a particular directory, or you can use a command line flag \emph{-rnx <rnxfilename>} to add an additional file to process. The data should be uncompressed (gunzipped, and not in hatanaka format), that is the \emph{pea} expecting to just accept RINEX format]. 
%
\begin{lstlisting}[language=xml,caption=pea yaml processing one station example]
station_data:

	root_stations_directory: /data/ginarn/proc/data

	rnxfiles:
		- ALIC00AUS_R_20191990000_01D_30S_MO.rnx                          
\end{lstlisting}\label{lst:pea-yaml-single-station}
%
To process one RINEX file, you need to first specify the root dircetory of where the data is being stored in \emph{root\_stations\_directory}, and then the name of the RINEX file as a single entry under \emph{rnxfiles:}, as shown in the listing~ref{lst:pea-yaml-single-station}. 
%
\begin{lstlisting}[language=bash,caption=Example showing how to add a RINEX file to the processing list form the command line]
$ pea --rnx CAS100ATA_R_20191990000_01D_30S_MO.rnx
\end{lstlisting}\label{lst:pea-cmd-single-station}
%
If you wanted to process another file, located in the same \emph{root\_stations\_directory}, this can be achieved at the command line using the \emph{-rnx <rnxfilename} flag, see listing~ref{lst:pea-yaml-add-station}.
%
\begin{lstlisting}[language=xml,caption=yaml input files configuration example]
station_data:

root_stations_directory: /data/acs/pea/proc/exs/data

rnxfiles:
- "*.rnx"                           #- searching all in file_root directory
\end{lstlisting}

\section{Real-time streams}
To process data in real-time you will need to set up the location, username annd password for the caster that you will be obtaining the input data streams from in the configuration file.

The pea supports obtaining streams from casters that use NTRIP 2.0 over http and https. 
%TODO check I believe we can do multiple casters for input At the moment the pea is able to connect to one caster for input data streams, and a different caster for the output of correction messages.
\begin{lstlisting}[language=xml,caption=yaml input files configuration example]
station_data:

	stream_root: "http://<usedname>:<password>@auscors.ga.gov.au:2101/"

	streams:
		- BCEP00BKG0
		- SSRA00CNE0
		- STR100AUS0
\end{lstlisting}\label{lst:pea-yaml-realtime-single-station}
%
As shown in listing:~\ref{lst:pea-yaml-realtime-single-station}, the caster url, username and password are specified within double quotes with the \emph{stream\_root} tag. In this example the streams are being obtained from the auscors caster run by Geoscience Australia. 
The broadcast information is being obtained from the stream \emph{BCEP00BKG0} being supplied by BKG, and corrections to the utlra-rapid predicted orbit are being obtained from the stream \emph{SSRA00CNE0}. 
The real-time data being processed is for the continuous GNSS station located at Mount Stromlo obtained from the stream \emph{STR100AUS0}.
%
\section{output files}
\begin{lstlisting}[language=xml,caption=yaml input files configuration example]
output_files:

root_output_directory:          /data/acs/pea/output/<CONFIG>/

log_level:                      warn                             #debug, info, warn, error as defined in boost::log

output_trace:                   true
trace_level:                    2
trace_directory:                ./
trace_filename:                 <CONFIG>-<STATION><YYYY><DDD><HH>.TRACE

output_residuals:               true

output_persistance:             false
input_persistance:              false
persistance_directory:          ./
persistance_filename:           <CONFIG>.persist

output_config:                  false

output_summary:                 true
summary_directory:              ./
summary_filename:               PEA<YYYY><DDD><HH>.SUM

output_ionex:                   false
ionex_directory:                ./
ionex_filename:                 IONEX.ionex
iondsb_filename:                IONEX.iondcb

output_clocks:                  true
clocks_directory:               ./
clocks_filename:                <CONFIG>.clk
output_AR_clocks:               true

output_sinex:                   true
sinex_directory:                ./
\end{lstlisting}

\section{output options}

\begin{lstlisting}[language=xml,caption=yaml input files configuration example]
output_options:

config_description:             EX03_AR
analysis_agency:                GAA
analysis_center:                Geoscience Australia
analysis_program:               AUSACS
rinex_comment:                  AUSNETWORK1
\end{lstlisting}

\section{processing Options}

\begin{lstlisting}[language=xml,caption=yaml input files configuration example]
processing_options:

#start_epoch:               2019-07-18 00:00:00
#end_epoch:                 2017-03-29 23:59:30
#max_epochs:                300        #0 is infinite
epoch_interval:             30          #seconds

process_modes:
user:                   false
network:                true
minimum_constraints:    false
rts:                    false
ionosphere:             false
unit_tests:             false

process_sys:
gps:            true
#glo:           true
gal:            false
#bds:           true

elevation_mask:     10   #degrees

tide_solid:         true
tide_pole:          true
tide_otl:           true

phase_windup:       true
reject_eclipse:     true            #  reject observation during satellite eclipse periods
raim:               true
antexacs:           true

cycle_slip:
thres_slip: 0.05

max_inno:   0
max_gdop:   30

troposphere:
model:      gpt2    #vmf3
gpt2grid:   gpt_25.grd
#vmf3dir:    grid5/
#orography:  orography_ell_5x5

ionosphere:
corr_mode:      iono_free_linear_combo
iflc_freqs:     l1l2_only   #any l1l2_only l1l5_only

pivot_station:        "USN7"
#pivot_satellite:      "G01"

code_priorities: [  L1C, L1P, L1Y, L1W, L1M, L1N, L1S, L1L,
L2W, L2P, L2Y, L2C, L2M, L2N, L2D, L2S, L2L, L2X,
L5I, L5Q, L5X]
\end{lstlisting}

\section{Network filter parameters}
\begin{lstlisting}[language=xml,caption=yaml input files configuration example]
network_filter_parameters:

process_mode:               kalman      #lsq
inverter:                   LLT         #LLT LDLT INV

max_filter_iterations:      3
max_filter_removals:        3

rts_lag:                    -1      #-ve for full reverse, +ve for limited epochs
rts_directory:              ./
rts_filename:               <CONFIG>-Orbits.rts
\end{lstlisting}

\section{Default filter parameters: stations}

\begin{lstlisting}[language=xml,caption=yaml input files configuration example]
default_filter_parameters:

	stations:	
		error_model:        elevation_dependent         #uniform elevation_dependent
			code_sigmas:        [0.30]
			phase_sigmas:       [0.003]
		pos:
			estimated:          false
			sigma:              [1.0]
			proc_noise:         [0]
			#apriori:                                   # taken from other source, rinex file etc.
			#frame:              xyz #ned
			#proc_noise_model:   Gaussian
			#clamp_max:          [+0.5]
			#clamp_min:          [-0.5]
		clk:
			estimated:          true
			sigma:              [0]
			proc_noise:         [1.8257418583505538]
			#proc_noise:         [10]
			#proc_noise_model:   Gaussian
		clk_rate:
			estimated:          false
			sigma:              [10]
			proc_noise:         [1e-4]
			#clamp_max:          [1000]
			#clamp_min:          [-1000]
		amb:
			estimated:          true
			sigma:              [100]
			proc_noise:         [0]
		trop:
			estimated:          true
			sigma:              [0.1]
			proc_noise:         [0.000083333]
		trop_grads:
			estimated:          false
			sigma:              [0.1]
			proc_noise:         [0.0000036]
\end{lstlisting}

\section{Default filter parameters: satellites}

\begin{lstlisting}[language=xml,caption=yaml input files configuration example]
    satellites:

		clk:
			estimated:          true
			sigma:              [0]
			proc_noise:         [0.03651483716701108]
		
		clk_rate:
			estimated:          false
			sigma:              [0.01]
			proc_noise:         [1e-6]
		
		orb:
			estimated:          true
			sigma:              [5e-1, 5e-1, 5e-1, 5e-3, 5e-3, 5e-3, 5e-1]
\end{lstlisting}

\section{Default filter parameters: eop}

\begin{lstlisting}[language=xml,caption=yaml input files configuration example]
	eop:
		estimated:  true
		sigma:      [30]
		#proc_noise: [0.0000036]

\end{lstlisting}


\section{Ambiguity Resolution Options}
%
\begin{lstlisting}[language=xml,caption=yaml ambiguity configuration example]
	ambiguity_resolution_options:
		Min_elev_for_AR:            15.0
		GPS_amb_resol:              true
		GAL_amb_resol:              false
		#Set_size_for_lambda:        10

	WL_mode:                    iter_rnd        # AR mode for WL: off, round, iter_rnd, bootst, lambda, lambda_alt, lambda_al2, lambda_bie
		WL_succ_rate_thres:         0.9999
		WL_sol_ratio_thres:         3.0
		WL_procs_noise_sat:         0.00001
		WL_procs_noise_sta:         0.0001

	NL_mode:                    iter_rnd        # AR mode for WL: off, round, iter_rnd, bootst, lambda, lambda_alt, lambda_al2, lambda_bie
		NL_succ_rate_thres:         0.9999
		NL_sol_ratio_thres:         3.0
		NL_proc_start:              86310

	bias_read_mode:             30              # +1: read DSB biases, +2: read OSB biases, +4: read code biases, +8: read phase biases, +16: read satellite bias, +32: read station bias,
		bias_output_rate:           300.0
\end{lstlisting}
